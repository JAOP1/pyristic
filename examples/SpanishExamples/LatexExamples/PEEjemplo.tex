
% Default to the notebook output style

    


% Inherit from the specified cell style.




    
\documentclass[11pt]{article}

    
    
    \usepackage[T1]{fontenc}
    % Nicer default font (+ math font) than Computer Modern for most use cases
    \usepackage{mathpazo}

    % Basic figure setup, for now with no caption control since it's done
    % automatically by Pandoc (which extracts ![](path) syntax from Markdown).
    \usepackage{graphicx}
    % We will generate all images so they have a width \maxwidth. This means
    % that they will get their normal width if they fit onto the page, but
    % are scaled down if they would overflow the margins.
    \makeatletter
    \def\maxwidth{\ifdim\Gin@nat@width>\linewidth\linewidth
    \else\Gin@nat@width\fi}
    \makeatother
    \let\Oldincludegraphics\includegraphics
    % Set max figure width to be 80% of text width, for now hardcoded.
    \renewcommand{\includegraphics}[1]{\Oldincludegraphics[width=.8\maxwidth]{#1}}
    % Ensure that by default, figures have no caption (until we provide a
    % proper Figure object with a Caption API and a way to capture that
    % in the conversion process - todo).
    \usepackage{caption}
    \DeclareCaptionLabelFormat{nolabel}{}
    \captionsetup{labelformat=nolabel}

    \usepackage{adjustbox} % Used to constrain images to a maximum size 
    \usepackage{xcolor} % Allow colors to be defined
    \usepackage{enumerate} % Needed for markdown enumerations to work
    \usepackage{geometry} % Used to adjust the document margins
    \usepackage{amsmath} % Equations
    \usepackage{amssymb} % Equations
    \usepackage{textcomp} % defines textquotesingle
    % Hack from http://tex.stackexchange.com/a/47451/13684:
    \AtBeginDocument{%
        \def\PYZsq{\textquotesingle}% Upright quotes in Pygmentized code
    }
    \usepackage{upquote} % Upright quotes for verbatim code
    \usepackage{eurosym} % defines \euro
    \usepackage[mathletters]{ucs} % Extended unicode (utf-8) support
    \usepackage[utf8x]{inputenc} % Allow utf-8 characters in the tex document
    \usepackage{fancyvrb} % verbatim replacement that allows latex
    \usepackage{grffile} % extends the file name processing of package graphics 
                         % to support a larger range 
    % The hyperref package gives us a pdf with properly built
    % internal navigation ('pdf bookmarks' for the table of contents,
    % internal cross-reference links, web links for URLs, etc.)
    \usepackage{hyperref}
    \usepackage{longtable} % longtable support required by pandoc >1.10
    \usepackage{booktabs}  % table support for pandoc > 1.12.2
    \usepackage[inline]{enumitem} % IRkernel/repr support (it uses the enumerate* environment)
    \usepackage[normalem]{ulem} % ulem is needed to support strikethroughs (\sout)
                                % normalem makes italics be italics, not underlines
    \usepackage{mathrsfs}
    

    
    
    % Colors for the hyperref package
    \definecolor{urlcolor}{rgb}{0,.145,.698}
    \definecolor{linkcolor}{rgb}{.71,0.21,0.01}
    \definecolor{citecolor}{rgb}{.12,.54,.11}

    % ANSI colors
    \definecolor{ansi-black}{HTML}{3E424D}
    \definecolor{ansi-black-intense}{HTML}{282C36}
    \definecolor{ansi-red}{HTML}{E75C58}
    \definecolor{ansi-red-intense}{HTML}{B22B31}
    \definecolor{ansi-green}{HTML}{00A250}
    \definecolor{ansi-green-intense}{HTML}{007427}
    \definecolor{ansi-yellow}{HTML}{DDB62B}
    \definecolor{ansi-yellow-intense}{HTML}{B27D12}
    \definecolor{ansi-blue}{HTML}{208FFB}
    \definecolor{ansi-blue-intense}{HTML}{0065CA}
    \definecolor{ansi-magenta}{HTML}{D160C4}
    \definecolor{ansi-magenta-intense}{HTML}{A03196}
    \definecolor{ansi-cyan}{HTML}{60C6C8}
    \definecolor{ansi-cyan-intense}{HTML}{258F8F}
    \definecolor{ansi-white}{HTML}{C5C1B4}
    \definecolor{ansi-white-intense}{HTML}{A1A6B2}
    \definecolor{ansi-default-inverse-fg}{HTML}{FFFFFF}
    \definecolor{ansi-default-inverse-bg}{HTML}{000000}

    % commands and environments needed by pandoc snippets
    % extracted from the output of `pandoc -s`
    \providecommand{\tightlist}{%
      \setlength{\itemsep}{0pt}\setlength{\parskip}{0pt}}
    \DefineVerbatimEnvironment{Highlighting}{Verbatim}{commandchars=\\\{\}}
    % Add ',fontsize=\small' for more characters per line
    \newenvironment{Shaded}{}{}
    \newcommand{\KeywordTok}[1]{\textcolor[rgb]{0.00,0.44,0.13}{\textbf{{#1}}}}
    \newcommand{\DataTypeTok}[1]{\textcolor[rgb]{0.56,0.13,0.00}{{#1}}}
    \newcommand{\DecValTok}[1]{\textcolor[rgb]{0.25,0.63,0.44}{{#1}}}
    \newcommand{\BaseNTok}[1]{\textcolor[rgb]{0.25,0.63,0.44}{{#1}}}
    \newcommand{\FloatTok}[1]{\textcolor[rgb]{0.25,0.63,0.44}{{#1}}}
    \newcommand{\CharTok}[1]{\textcolor[rgb]{0.25,0.44,0.63}{{#1}}}
    \newcommand{\StringTok}[1]{\textcolor[rgb]{0.25,0.44,0.63}{{#1}}}
    \newcommand{\CommentTok}[1]{\textcolor[rgb]{0.38,0.63,0.69}{\textit{{#1}}}}
    \newcommand{\OtherTok}[1]{\textcolor[rgb]{0.00,0.44,0.13}{{#1}}}
    \newcommand{\AlertTok}[1]{\textcolor[rgb]{1.00,0.00,0.00}{\textbf{{#1}}}}
    \newcommand{\FunctionTok}[1]{\textcolor[rgb]{0.02,0.16,0.49}{{#1}}}
    \newcommand{\RegionMarkerTok}[1]{{#1}}
    \newcommand{\ErrorTok}[1]{\textcolor[rgb]{1.00,0.00,0.00}{\textbf{{#1}}}}
    \newcommand{\NormalTok}[1]{{#1}}
    
    % Additional commands for more recent versions of Pandoc
    \newcommand{\ConstantTok}[1]{\textcolor[rgb]{0.53,0.00,0.00}{{#1}}}
    \newcommand{\SpecialCharTok}[1]{\textcolor[rgb]{0.25,0.44,0.63}{{#1}}}
    \newcommand{\VerbatimStringTok}[1]{\textcolor[rgb]{0.25,0.44,0.63}{{#1}}}
    \newcommand{\SpecialStringTok}[1]{\textcolor[rgb]{0.73,0.40,0.53}{{#1}}}
    \newcommand{\ImportTok}[1]{{#1}}
    \newcommand{\DocumentationTok}[1]{\textcolor[rgb]{0.73,0.13,0.13}{\textit{{#1}}}}
    \newcommand{\AnnotationTok}[1]{\textcolor[rgb]{0.38,0.63,0.69}{\textbf{\textit{{#1}}}}}
    \newcommand{\CommentVarTok}[1]{\textcolor[rgb]{0.38,0.63,0.69}{\textbf{\textit{{#1}}}}}
    \newcommand{\VariableTok}[1]{\textcolor[rgb]{0.10,0.09,0.49}{{#1}}}
    \newcommand{\ControlFlowTok}[1]{\textcolor[rgb]{0.00,0.44,0.13}{\textbf{{#1}}}}
    \newcommand{\OperatorTok}[1]{\textcolor[rgb]{0.40,0.40,0.40}{{#1}}}
    \newcommand{\BuiltInTok}[1]{{#1}}
    \newcommand{\ExtensionTok}[1]{{#1}}
    \newcommand{\PreprocessorTok}[1]{\textcolor[rgb]{0.74,0.48,0.00}{{#1}}}
    \newcommand{\AttributeTok}[1]{\textcolor[rgb]{0.49,0.56,0.16}{{#1}}}
    \newcommand{\InformationTok}[1]{\textcolor[rgb]{0.38,0.63,0.69}{\textbf{\textit{{#1}}}}}
    \newcommand{\WarningTok}[1]{\textcolor[rgb]{0.38,0.63,0.69}{\textbf{\textit{{#1}}}}}
    
    
    % Define a nice break command that doesn't care if a line doesn't already
    % exist.
    \def\br{\hspace*{\fill} \\* }
    % Math Jax compatibility definitions
    \def\gt{>}
    \def\lt{<}
    \let\Oldtex\TeX
    \let\Oldlatex\LaTeX
    \renewcommand{\TeX}{\textrm{\Oldtex}}
    \renewcommand{\LaTeX}{\textrm{\Oldlatex}}
    % Document parameters
    % Document title
    \title{PEEjemplo}
    
    
    
    
    

    % Pygments definitions
    
\makeatletter
\def\PY@reset{\let\PY@it=\relax \let\PY@bf=\relax%
    \let\PY@ul=\relax \let\PY@tc=\relax%
    \let\PY@bc=\relax \let\PY@ff=\relax}
\def\PY@tok#1{\csname PY@tok@#1\endcsname}
\def\PY@toks#1+{\ifx\relax#1\empty\else%
    \PY@tok{#1}\expandafter\PY@toks\fi}
\def\PY@do#1{\PY@bc{\PY@tc{\PY@ul{%
    \PY@it{\PY@bf{\PY@ff{#1}}}}}}}
\def\PY#1#2{\PY@reset\PY@toks#1+\relax+\PY@do{#2}}

\expandafter\def\csname PY@tok@w\endcsname{\def\PY@tc##1{\textcolor[rgb]{0.73,0.73,0.73}{##1}}}
\expandafter\def\csname PY@tok@c\endcsname{\let\PY@it=\textit\def\PY@tc##1{\textcolor[rgb]{0.25,0.50,0.50}{##1}}}
\expandafter\def\csname PY@tok@cp\endcsname{\def\PY@tc##1{\textcolor[rgb]{0.74,0.48,0.00}{##1}}}
\expandafter\def\csname PY@tok@k\endcsname{\let\PY@bf=\textbf\def\PY@tc##1{\textcolor[rgb]{0.00,0.50,0.00}{##1}}}
\expandafter\def\csname PY@tok@kp\endcsname{\def\PY@tc##1{\textcolor[rgb]{0.00,0.50,0.00}{##1}}}
\expandafter\def\csname PY@tok@kt\endcsname{\def\PY@tc##1{\textcolor[rgb]{0.69,0.00,0.25}{##1}}}
\expandafter\def\csname PY@tok@o\endcsname{\def\PY@tc##1{\textcolor[rgb]{0.40,0.40,0.40}{##1}}}
\expandafter\def\csname PY@tok@ow\endcsname{\let\PY@bf=\textbf\def\PY@tc##1{\textcolor[rgb]{0.67,0.13,1.00}{##1}}}
\expandafter\def\csname PY@tok@nb\endcsname{\def\PY@tc##1{\textcolor[rgb]{0.00,0.50,0.00}{##1}}}
\expandafter\def\csname PY@tok@nf\endcsname{\def\PY@tc##1{\textcolor[rgb]{0.00,0.00,1.00}{##1}}}
\expandafter\def\csname PY@tok@nc\endcsname{\let\PY@bf=\textbf\def\PY@tc##1{\textcolor[rgb]{0.00,0.00,1.00}{##1}}}
\expandafter\def\csname PY@tok@nn\endcsname{\let\PY@bf=\textbf\def\PY@tc##1{\textcolor[rgb]{0.00,0.00,1.00}{##1}}}
\expandafter\def\csname PY@tok@ne\endcsname{\let\PY@bf=\textbf\def\PY@tc##1{\textcolor[rgb]{0.82,0.25,0.23}{##1}}}
\expandafter\def\csname PY@tok@nv\endcsname{\def\PY@tc##1{\textcolor[rgb]{0.10,0.09,0.49}{##1}}}
\expandafter\def\csname PY@tok@no\endcsname{\def\PY@tc##1{\textcolor[rgb]{0.53,0.00,0.00}{##1}}}
\expandafter\def\csname PY@tok@nl\endcsname{\def\PY@tc##1{\textcolor[rgb]{0.63,0.63,0.00}{##1}}}
\expandafter\def\csname PY@tok@ni\endcsname{\let\PY@bf=\textbf\def\PY@tc##1{\textcolor[rgb]{0.60,0.60,0.60}{##1}}}
\expandafter\def\csname PY@tok@na\endcsname{\def\PY@tc##1{\textcolor[rgb]{0.49,0.56,0.16}{##1}}}
\expandafter\def\csname PY@tok@nt\endcsname{\let\PY@bf=\textbf\def\PY@tc##1{\textcolor[rgb]{0.00,0.50,0.00}{##1}}}
\expandafter\def\csname PY@tok@nd\endcsname{\def\PY@tc##1{\textcolor[rgb]{0.67,0.13,1.00}{##1}}}
\expandafter\def\csname PY@tok@s\endcsname{\def\PY@tc##1{\textcolor[rgb]{0.73,0.13,0.13}{##1}}}
\expandafter\def\csname PY@tok@sd\endcsname{\let\PY@it=\textit\def\PY@tc##1{\textcolor[rgb]{0.73,0.13,0.13}{##1}}}
\expandafter\def\csname PY@tok@si\endcsname{\let\PY@bf=\textbf\def\PY@tc##1{\textcolor[rgb]{0.73,0.40,0.53}{##1}}}
\expandafter\def\csname PY@tok@se\endcsname{\let\PY@bf=\textbf\def\PY@tc##1{\textcolor[rgb]{0.73,0.40,0.13}{##1}}}
\expandafter\def\csname PY@tok@sr\endcsname{\def\PY@tc##1{\textcolor[rgb]{0.73,0.40,0.53}{##1}}}
\expandafter\def\csname PY@tok@ss\endcsname{\def\PY@tc##1{\textcolor[rgb]{0.10,0.09,0.49}{##1}}}
\expandafter\def\csname PY@tok@sx\endcsname{\def\PY@tc##1{\textcolor[rgb]{0.00,0.50,0.00}{##1}}}
\expandafter\def\csname PY@tok@m\endcsname{\def\PY@tc##1{\textcolor[rgb]{0.40,0.40,0.40}{##1}}}
\expandafter\def\csname PY@tok@gh\endcsname{\let\PY@bf=\textbf\def\PY@tc##1{\textcolor[rgb]{0.00,0.00,0.50}{##1}}}
\expandafter\def\csname PY@tok@gu\endcsname{\let\PY@bf=\textbf\def\PY@tc##1{\textcolor[rgb]{0.50,0.00,0.50}{##1}}}
\expandafter\def\csname PY@tok@gd\endcsname{\def\PY@tc##1{\textcolor[rgb]{0.63,0.00,0.00}{##1}}}
\expandafter\def\csname PY@tok@gi\endcsname{\def\PY@tc##1{\textcolor[rgb]{0.00,0.63,0.00}{##1}}}
\expandafter\def\csname PY@tok@gr\endcsname{\def\PY@tc##1{\textcolor[rgb]{1.00,0.00,0.00}{##1}}}
\expandafter\def\csname PY@tok@ge\endcsname{\let\PY@it=\textit}
\expandafter\def\csname PY@tok@gs\endcsname{\let\PY@bf=\textbf}
\expandafter\def\csname PY@tok@gp\endcsname{\let\PY@bf=\textbf\def\PY@tc##1{\textcolor[rgb]{0.00,0.00,0.50}{##1}}}
\expandafter\def\csname PY@tok@go\endcsname{\def\PY@tc##1{\textcolor[rgb]{0.53,0.53,0.53}{##1}}}
\expandafter\def\csname PY@tok@gt\endcsname{\def\PY@tc##1{\textcolor[rgb]{0.00,0.27,0.87}{##1}}}
\expandafter\def\csname PY@tok@err\endcsname{\def\PY@bc##1{\setlength{\fboxsep}{0pt}\fcolorbox[rgb]{1.00,0.00,0.00}{1,1,1}{\strut ##1}}}
\expandafter\def\csname PY@tok@kc\endcsname{\let\PY@bf=\textbf\def\PY@tc##1{\textcolor[rgb]{0.00,0.50,0.00}{##1}}}
\expandafter\def\csname PY@tok@kd\endcsname{\let\PY@bf=\textbf\def\PY@tc##1{\textcolor[rgb]{0.00,0.50,0.00}{##1}}}
\expandafter\def\csname PY@tok@kn\endcsname{\let\PY@bf=\textbf\def\PY@tc##1{\textcolor[rgb]{0.00,0.50,0.00}{##1}}}
\expandafter\def\csname PY@tok@kr\endcsname{\let\PY@bf=\textbf\def\PY@tc##1{\textcolor[rgb]{0.00,0.50,0.00}{##1}}}
\expandafter\def\csname PY@tok@bp\endcsname{\def\PY@tc##1{\textcolor[rgb]{0.00,0.50,0.00}{##1}}}
\expandafter\def\csname PY@tok@fm\endcsname{\def\PY@tc##1{\textcolor[rgb]{0.00,0.00,1.00}{##1}}}
\expandafter\def\csname PY@tok@vc\endcsname{\def\PY@tc##1{\textcolor[rgb]{0.10,0.09,0.49}{##1}}}
\expandafter\def\csname PY@tok@vg\endcsname{\def\PY@tc##1{\textcolor[rgb]{0.10,0.09,0.49}{##1}}}
\expandafter\def\csname PY@tok@vi\endcsname{\def\PY@tc##1{\textcolor[rgb]{0.10,0.09,0.49}{##1}}}
\expandafter\def\csname PY@tok@vm\endcsname{\def\PY@tc##1{\textcolor[rgb]{0.10,0.09,0.49}{##1}}}
\expandafter\def\csname PY@tok@sa\endcsname{\def\PY@tc##1{\textcolor[rgb]{0.73,0.13,0.13}{##1}}}
\expandafter\def\csname PY@tok@sb\endcsname{\def\PY@tc##1{\textcolor[rgb]{0.73,0.13,0.13}{##1}}}
\expandafter\def\csname PY@tok@sc\endcsname{\def\PY@tc##1{\textcolor[rgb]{0.73,0.13,0.13}{##1}}}
\expandafter\def\csname PY@tok@dl\endcsname{\def\PY@tc##1{\textcolor[rgb]{0.73,0.13,0.13}{##1}}}
\expandafter\def\csname PY@tok@s2\endcsname{\def\PY@tc##1{\textcolor[rgb]{0.73,0.13,0.13}{##1}}}
\expandafter\def\csname PY@tok@sh\endcsname{\def\PY@tc##1{\textcolor[rgb]{0.73,0.13,0.13}{##1}}}
\expandafter\def\csname PY@tok@s1\endcsname{\def\PY@tc##1{\textcolor[rgb]{0.73,0.13,0.13}{##1}}}
\expandafter\def\csname PY@tok@mb\endcsname{\def\PY@tc##1{\textcolor[rgb]{0.40,0.40,0.40}{##1}}}
\expandafter\def\csname PY@tok@mf\endcsname{\def\PY@tc##1{\textcolor[rgb]{0.40,0.40,0.40}{##1}}}
\expandafter\def\csname PY@tok@mh\endcsname{\def\PY@tc##1{\textcolor[rgb]{0.40,0.40,0.40}{##1}}}
\expandafter\def\csname PY@tok@mi\endcsname{\def\PY@tc##1{\textcolor[rgb]{0.40,0.40,0.40}{##1}}}
\expandafter\def\csname PY@tok@il\endcsname{\def\PY@tc##1{\textcolor[rgb]{0.40,0.40,0.40}{##1}}}
\expandafter\def\csname PY@tok@mo\endcsname{\def\PY@tc##1{\textcolor[rgb]{0.40,0.40,0.40}{##1}}}
\expandafter\def\csname PY@tok@ch\endcsname{\let\PY@it=\textit\def\PY@tc##1{\textcolor[rgb]{0.25,0.50,0.50}{##1}}}
\expandafter\def\csname PY@tok@cm\endcsname{\let\PY@it=\textit\def\PY@tc##1{\textcolor[rgb]{0.25,0.50,0.50}{##1}}}
\expandafter\def\csname PY@tok@cpf\endcsname{\let\PY@it=\textit\def\PY@tc##1{\textcolor[rgb]{0.25,0.50,0.50}{##1}}}
\expandafter\def\csname PY@tok@c1\endcsname{\let\PY@it=\textit\def\PY@tc##1{\textcolor[rgb]{0.25,0.50,0.50}{##1}}}
\expandafter\def\csname PY@tok@cs\endcsname{\let\PY@it=\textit\def\PY@tc##1{\textcolor[rgb]{0.25,0.50,0.50}{##1}}}

\def\PYZbs{\char`\\}
\def\PYZus{\char`\_}
\def\PYZob{\char`\{}
\def\PYZcb{\char`\}}
\def\PYZca{\char`\^}
\def\PYZam{\char`\&}
\def\PYZlt{\char`\<}
\def\PYZgt{\char`\>}
\def\PYZsh{\char`\#}
\def\PYZpc{\char`\%}
\def\PYZdl{\char`\$}
\def\PYZhy{\char`\-}
\def\PYZsq{\char`\'}
\def\PYZdq{\char`\"}
\def\PYZti{\char`\~}
% for compatibility with earlier versions
\def\PYZat{@}
\def\PYZlb{[}
\def\PYZrb{]}
\makeatother


    % Exact colors from NB
    \definecolor{incolor}{rgb}{0.0, 0.0, 0.5}
    \definecolor{outcolor}{rgb}{0.545, 0.0, 0.0}



    
    % Prevent overflowing lines due to hard-to-break entities
    \sloppy 
    % Setup hyperref package
    \hypersetup{
      breaklinks=true,  % so long urls are correctly broken across lines
      colorlinks=true,
      urlcolor=urlcolor,
      linkcolor=linkcolor,
      citecolor=citecolor,
      }
    % Slightly bigger margins than the latex defaults
    
    \geometry{verbose,tmargin=1in,bmargin=1in,lmargin=1in,rmargin=1in}
    
    

    \begin{document}
    
    
    \maketitle
    
    

    
    \section{Programación Evolutiva}\label{programaciuxf3n-evolutiva}

La librería \textbf{Pyristic} incluye una clase llamada
\texttt{EvolutionaryProgramming} inspirada en la metaheurística de
\emph{Programación Evolutiva} (PE) para resolver problemas de
minimización. Para trabajar con esta clase se requiere hacer lo
siguiente:

\begin{enumerate}
\def\labelenumi{\arabic{enumi}.}
\item
  Definir:

  \begin{itemize}
  \tightlist
  \item
    La función objetivo \(f\).
  \item
    La lista de restricciones.
  \item
    Lista de límites inferiores y superiores.
  \item
    Configuración de operadores de la metaheurística (opcional).
  \end{itemize}
\item
  Crear una clase que hereda de \texttt{EvolutionaryProgramming}.
\item
  Sobreescribir las siguientes funciones:

  \begin{itemize}
  \tightlist
  \item
    mutation\_operator (opcional)
  \item
    adaptive\_mutation (opcional)
  \item
    survivor\_selection (opcional)
  \item
    initialize\_step\_weights (opcional)
  \item
    initialize\_population (opcional)
  \item
    fixer (opcional)
  \end{itemize}
\end{enumerate}

A continuación se muestran los elementos que se deben importar.

    \begin{Verbatim}[commandchars=\\\{\}]
{\color{incolor}In [{\color{incolor}1}]:} \PY{k+kn}{import} \PY{n+nn}{sys}
        \PY{k+kn}{import} \PY{n+nn}{os}
        
        \PY{c+c1}{\PYZsh{}library\PYZus{}path is the path where the Optimpy library is located.}
        \PY{n}{library\PYZus{}path} \PY{o}{=} \PY{l+s+s2}{\PYZdq{}}\PY{l+s+s2}{/home/dell/Documentos/Git\PYZus{}proejcts/optimizacion\PYZhy{}con\PYZhy{}metaheuristicas/}\PY{l+s+s2}{\PYZdq{}}
        \PY{c+c1}{\PYZsh{}library\PYZus{}path = \PYZdq{}/Users/adrianamenchacamendez/Documentos/enes\PYZus{}morelia/papime/optimizacion\PYZhy{}con\PYZhy{}metaheuristicas/\PYZdq{}}
        \PY{n}{sys}\PY{o}{.}\PY{n}{path}\PY{o}{.}\PY{n}{append}\PY{p}{(}\PY{n}{os}\PY{o}{.}\PY{n}{path}\PY{o}{.}\PY{n}{abspath}\PY{p}{(}\PY{n}{library\PYZus{}path}\PY{p}{)}\PY{p}{)}
\end{Verbatim}

    \subsubsection{Librerías externas}\label{libreruxedas-externas}

    \begin{Verbatim}[commandchars=\\\{\}]
{\color{incolor}In [{\color{incolor}2}]:} \PY{k+kn}{from} \PY{n+nn}{pprint} \PY{k}{import} \PY{n}{pprint}
        \PY{k+kn}{import} \PY{n+nn}{math}
        \PY{k+kn}{import} \PY{n+nn}{numpy} \PY{k}{as} \PY{n+nn}{np} 
        \PY{k+kn}{import} \PY{n+nn}{copy}
\end{Verbatim}

    \begin{Verbatim}[commandchars=\\\{\}]
{\color{incolor}In [{\color{incolor}4}]:} \PY{k+kn}{from} \PY{n+nn}{IPython}\PY{n+nn}{.}\PY{n+nn}{display} \PY{k}{import} \PY{n}{Image}
        \PY{k+kn}{from} \PY{n+nn}{IPython}\PY{n+nn}{.}\PY{n+nn}{core}\PY{n+nn}{.}\PY{n+nn}{display} \PY{k}{import} \PY{n}{HTML} 
\end{Verbatim}

    \subsubsection{\texorpdfstring{Componentes de
\texttt{pyristic}}{Componentes de pyristic}}\label{componentes-de-pyristic}

La estructura que está organizada la librería es: * Las metaheurísticas
están ubicadas en \texttt{heuristic}. * Las funciones de prueba están
ubicadas en \texttt{utils.test\_function}. * Las clases auxiliares para
mantener la información de los operadores que serán empleados para
alguna de las metaheurísticas basadas en los paradigmas del cómputo
evolutivo están ubicadas en \texttt{utils.helpers}. * Las
metaheurísticas basadas en los paradigmas del cómputo evolutivo dependen
de un conjunto de operadores (selección, mutación y cruza). Estos
operadores están ubicados en \texttt{utils.operators}.

Para demostrar el uso de nuestra metaheurística basada en
\emph{algoritmos geneticos} tenemos que importar la clase llamada
\texttt{Genetic} que se encuentra en
\texttt{heuristic.GeneticAlgorithm\_search}.

    \begin{Verbatim}[commandchars=\\\{\}]
{\color{incolor}In [{\color{incolor}3}]:} \PY{k+kn}{from} \PY{n+nn}{optimpy}\PY{n+nn}{.}\PY{n+nn}{heuristic}\PY{n+nn}{.}\PY{n+nn}{EvolutiveProgramming\PYZus{}search} \PY{k}{import} \PY{n}{EvolutionaryProgramming}
        \PY{k+kn}{from} \PY{n+nn}{optimpy}\PY{n+nn}{.}\PY{n+nn}{utils}\PY{n+nn}{.}\PY{n+nn}{helpers} \PY{k}{import} \PY{n}{EvolutionaryProgrammingConfig}\PY{p}{,} \PY{n}{get\PYZus{}stats}\PY{p}{,} \PY{n}{ContinuosFixer}
        \PY{k+kn}{from} \PY{n+nn}{optimpy}\PY{n+nn}{.}\PY{n+nn}{utils}\PY{n+nn}{.}\PY{n+nn}{test\PYZus{}function} \PY{k}{import} \PY{n}{beale\PYZus{}}\PY{p}{,} \PY{n}{ackley\PYZus{}}
        \PY{k+kn}{from} \PY{n+nn}{optimpy}\PY{n+nn}{.}\PY{n+nn}{utils}\PY{n+nn}{.}\PY{n+nn}{operators} \PY{k}{import} \PY{n}{mutation}\PY{p}{,} \PY{n}{selection}
\end{Verbatim}

    \subsection{\texorpdfstring{Clase
\texttt{EvolutionaryProgramming}}{Clase EvolutionaryProgramming}}\label{clase-evolutionaryprogramming}

\paragraph{Variables}\label{variables}

\begin{itemize}
\item
  \emph{\textbf{logger.}} Diccionario con información relacionada a la
  búsqueda con las siguientes llaves:

  \begin{itemize}
  \tightlist
  \item
    \texttt{best\_individual.} Mejor individuo encontrado.
  \item
    \texttt{best\_f.} El vlor obtenido de la función objetivo evaluado
    en el individuo almacenado en individual.
  \item
    \texttt{current\_iter.} Iteración actual de la búsqueda.
  \item
    \texttt{total\_iter.} Número total de iteraciones.
  \item
    \texttt{parent\_population\_x.} Arreglo bidimensional de numpy que
    representa cada fila a un individuo de la población, mientras, las
    columnas representan el número de variables de decisión.
  \item
    \texttt{offspring\_population\_x.} Arreglo bidimensional de numpy
    que representa cada fila a un individuo de la población, mientras,
    las columnas representan el número de variables de decisión.
  \item
    \texttt{parent\_population\_sigma.} Arreglo de numpy que representa
    el desplazamiento de por variable de decisión de cada uno de los
    individuos.
  \item
    \texttt{offspring\_population\_sigma.} Arreglo de numpy que
    representa el desplazamiento de por variable de decisión de cada uno
    de los individuos.
  \item
    \texttt{parent\_population\_f.} Arreglo de numpy que contiene el
    valor de la función objetivo para cada uno de los individuos de la
    población de parent\_population\_x.
  \item
    \texttt{offspring\_population\_f.} Arreglo de numpy que contiene el
    valor de la función objetivo para cada uno de los individuos de la
    población de offspring\_population\_x.
  \end{itemize}
\item
  \emph{\textbf{f.}} Función objetivo.
\item
  \emph{\textbf{Constraints.}} Lista de restricciones del problema. Las
  restricciones deben ser funciones que retornan True o False, indicando
  si cumple dicha restricción.
\item
  \emph{\textbf{Bounds.}} Representa los límites definidos para cada una
  de las variables del problema. Se aceptan las siguientes
  representaciones:

  \begin{itemize}
  \tightlist
  \item
    Arreglo de numpy con solo dos componentes numéricas, donde, la
    primera componente es el límite inferior y la segunda componente es
    el límite superior. Esto significa que todas las variables de
    decisión estarán definidas para el mismo intervalo.
  \item
    Arreglo bidimensional de numpy con dos arreglos de numpy, donde, el
    primer arreglo de numpy representa el límite inferior para cada
    variable de decisión, mientras, la segunda componente representa el
    límite superior para cada variable de decisión.
  \end{itemize}
\item
  \emph{\textbf{Decision\_variables.}} El número de variables de
  decisión del problema.
\end{itemize}

\paragraph{Métodos}\label{muxe9todos}

\begin{itemize}
\tightlist
\item
  \emph{\textbf{\_\_init\_\_.}} Constructor de la clase.
\end{itemize}

Argumentos: * \texttt{function.} Función objetivo. *
\texttt{decision\_variables.} Número que indica las variables de
decisión del problema. * \texttt{constraints.} Lista con las
restricciones del problema. * \texttt{bounds.} Límites del espacio de
búsqueda de cada una de las variables de decisión del problema. *
\texttt{config.} Estructura de datos
(\texttt{EvolutionaryProgrammingConfig}) con los operadores que se
emplearán en la búsqueda.

Valor de retorno: * Ninguno.

\begin{itemize}
\tightlist
\item
  \emph{\textbf{optimize.}} método principal, realiza la ejecución de la
  metaheurística.
\end{itemize}

Argumentos: * \texttt{generations.} Número de generaciones (iteraciones
de la metaheurística). * \texttt{size\_population.} Tamaño de la
población (número de individuos). * \texttt{verbose.} Indica si se
imprime en qué iteración se encuentra nuestra búsqueda. Por defecto,
está en True. * \texttt{**kwargs.} Diccionario con argumentos externos a
la búsqueda. Estos argumentos pueden ser empleados cuando se
sobreescribe alguno de los métodos que tiene la clase.

Valor de retorno: * Ninguno.

\begin{itemize}
\tightlist
\item
  \emph{\textbf{mutatio\_operator.}} Muta las variables de decisión que
  se encuentran almacenadas en el diccionario \texttt{logger} con la
  llave \texttt{parent\_population\_x}.
\end{itemize}

Argumentos: * \texttt{**kwargs} Diccionario con argumentos externos a la
búsqueda. Estos argumentos pueden ser empleados cuando se sobreescribe
alguno de los métodos que tiene la clase.

Valor de retorno: * Un arreglo bidimensional de \emph{numpy}
representado a los nuevos individuos (se almacenarán en \texttt{logger}
con la llave \texttt{offspring\_population\_x}).

\begin{itemize}
\tightlist
\item
  \emph{\textbf{adaptive\_mutation.}} Muta los tamaños de paso que se
  encuentran almacenados en el diccionario \texttt{logger} con la llave
  \texttt{parent\_population\_sigma}.
\end{itemize}

Argumentos: * \texttt{**kwargs.} Diccionario con argumentos externos a
la búsqueda. Estos argumentos pueden ser empleados cuando se
sobreescribe alguno de los métodos que tiene la clase.

Valor de retorno: * Un arreglo bidimensional de numpy con los tamaños de
desplazamiento para cada una de las variables de decisión de los
individuos (se almacenarán en \texttt{logger} con la llave
\texttt{offspring\_population\_sigma}).

\begin{itemize}
\tightlist
\item
  \emph{\textbf{survivor\_selection.}} Selección de los individuos que
  pasarán a la siguiente generación.
\end{itemize}

Argumentos: * \texttt{**kwargs.} Diccionario con argumentos externos a
la búsqueda. Estos argumentos pueden ser empleados cuando se
sobreescribe alguno de los métodos que tiene la clase.

Valor de retorno: * Un diccionario con las siguientes llaves: *
\texttt{parent\_population\_f.} El valor de la función objetivo de cada
individuo que pasará a la siguiente generación. *
\texttt{parent\_population\_sigma.} El/los valor(es) de desplazamiento
de los individuos seleccionados. * \texttt{parent\_population\_x.} el
vector \(\vec{x}\) de cada uno de los individuos.

Por defecto la metaheurística utiliza el esquema de selección
\((\mu + \lambda)\) que se encuentra en utils.operators.selection con el
nombre de merge\_selector.

\begin{itemize}
\tightlist
\item
  \emph{\textbf{initialize\_population.}} Crea una población de
  individuos aleatorios. Para ello se utiliza una distribución uniforme
  y se generan números aleatorios dentro de los límites indicados para
  cada variable. Los individuos generados son almacenados en
  \texttt{logger} con la llave \texttt{parent\_population\_x}. Esta
  función es llamada dentro de la función optimize.
\end{itemize}

Argumentos: * \texttt{**kwargs.} Diccionario con argumentos externos a
la búsqueda. Estos argumentos pueden ser empleados cuando se
sobreescribe alguno de los métodos que tiene la clase.

Valor de retorno: * Un arreglo bidimensional de \emph{numpy}. El número
de filas es igual al tamaño de la población y el número de columnas es
igual al número de variables que tiene el problema que se está
resolviendo.

\begin{itemize}
\tightlist
\item
  \emph{\textbf{initialize\_step\_weights.}} Inicializa el tamaño de
  desplazamiento de cada una de las variables de decisión pertenecientes
  a cada individuo de la población. Para ello se generan números
  aleatorios en el intervalo \([0,1]\), utilizando una distribución
  uniforme. Los tamaños de desplazamiento están almacenados en
  \texttt{logger} con la llave \texttt{parent\_population\_sigma}.
\end{itemize}

Argumentos: * \texttt{**kwargs.} Diccionario con argumentos externos a
la búsqueda. Estos argumentos pueden ser empleados cuando se
sobreescribe alguno de los métodos que tiene la clase.

Valor de retorno: * Un arreglo bidimensional de numpy. El número de
filas es igual al tamaño de la población y el número de columnas es
igual al número de variables que tiene el problema que se está
resolviendo. Cada variable tiene su propio tamaño de paso.

\begin{itemize}
\tightlist
\item
  \emph{\textbf{fixer.}} Si la solución no está dentro de los límites
  definidos para cada variable (restricciones de caja), actualiza el
  valor de la variable con el valor del límite que rebasó. De lo
  contrario, regresa la misma solución.
\end{itemize}

Argumentos: * \texttt{ind.} Índice del individuo.

Valor de retorno: * Un arreglo de \emph{numpy} que reemplazará la
solución infactible de la población con la llave
\texttt{offspring\_population\_x}.

    \subsection{\texorpdfstring{Clase
\texttt{EvolutionaryProgrammingConfig}}{Clase EvolutionaryProgrammingConfig}}\label{clase-evolutionaryprogrammingconfig}

    \paragraph{Variables}\label{variables}

\begin{itemize}
\tightlist
\item
  \emph{\textbf{mutation\_op.}} Variable con el operador de mutación.
\item
  \emph{\textbf{survivor\_selector.}} Variable con el esquema de
  selección de los individuos que pasan a la siguiente generación.
\item
  \emph{\textbf{fixer.}} Variable con una función auxiliar para los
  individuos que no cumplen las restricciones del problema.
\item
  \emph{\textbf{adaptive\_mutation\_op.}} Variable con el operador de
  mutación de los tamaños de paso \(\sigma\).
\end{itemize}

\paragraph{Métodos}\label{muxe9todos}

\begin{itemize}
\tightlist
\item
  \emph{\textbf{mutate.}} Actualiza el operador de mutación de la
  variable \texttt{mutation\_op}.
\end{itemize}

Argumentos: * \texttt{mutate\_.} Función o clase que realiza la mutación
de la población almacenada con la llave
\texttt{offspring\_population\_x}.

Valor de retorno: * Retorna la configuración con la actualización del
operador de cruza. El objetivo es poder aplicar varios operadores en
cascada.

\begin{itemize}
\tightlist
\item
  \emph{\textbf{survivor\_selection.}} Actualiza el esquema de selección
  de la variable \texttt{survivor\_selector}.
\end{itemize}

Argumentos: * \texttt{survivor\_function.}Función o clase que realiza la
selección de individuos para la próxima generación.

Valor de retorno: * Retorna la configuración con la actualización del
operador de cruza. El objetivo es poder aplicar varios operadores en
cascada.

\begin{itemize}
\tightlist
\item
  \emph{\textbf{fixer\_invalide\_solutions.}} Actualiza la función
  auxiliar de la variable \texttt{fixer}.
\end{itemize}

Argumentos: * \texttt{fixer\_function.} Función o clase que ajustará los
individuos de la población que no cumplen con al menos una de las
restricciones del problema.

Valor de retorno: * Retorna la configuración con la actualización del
operador de cruza. El objetivo es poder aplicar varios operadores en
cascada.

\begin{itemize}
\tightlist
\item
  \emph{\textbf{adaptive\_mutation.}} Actualiza el operador de mutación
  de los \(\sigma\) de la variable \texttt{adaptive\_mutation\_op}.
\end{itemize}

Argumentos: * \texttt{adaptive\_mutation\_function.} Función o clase que
muta los tamaños de paso que se encuentran en \texttt{logger} con la
llave \texttt{parent\_population\_sigma}.

Valor de retorno: * Retorna la configuración con la actualización del
operador de cruza. El objetivo es poder aplicar varios operadores en
cascada.

    \subsection{Descripción de
operadores}\label{descripciuxf3n-de-operadores}

Los operadores de mutación, cruza y selección con los que cuenta la
librería \textbf{Pyristic} son clases. La finalidad es unificar el
formato de todos los operadores al ser llamados por los métodos de la
clase \texttt{EvolutionaryProgramming}.

\subsubsection{Operadores de mutación}\label{operadores-de-mutaciuxf3n}

\begin{itemize}
\item
  \emph{\textbf{sigma\_mutator.}} Operador de mutación en cada una de
  las soluciones de la población, donde, realiza la mutación de la
  siguiente manera:

  \begin{equation}
   x'_j = x_j + \sigma'_j \cdot N(0,1)
  \end{equation}

  donde \(x'_j\) es la variable mutada, \(x_j\) la variable a mutar,
  \(\sigma'_j\) el tamaño de paso (previamente mutado) y N(0,1) devuelve
  un número aleatorio usando una distribución normal con media \(0\) y
  desviación estándar igual con \(1\).
\end{itemize}

Constructor: * No recibe ningún argumento.

Métodos: * \emph{\textbf{\_\_call\_\_.}} Este método nos permite hacer
que nuestra clase se comporte como una función.

Argumentos: * \texttt{X.} Arreglo bidimensional de \emph{numpy}
representando a la población de soluciones de la iteración actual,
donde, el número de filas es igual al tamaño de la población y el número
de columnas es igual al número de variables que tiene el problema que se
está resolviendo.

\begin{verbatim}
* `Sigma.` Arreglo bidimensional de *numpy*, donde, cada fila representa los tamaños de paso y cada columna es una de las variables que tiene el problema que se está resolviendo.
\end{verbatim}

Valor de retorno: * Un arreglo bidimensional de \emph{numpy} del mismo
tamaño que el arreglo bidimensional de entrada \(X\).

\begin{itemize}
\item
  \emph{\textbf{sigma\_ep\_adaptive\_mutator.}} Operador de mutación en
  los tamaños de desplazamiento de cada uno de los individuos de la
  población. La mutación se realiza de la siguiente manera:

  \begin{equation}
   \sigma'_j = \sigma_j \cdot ( 1 + \alpha \cdot N(0,1))
  \end{equation}

  donde \(\sigma'_j\) es la variable mutada, \(\sigma_j\) la variable a
  mutar, \(\alpha\) parámetro de entrada por el usuario y N(0,1)
  devuelve un número aleatorio usando una distribución normal con media
  \(0\) y desviación estándar igual con \(1\).
\end{itemize}

Constructor * \texttt{decision\_variables.} Número de variables de
decisión del problema. * \texttt{alpha.} Número que será empleado en la
actualización de \(\sigma\).

Métodos: * \emph{\textbf{\_\_call\_\_.}} Este método nos permite hacer
que nuestra clase se comporte como una función.

\begin{verbatim}
Argumentos:
   * `X.` Arreglo bidimensional de *numpy* que representa los tamaños de paso de cada uno de los individuos de la población.
\end{verbatim}

Valor de retorno: * Arreglo bidimensional de \emph{numpy} con los nuevos
valores de tamaño de paso.

\subsubsection{Operadores de selección de
sobrevivientes}\label{operadores-de-selecciuxf3n-de-sobrevivientes}

\begin{itemize}
\tightlist
\item
  \emph{\textbf{merge\_selector.}} Esquema \((\mu + \lambda)\),
  selecciona \(\mu\) individuos que son obtenidos al unir la población
  de hijos y la población actual. Los individuos que permanecerán en la
  próxima generación son aquellos que tengan un mejor valor de aptitud.
\end{itemize}

Constructor: * No recibe ningún argumento.

Métodos: * \emph{\textbf{\_\_call\_\_.}} Este método nos permite hacer
que nuestra clase se comporte como una función.

Argumentos: * \texttt{parent\_f.} Arreglo de numpy de la población
almacenada en \texttt{parent\_population\_f}, donde, cada componente
representa el valor de la función objetivo por el individuo \(i\). *
\texttt{offspring\_f.} Arreglo de numpy de la población almacenada en
\texttt{offspring\_population\_f}, donde, cada componente representa el
valor de la función objetivo por el individuo \(i\). *
\texttt{features.} Diccionario que tiene las llaves de la información
que se desea mantener. Cada llave contiene un arreglo de dos
componentes, donde, la primera es la información de
\texttt{parent\_population} y la segunda componente es la información de
\texttt{offspring\_population}.

Valor de retorno: * Diccionario con los individuos seleccionados por
dicho esquema. Las llaves de este diccionario serán las mismas llaves
recibidas en el parámetro features y adicional otra llave con el nombre
\texttt{parent\_population\_f}, sin embargo, ahora sólo contendrá la
información de los individuos que pasarán a la próxima generación.

\begin{itemize}
\tightlist
\item
  \emph{\textbf{replacement\_selector.}} El esquema \((\mu, \lambda)\),
  reemplaza la población actual con los \(\mu\) mejores hijos de acuerdo
  a su valor de aptitud.
\end{itemize}

Constructor: * No recibe ningún argumento.

Métodos: * \emph{\textbf{\_\_call\_\_.}} Este método nos permite hacer
que nuestra clase se comporte como una función.

Argumentos: * \texttt{parent\_f.} Arreglo de numpy de la población
almacenada en \texttt{parent\_population\_f}, donde, cada componente
representa el valor de la función objetivo por el individuo \(i\). *
\texttt{offspring\_f.} Arreglo de numpy de la población almacenada en
\texttt{offspring\_population\_f}, donde, cada componente representa el
valor de la función objetivo por el individuo \(i\). *
\texttt{features.} Diccionario que tiene las llaves de la información
que se desea mantener. Cada llave contiene un arreglo de dos
componentes, donde, la primera es la información de
\texttt{parent\_population} y la segunda componente es la información de
\texttt{offspring\_population}.

Valor de retorno: * Diccionario con los individuos seleccionados por
dicho esquema. Las llaves de este diccionario serán las mismas llaves
recibidas en el parámetro features y adicional otra llave con el nombre
\texttt{parent\_population\_f}, sin embargo, ahora sólo contendrá la
información de los individuos que pasarán a la próxima generación.

    \subsection{Función de Beale}\label{funciuxf3n-de-beale}

\begin{equation}
  \label{eq:BF}
  \begin{array}{rll}
  \text{minimizar:} & f(x_1, x_2) = (1.5 - x_1 + x_1x_2)^2 + (2.25 - x_1 + x_1x_2^2)^2 + (2.625 - x_1 + x_1x_2^3)^2
  &  \\
  \text{tal que: } & -4.5 \leq x_1,x_2 \leq 4.5 &  
  \end{array}
\end{equation}

El mínimo global se encuentra en \(x^* = (3, 0.5)\) y \(f(x^*) = 0\).

    \begin{Verbatim}[commandchars=\\\{\}]
{\color{incolor}In [{\color{incolor}5}]:} \PY{n}{Image}\PY{p}{(}\PY{n}{filename}\PY{o}{=}\PY{l+s+s2}{\PYZdq{}}\PY{l+s+s2}{include/beale.png}\PY{l+s+s2}{\PYZdq{}}\PY{p}{,} \PY{n}{width}\PY{o}{=}\PY{l+m+mi}{500}\PY{p}{,} \PY{n}{height}\PY{o}{=}\PY{l+m+mi}{300}\PY{p}{)}
\end{Verbatim}
\texttt{\color{outcolor}Out[{\color{outcolor}5}]:}
    
    \begin{center}
    \adjustimage{max size={0.9\linewidth}{0.9\paperheight}}{PEEjemplo_files/PEEjemplo_12_0.png}
    \end{center}
    { \hspace*{\fill} \\}
    

    Para inicializar un objeto de la clase \texttt{EvolutionaryProgramming},
es necesario implementar los siguientes elementos.

    \paragraph{Función objetivo}\label{funciuxf3n-objetivo}

    \begin{Verbatim}[commandchars=\\\{\}]
{\color{incolor}In [{\color{incolor} }]:} \PY{k}{def} \PY{n+nf}{f}\PY{p}{(}\PY{n}{x} \PY{p}{:} \PY{n}{np}\PY{o}{.}\PY{n}{ndarray}\PY{p}{)} \PY{o}{\PYZhy{}}\PY{o}{\PYZgt{}} \PY{n+nb}{float}\PY{p}{:}
            \PY{n}{a} \PY{o}{=} \PY{p}{(}\PY{l+m+mf}{1.5} \PY{o}{\PYZhy{}} \PY{n}{x}\PY{p}{[}\PY{l+m+mi}{0}\PY{p}{]} \PY{o}{+} \PY{n}{x}\PY{p}{[}\PY{l+m+mi}{0}\PY{p}{]}\PY{o}{*}\PY{n}{x}\PY{p}{[}\PY{l+m+mi}{1}\PY{p}{]}\PY{p}{)}\PY{o}{*}\PY{o}{*}\PY{l+m+mi}{2}
            \PY{n}{b} \PY{o}{=} \PY{p}{(}\PY{l+m+mf}{2.25} \PY{o}{\PYZhy{}} \PY{n}{x}\PY{p}{[}\PY{l+m+mi}{0}\PY{p}{]} \PY{o}{+} \PY{n}{x}\PY{p}{[}\PY{l+m+mi}{0}\PY{p}{]}\PY{o}{*}\PY{n}{x}\PY{p}{[}\PY{l+m+mi}{1}\PY{p}{]}\PY{o}{*}\PY{o}{*}\PY{l+m+mi}{2}\PY{p}{)}\PY{o}{*}\PY{o}{*}\PY{l+m+mi}{2}
            \PY{n}{c} \PY{o}{=} \PY{p}{(}\PY{l+m+mf}{2.625} \PY{o}{\PYZhy{}} \PY{n}{x}\PY{p}{[}\PY{l+m+mi}{0}\PY{p}{]} \PY{o}{+} \PY{n}{x}\PY{p}{[}\PY{l+m+mi}{0}\PY{p}{]}\PY{o}{*}\PY{n}{x}\PY{p}{[}\PY{l+m+mi}{1}\PY{p}{]}\PY{o}{*}\PY{o}{*}\PY{l+m+mi}{3}\PY{p}{)}\PY{o}{*}\PY{o}{*}\PY{l+m+mi}{2}
            \PY{k}{return} \PY{n}{a}\PY{o}{+}\PY{n}{b}\PY{o}{+}\PY{n}{c}
\end{Verbatim}

    \paragraph{Restricciones}\label{restricciones}

Las restricciones se definen como una lista de funciones que retornan
valores booleanos, estos valores permiten revisar si una solución es
factible o no. En el caso de la función de Beale, solo se tienen
restricciones de caja.

    \begin{Verbatim}[commandchars=\\\{\}]
{\color{incolor}In [{\color{incolor} }]:} \PY{k}{def} \PY{n+nf}{is\PYZus{}feasible}\PY{p}{(}\PY{n}{x} \PY{p}{:} \PY{n}{np}\PY{o}{.}\PY{n}{ndarray}\PY{p}{)} \PY{o}{\PYZhy{}}\PY{o}{\PYZgt{}} \PY{n+nb}{bool}\PY{p}{:}
            \PY{k}{for} \PY{n}{i} \PY{o+ow}{in} \PY{n+nb}{range}\PY{p}{(}\PY{n+nb}{len}\PY{p}{(}\PY{n}{x}\PY{p}{)}\PY{p}{)}\PY{p}{:}
                \PY{k}{if} \PY{o}{\PYZhy{}}\PY{l+m+mf}{4.5}\PY{o}{\PYZgt{}}\PY{n}{x}\PY{p}{[}\PY{n}{i}\PY{p}{]} \PY{o+ow}{or} \PY{n}{x}\PY{p}{[}\PY{n}{i}\PY{p}{]} \PY{o}{\PYZgt{}} \PY{l+m+mf}{4.5}\PY{p}{:}
                    \PY{k}{return} \PY{k+kc}{False} 
            \PY{n}{is\PYZus{}feasible}\PY{o}{.}\PY{n+nv+vm}{\PYZus{}\PYZus{}doc\PYZus{}\PYZus{}}\PY{o}{=}\PY{l+s+s2}{\PYZdq{}}\PY{l+s+s2}{x1: \PYZhy{}4.5 \PYZlt{}= }\PY{l+s+si}{\PYZob{}:.2f\PYZcb{}}\PY{l+s+s2}{ \PYZlt{}= 4.5 }\PY{l+s+se}{\PYZbs{}n}\PY{l+s+s2}{ x2: \PYZhy{}4.5 \PYZlt{}= }\PY{l+s+si}{\PYZob{}:.2f\PYZcb{}}\PY{l+s+s2}{ \PYZlt{}= 4.5}\PY{l+s+s2}{\PYZdq{}}\PY{o}{.}\PY{n}{format}\PY{p}{(}\PY{n}{x}\PY{p}{[}\PY{l+m+mi}{0}\PY{p}{]}\PY{p}{,}\PY{n}{x}\PY{p}{[}\PY{l+m+mi}{1}\PY{p}{]}\PY{p}{)}
        
            \PY{k}{return} \PY{k+kc}{True}
        
        \PY{n}{constraints\PYZus{}} \PY{o}{=} \PY{p}{[}\PY{n}{is\PYZus{}feasible}\PY{p}{]}
\end{Verbatim}

    \paragraph{Límites de las variables del
problema}\label{luxedmites-de-las-variables-del-problema}

Los límites de las variables de decisión son establecidos en una matriz
(lista de listas) de tamaño (2 x \(n\)), donde \(n\) es el número de
variables. La primera fila contiene los límites inferiores de cada una
de las variables de decisión y la segunda fila los límites superiores.

    \begin{Verbatim}[commandchars=\\\{\}]
{\color{incolor}In [{\color{incolor}8}]:} \PY{n}{lower\PYZus{}bound} \PY{o}{=} \PY{p}{[}\PY{o}{\PYZhy{}}\PY{l+m+mf}{4.5}\PY{p}{,}\PY{o}{\PYZhy{}}\PY{l+m+mf}{4.5}\PY{p}{]}
        \PY{n}{upper\PYZus{}bound} \PY{o}{=} \PY{p}{[}\PY{l+m+mf}{4.5}\PY{p}{,} \PY{l+m+mf}{4.5}\PY{p}{]}
        \PY{n}{bounds} \PY{o}{=} \PY{p}{[}\PY{n}{lower\PYZus{}bound}\PY{p}{,} \PY{n}{upper\PYZus{}bound}\PY{p}{]}
\end{Verbatim}

    \emph{Nota:} En el caso de que todas las variables de decisión se
encuentren en el mismo rango de búsqueda, se puede utilizar una única
lista con dos valores numéricos, donde, el primer valor representa el
límite inferior y el segundo el límite superior.

Por ejemplo, la función de Beale es una función de dos variables
(\(x_{1}, x_{2}\)). La dos variables están acotadas en el mismo espacio
de búsqueda \$ -4.5 \textless{} x\_\{i\} \textless{} 4.5, i \in [1,2]\$,
entonces, en lugar de emplear la representación descrita antes, podemos
sustituirla por:

\begin{Shaded}
\begin{Highlighting}[]
\NormalTok{bounds }\OperatorTok{=}\NormalTok{ [}\OperatorTok{-}\FloatTok{4.5}\NormalTok{, }\FloatTok{4.5}\NormalTok{]}
\end{Highlighting}
\end{Shaded}

En esta representación, la primera componente se refiere al límite
inferior, mientras, la segunda componente es el límite superior. Este
arreglo será interpretado como el espacio de búsqueda para todas las
variables de decisión.

    La librería \textbf{Pyristic} tiene implementados algunos problemas de
prueba en \texttt{utils.helpers.test\_function}, entre ellos la función
de Beale. Los problemas de prueba están definidos como diccionarios con
las siguientes llaves:

\begin{itemize}
\tightlist
\item
  \texttt{function.} Función objetivo.
\item
  \texttt{contraints.} Restricciones del problema, lista con al menos
  una función que regresa un valor booleano.
\item
  \texttt{bounds.} Límites para cada una de las variables del problema.
  En el caso de que todas las variables del problema se encuentren en el
  mismo intervalo de búsqueda, se puede emplear una lista con dos
  valores numéricos.
\item
  \texttt{decision\_variables.} Número de variables de decisión.
\end{itemize}

    \begin{Verbatim}[commandchars=\\\{\}]
{\color{incolor}In [{\color{incolor} }]:} \PY{n}{beale\PYZus{}}
\end{Verbatim}

    \subsubsection{\texorpdfstring{Declaración de
\texttt{EvolutionaryProgramming}}{Declaración de EvolutionaryProgramming}}\label{declaraciuxf3n-de-evolutionaryprogramming}

La metaheurística de PE implementada en la librería \textbf{Pyristic} se
puede utilizar de las siguientes maneras:

\begin{itemize}
\tightlist
\item
  Crear una clase que herede de la clase
  \texttt{EvolutionaryProgramming} y sobreescribir las funciones antes
  mencionadas.
\item
  Declarar un objeto del tipo \texttt{EvolutionaryProgrammingConfig} y
  ajustar los operadores.
\item
  Realizar una combinación de las dos anteriores.
\end{itemize}

    Es importante resaltar que se puede hacer uso de
\texttt{EvolutionaryProgramming} sin modificar los operadores que tiene
por defecto.

    \subsubsection{Ejecución de la
metaheurística}\label{ejecuciuxf3n-de-la-metaheuruxedstica}

Como mencionamos antes, una forma de utilizar la metaheurística es hacer
una instancia de la clase \texttt{EvolutionProgramming} dejando su
configuración por defecto.

Los argumentos que se deben indicar al momento de inicializar son: *
función objetivo * restricciones del problema * límites inferior y
superior (por cada variable de decisión) * número de variables que tiene
el problema.

    \begin{Verbatim}[commandchars=\\\{\}]
{\color{incolor}In [{\color{incolor}10}]:} \PY{n}{Beale\PYZus{}optimizer} \PY{o}{=} \PY{n}{EvolutionaryProgramming}\PY{p}{(}\PY{o}{*}\PY{o}{*}\PY{n}{beale\PYZus{}}\PY{p}{)}
\end{Verbatim}

    Recordemos que \texttt{beale\_} es un diccionario con la información
requerida por el constructor de la clase
\texttt{EvolutionaryProgramming}.

Finalmente, se llama a la función \texttt{optimize}. Recibe los
siguientes parámetros:

\begin{itemize}
\tightlist
\item
  \textbf{generations}. Número de generaciones (iteraciones de la
  metaheurística).
\item
  \textbf{size\_population}. Tamaño de la población (número de
  individuos).
\item
  \textbf{verbose}. Muestra en qué iteración se encuentra nuestra
  búsqueda, por defecto está en True.
\item
  ****kwargs**. Argumentos externos a la búsqueda.
\end{itemize}

Para resolver la función de Beale utilizaremos los siguientes
parámetros: * \textbf{generations} = \(200\) * \textbf{size\_population}
= \(100\) * \textbf{verbose} = True.

    \begin{Verbatim}[commandchars=\\\{\}]
{\color{incolor}In [{\color{incolor}11}]:} \PY{n}{Beale\PYZus{}optimizer}\PY{o}{.}\PY{n}{optimize}\PY{p}{(}\PY{l+m+mi}{200}\PY{p}{,}\PY{l+m+mi}{100}\PY{p}{)}
\end{Verbatim}

    \begin{Verbatim}[commandchars=\\\{\}]
100\%|██████████| 200/200 [00:00<00:00, 271.94it/s]

    \end{Verbatim}

    \begin{Verbatim}[commandchars=\\\{\}]
{\color{incolor}In [{\color{incolor}12}]:} \PY{n+nb}{print}\PY{p}{(}\PY{n}{Beale\PYZus{}optimizer}\PY{p}{)}
\end{Verbatim}

    \begin{Verbatim}[commandchars=\\\{\}]
Evolutive Programming search: 
 f(X) = 2.0092647591277975e-21 
 X = [3.  0.5] 
 Constraints: 
 x1: -4.5 <= 3.00 <= 4.5 
 x2: -4.5 <= 0.50 <= 4.5 


    \end{Verbatim}

    Para revisar el comportamiento de la metaheurística en determinado
problema, la librería \textbf{Pyristic} cuenta con una función llamada
\texttt{get\_stats}. Esta función se encuentra en \texttt{utils.helpers}
y recibe como parámetros:

\begin{itemize}
\tightlist
\item
  Objeto que realiza la búsqueda de soluciones.
\item
  El número de veces que se quiere ejecutar la metaheurística.
\item
  Los argumentos que recibe la función \texttt{optimize} (debe ser una
  tupla).
\item
  Argumentos adicionales a la búsqueda, estos argumentos deben estar
  contenidos en un diccionario (opcional).
\end{itemize}

La función \texttt{get\_stats} retorna un diccionario con algunas
estadísticas de las ejecuciones.

    \begin{Verbatim}[commandchars=\\\{\}]
{\color{incolor}In [{\color{incolor}13}]:} \PY{n}{args} \PY{o}{=} \PY{p}{(}\PY{l+m+mi}{700}\PY{p}{,} \PY{l+m+mi}{100}\PY{p}{,} \PY{k+kc}{False}\PY{p}{)}
         \PY{n}{statistics} \PY{o}{=} \PY{n}{get\PYZus{}stats}\PY{p}{(}\PY{n}{Beale\PYZus{}optimizer}\PY{p}{,} \PY{l+m+mi}{21}\PY{p}{,} \PY{n}{args}\PY{p}{)}
\end{Verbatim}

    \begin{Verbatim}[commandchars=\\\{\}]
{\color{incolor}In [{\color{incolor}14}]:} \PY{n}{pprint}\PY{p}{(}\PY{n}{statistics}\PY{p}{)}
\end{Verbatim}

    \begin{Verbatim}[commandchars=\\\{\}]
\{'Best solution': \{'f': 0.0, 'x': array([3. , 0.5])\},
 'Mean': 3.239004768779724e-28,
 'Median': 0.0,
 'Standard deviation': 1.2685335049178072e-27,
 'Worst solution': \{'f': 5.945795635558406e-27, 'x': array([3. , 0.5])\}\}

    \end{Verbatim}

    \subsection{Función de Ackley}\label{funciuxf3n-de-ackley}

\begin{equation}
  \min f(\vec{x}) = -20\exp \left( -0.2 \sqrt{\frac{1}{n} \sum_{i=1}^n x_i^2} \right) 
  - exp \left( \frac{1}{n} \sum_{i=1}^n \cos (2\pi x_i) \right)
  + 20 + e  
\end{equation}

El mínimo global está en \$x\^{}* = 0 \$ y \(f(\vec{x}) = 0\) y su
dominio es \(|x_{i}| < 30\).

    \begin{Verbatim}[commandchars=\\\{\}]
{\color{incolor}In [{\color{incolor}15}]:} \PY{n}{Image}\PY{p}{(}\PY{n}{filename}\PY{o}{=}\PY{l+s+s2}{\PYZdq{}}\PY{l+s+s2}{include/ackley.jpg}\PY{l+s+s2}{\PYZdq{}}\PY{p}{,} \PY{n}{width}\PY{o}{=}\PY{l+m+mi}{500}\PY{p}{,} \PY{n}{height}\PY{o}{=}\PY{l+m+mi}{300}\PY{p}{)}
\end{Verbatim}
\texttt{\color{outcolor}Out[{\color{outcolor}15}]:}
    
    \begin{center}
    \adjustimage{max size={0.9\linewidth}{0.9\paperheight}}{PEEjemplo_files/PEEjemplo_34_0.jpeg}
    \end{center}
    { \hspace*{\fill} \\}
    

    \begin{Verbatim}[commandchars=\\\{\}]
{\color{incolor}In [{\color{incolor}16}]:} \PY{n}{ackley\PYZus{}}
\end{Verbatim}

\begin{Verbatim}[commandchars=\\\{\}]
{\color{outcolor}Out[{\color{outcolor}16}]:} \{'function': CPUDispatcher(<function ackley\_function at 0x7ffb974a8378>),
          'constraints': [CPUDispatcher(<function constraint1\_ackley at 0x7ffb974a8620>)],
          'bounds': [-30.0, 30.0],
          'decision\_variables': 10\}
\end{Verbatim}
            
    En el caso de la función de Ackley, al no estár con restricción de
variables de decisión, podemos modificar el número de variables de
decisión que tiene por defecto de la siguiente manera:

\begin{Shaded}
\begin{Highlighting}[]
\NormalTok{ackley_[}\StringTok{'decision_variable'}\NormalTok{] }\OperatorTok{=} \DecValTok{5}
\end{Highlighting}
\end{Shaded}

    Para resolver la función de Ackley con 10 variables de decisión
utilizaremos los siguientes parámetros: * \textbf{generations} = \(500\)
* \textbf{size\_population} = \(100\)

    \begin{Verbatim}[commandchars=\\\{\}]
{\color{incolor}In [{\color{incolor}17}]:} \PY{n}{Optimizer\PYZus{}by\PYZus{}default} \PY{o}{=} \PY{n}{EvolutionaryProgramming}\PY{p}{(}\PY{o}{*}\PY{o}{*}\PY{n}{ackley\PYZus{}}\PY{p}{)}
\end{Verbatim}

    \begin{Verbatim}[commandchars=\\\{\}]
{\color{incolor}In [{\color{incolor}18}]:} \PY{n}{Optimizer\PYZus{}by\PYZus{}default}\PY{o}{.}\PY{n}{optimize}\PY{p}{(}\PY{l+m+mi}{500}\PY{p}{,}\PY{l+m+mi}{100}\PY{p}{)}
\end{Verbatim}

    \begin{Verbatim}[commandchars=\\\{\}]
100\%|██████████| 500/500 [00:01<00:00, 287.94it/s]

    \end{Verbatim}

    \begin{Verbatim}[commandchars=\\\{\}]
{\color{incolor}In [{\color{incolor}19}]:} \PY{n+nb}{print}\PY{p}{(}\PY{n}{Optimizer\PYZus{}by\PYZus{}default}\PY{p}{)}
\end{Verbatim}

    \begin{Verbatim}[commandchars=\\\{\}]
Evolutive Programming search: 
 f(X) = 14.236808930846905 
 X = [ 2.05309403e+00 -3.99185951e+00  1.23496472e-02  2.24880725e-04
  6.01696518e-09 -5.80408096e+00 -1.40003025e+01  6.90122701e+00
 -9.91887617e-01  7.98370535e+00] 
 Constraints: 
 x1: -30 <= 2.05 <= 30 
 x2: -30 <= -3.99 <= 30 
 x3: -30 <= 0.01 <= 30 
 x4: -30 <= 0.00 <= 30 
 x5: -30 <= 0.00 <= 30 
 x6: -30 <= -5.80 <= 30 
 x7: -30 <= -14.00 <= 30 
 x8: -30 <= 6.90 <= 30 
 x9: -30 <= -0.99 <= 30 
 x10: -30 <= 7.98 <= 30 
  


    \end{Verbatim}

    La solución encontrada por PE no es la solución óptima. A continuación
mostraremos la ejecución de la metaheurística modificando la
configuración por defecto.

    \subsubsection{\texorpdfstring{Declaración de
\texttt{EvolutionaryProgramming} por
configuración}{Declaración de EvolutionaryProgramming por configuración}}\label{declaraciuxf3n-de-evolutionaryprogramming-por-configuraciuxf3n}

    A continuación vamos a mostrar la forma en que se declara un objeto del
tipo \texttt{EvolutionaryProgrammingConfig} que es una clase auxiliar,
donde, contendrá los operadores que se emplea en la ejecución de la
metaheurística.

    \begin{Verbatim}[commandchars=\\\{\}]
{\color{incolor}In [{\color{incolor}20}]:} \PY{n}{configuration} \PY{o}{=} \PY{p}{(}\PY{n}{EvolutionaryProgrammingConfig}\PY{p}{(}\PY{p}{)}
                          \PY{o}{.}\PY{n}{adaptive\PYZus{}mutation}\PY{p}{(}
                              \PY{n}{mutation}\PY{o}{.}\PY{n}{sigma\PYZus{}ep\PYZus{}adaptive\PYZus{}mutator}\PY{p}{(}\PY{n}{ackley\PYZus{}}\PY{p}{[}\PY{l+s+s1}{\PYZsq{}}\PY{l+s+s1}{decision\PYZus{}variables}\PY{l+s+s1}{\PYZsq{}}\PY{p}{]}\PY{p}{,} \PY{l+m+mf}{2.0}\PY{p}{)}
                              \PY{p}{)} 
                          \PY{p}{)}
\end{Verbatim}

    En este ejemplo mostraremos el impacto de inicializar el operador de
mutación en los tamaños de paso con \(\alpha = 2\), el constructor en
caso de no incluir este parámetro lo inicializa con \(\alpha = 0.5\).

    \begin{Verbatim}[commandchars=\\\{\}]
{\color{incolor}In [{\color{incolor}21}]:} \PY{n+nb}{print}\PY{p}{(}\PY{n}{configuration}\PY{p}{)}
\end{Verbatim}

    \begin{Verbatim}[commandchars=\\\{\}]
--------------------------------
	Configuration
--------------------------------
Adaptive mutation: Sigma EP.
		-Alpha: 2.0

--------------------------------

    \end{Verbatim}

    \begin{Verbatim}[commandchars=\\\{\}]
{\color{incolor}In [{\color{incolor}22}]:} \PY{n}{Optimizer\PYZus{}by\PYZus{}configuration} \PY{o}{=} \PY{n}{EvolutionaryProgramming}\PY{p}{(}\PY{o}{*}\PY{o}{*}\PY{n}{ackley\PYZus{}}\PY{p}{,} \PY{n}{config}\PY{o}{=}\PY{n}{configuration}\PY{p}{)}
\end{Verbatim}

    A diferencia del ejemplo de la función de Beale, incluimos la
configuración de los operadores que deseamos utilizar en la variable
llamada \texttt{config} al crear un objeto de la clase
\texttt{EvolutionaryProgramming}.

    \begin{Verbatim}[commandchars=\\\{\}]
{\color{incolor}In [{\color{incolor}23}]:} \PY{n}{Optimizer\PYZus{}by\PYZus{}configuration}\PY{o}{.}\PY{n}{optimize}\PY{p}{(}\PY{l+m+mi}{500}\PY{p}{,}\PY{l+m+mi}{100}\PY{p}{)}
\end{Verbatim}

    \begin{Verbatim}[commandchars=\\\{\}]
100\%|██████████| 500/500 [00:01<00:00, 286.96it/s]

    \end{Verbatim}

    \begin{Verbatim}[commandchars=\\\{\}]
{\color{incolor}In [{\color{incolor}24}]:} \PY{n+nb}{print}\PY{p}{(}\PY{n}{Optimizer\PYZus{}by\PYZus{}configuration}\PY{p}{)}
\end{Verbatim}

    \begin{Verbatim}[commandchars=\\\{\}]
Evolutive Programming search: 
 f(X) = 1.904985106376913 
 X = [-0.193743   -1.12343605  0.11622617  0.13151275  0.03091606  0.01765212
 -0.10585762 -0.05992731 -0.04231196  0.07508058] 
 Constraints: 
 x1: -30 <= -0.19 <= 30 
 x2: -30 <= -1.12 <= 30 
 x3: -30 <= 0.12 <= 30 
 x4: -30 <= 0.13 <= 30 
 x5: -30 <= 0.03 <= 30 
 x6: -30 <= 0.02 <= 30 
 x7: -30 <= -0.11 <= 30 
 x8: -30 <= -0.06 <= 30 
 x9: -30 <= -0.04 <= 30 
 x10: -30 <= 0.08 <= 30 
  


    \end{Verbatim}

    \subsubsection{\texorpdfstring{Herencia desde
\emph{EvolutionaryProgramming}}{Herencia desde EvolutionaryProgramming}}\label{herencia-desde-evolutionaryprogramming}

    Otra forma de utilizar la metaheurística de nuestra librería es
definiendo una clase que herede de \texttt{EvolutionaryProgramming},
donde, vamos a sobreescribir el método \texttt{adaptive\_mutation} y así
permitir incluir distintos valores para \(\alpha\).

    \begin{Verbatim}[commandchars=\\\{\}]
{\color{incolor}In [{\color{incolor}25}]:} \PY{k}{class} \PY{n+nc}{EPAckley}\PY{p}{(}\PY{n}{EvolutionaryProgramming}\PY{p}{)}\PY{p}{:}
             \PY{k}{def} \PY{n+nf}{\PYZus{}\PYZus{}init\PYZus{}\PYZus{}}\PY{p}{(}\PY{n+nb+bp}{self}\PY{p}{,} \PY{n}{function}\PY{p}{,} \PY{n}{decision\PYZus{}variables}\PY{p}{,} \PY{n}{constraints}\PY{p}{,} \PY{n}{bounds}\PY{p}{)}\PY{p}{:}
                 \PY{n+nb}{super}\PY{p}{(}\PY{p}{)}\PY{o}{.}\PY{n+nf+fm}{\PYZus{}\PYZus{}init\PYZus{}\PYZus{}}\PY{p}{(}\PY{n}{function}\PY{p}{,} \PY{n}{decision\PYZus{}variables}\PY{p}{,} \PY{n}{constraints}\PY{p}{,} \PY{n}{bounds}\PY{p}{)}
             
             \PY{k}{def} \PY{n+nf}{adaptive\PYZus{}mutation}\PY{p}{(}\PY{n+nb+bp}{self}\PY{p}{,} \PY{o}{*}\PY{o}{*}\PY{n}{kwargs}\PY{p}{)}\PY{p}{:}
                 \PY{n}{alpha\PYZus{}} \PY{o}{=} \PY{n}{kwargs}\PY{p}{[}\PY{l+s+s1}{\PYZsq{}}\PY{l+s+s1}{alpha}\PY{l+s+s1}{\PYZsq{}}\PY{p}{]}
                 \PY{k}{return} \PY{n}{mutation}\PY{o}{.}\PY{n}{sigma\PYZus{}ep\PYZus{}adaptive}\PY{p}{(}\PY{n+nb+bp}{self}\PY{o}{.}\PY{n}{logger}\PY{p}{[}\PY{l+s+s1}{\PYZsq{}}\PY{l+s+s1}{parent\PYZus{}population\PYZus{}sigma}\PY{l+s+s1}{\PYZsq{}}\PY{p}{]}\PY{p}{,} \PY{n}{alpha\PYZus{}}\PY{p}{)}
\end{Verbatim}

    \begin{Verbatim}[commandchars=\\\{\}]
{\color{incolor}In [{\color{incolor}26}]:} \PY{n}{additional\PYZus{}arguments} \PY{o}{=} \PY{p}{\PYZob{}}\PY{l+s+s1}{\PYZsq{}}\PY{l+s+s1}{alpha}\PY{l+s+s1}{\PYZsq{}}\PY{p}{:}\PY{l+m+mf}{2.0}\PY{p}{\PYZcb{}}
\end{Verbatim}

    El diccionario que hemos definido tiene el parámetro que se utiliza en
la función \texttt{adaptive\_mutation}.

    \begin{Verbatim}[commandchars=\\\{\}]
{\color{incolor}In [{\color{incolor}27}]:} \PY{n}{Optimizer\PYZus{}by\PYZus{}class} \PY{o}{=} \PY{n}{EPAckley}\PY{p}{(}\PY{o}{*}\PY{o}{*}\PY{n}{ackley\PYZus{}}\PY{p}{)}
\end{Verbatim}

    \begin{Verbatim}[commandchars=\\\{\}]
{\color{incolor}In [{\color{incolor}28}]:} \PY{n}{Optimizer\PYZus{}by\PYZus{}class}\PY{o}{.}\PY{n}{optimize}\PY{p}{(}\PY{l+m+mi}{500}\PY{p}{,}\PY{l+m+mi}{100}\PY{p}{,}\PY{o}{*}\PY{o}{*}\PY{n}{additional\PYZus{}arguments}\PY{p}{)}
\end{Verbatim}

    \begin{Verbatim}[commandchars=\\\{\}]
100\%|██████████| 500/500 [00:01<00:00, 326.76it/s]

    \end{Verbatim}

    \begin{Verbatim}[commandchars=\\\{\}]
{\color{incolor}In [{\color{incolor}29}]:} \PY{n+nb}{print}\PY{p}{(}\PY{n}{Optimizer\PYZus{}by\PYZus{}class}\PY{p}{)}
\end{Verbatim}

    \begin{Verbatim}[commandchars=\\\{\}]
Evolutive Programming search: 
 f(X) = 1.3313378339717095 
 X = [-0.03217673 -0.20545881 -0.02026904  0.0380225  -0.04857584  0.13552289
  0.18398808 -0.18142151  0.04354127 -0.24854929] 
 Constraints: 
 x1: -30 <= -0.03 <= 30 
 x2: -30 <= -0.21 <= 30 
 x3: -30 <= -0.02 <= 30 
 x4: -30 <= 0.04 <= 30 
 x5: -30 <= -0.05 <= 30 
 x6: -30 <= 0.14 <= 30 
 x7: -30 <= 0.18 <= 30 
 x8: -30 <= -0.18 <= 30 
 x9: -30 <= 0.04 <= 30 
 x10: -30 <= -0.25 <= 30 
  


    \end{Verbatim}

    \subsubsection{Resultados}\label{resultados}

    \paragraph{Usando la configuración por
defecto}\label{usando-la-configuraciuxf3n-por-defecto}

    \begin{Verbatim}[commandchars=\\\{\}]
{\color{incolor}In [{\color{incolor}30}]:} \PY{n}{args} \PY{o}{=} \PY{p}{(}\PY{l+m+mi}{500}\PY{p}{,}\PY{l+m+mi}{100}\PY{p}{,}\PY{k+kc}{False}\PY{p}{)}
         \PY{n}{statistics} \PY{o}{=} \PY{n}{get\PYZus{}stats}\PY{p}{(}\PY{n}{Optimizer\PYZus{}by\PYZus{}default}\PY{p}{,} \PY{l+m+mi}{21}\PY{p}{,} \PY{n}{args}\PY{p}{)}
\end{Verbatim}

    \begin{Verbatim}[commandchars=\\\{\}]
{\color{incolor}In [{\color{incolor}31}]:} \PY{n}{pprint}\PY{p}{(}\PY{n}{statistics}\PY{p}{)}
\end{Verbatim}

    \begin{Verbatim}[commandchars=\\\{\}]
\{'Best solution': \{'f': 9.772165021436392,
                   'x': array([-1.98705450e+00, -2.00414714e-01, -1.97284063e+00,  5.39641110e-04,
        6.95424646e+00, -2.35497197e-02, -5.95536851e+00,  1.98709111e+00,
       -1.01744137e+00,  2.99379761e+00])\},
 'Mean': 14.438796656966852,
 'Median': 14.57877829622885,
 'Standard deviation': 1.9055002610689908,
 'Worst solution': \{'f': 17.81753107723958,
                    'x': array([ 13.02996391,  -2.99887137,  13.98922842,  13.99468984,
        -0.96250749,   8.99658661, -10.05791112,   0.99054141,
        20.99209058,   4.00012043])\}\}

    \end{Verbatim}

    \paragraph{Indicando la configuración que va a utilizar
PE}\label{indicando-la-configuraciuxf3n-que-va-a-utilizar-pe}

    \begin{Verbatim}[commandchars=\\\{\}]
{\color{incolor}In [{\color{incolor}32}]:} \PY{n}{args} \PY{o}{=} \PY{p}{(}\PY{l+m+mi}{500}\PY{p}{,}\PY{l+m+mi}{100}\PY{p}{,}\PY{k+kc}{False}\PY{p}{)}
         \PY{n}{statistics} \PY{o}{=} \PY{n}{get\PYZus{}stats}\PY{p}{(}\PY{n}{Optimizer\PYZus{}by\PYZus{}configuration}\PY{p}{,} \PY{l+m+mi}{21}\PY{p}{,} \PY{n}{args}\PY{p}{)}
\end{Verbatim}

    \begin{Verbatim}[commandchars=\\\{\}]
{\color{incolor}In [{\color{incolor}33}]:} \PY{n}{pprint}\PY{p}{(}\PY{n}{statistics}\PY{p}{)}
\end{Verbatim}

    \begin{Verbatim}[commandchars=\\\{\}]
\{'Best solution': \{'f': 0.12388792866390874,
                   'x': array([ 0.05835097,  0.01067072,  0.01612713,  0.00163387,  0.01306192,
       -0.00391465,  0.00555765,  0.00590317,  0.02159629,  0.03295279])\},
 'Mean': 3.3380048902920185,
 'Median': 2.171638680046055,
 'Standard deviation': 3.0037931500732697,
 'Worst solution': \{'f': 12.526042771165503,
                    'x': array([ 6.40082881e-02, -1.90140665e+00,  1.23486372e-02,  4.37177466e-02,
        5.15705963e+00, -7.71560336e-02,  9.83440971e-01, -1.96073006e+00,
       -1.05264738e+00,  1.32558098e+01])\}\}

    \end{Verbatim}

    \paragraph{\texorpdfstring{Creando una clase que hereda de
\emph{EvolutionaryProgramming}}{Creando una clase que hereda de EvolutionaryProgramming}}\label{creando-una-clase-que-hereda-de-evolutionaryprogramming}

    \begin{Verbatim}[commandchars=\\\{\}]
{\color{incolor}In [{\color{incolor}34}]:} \PY{n}{args} \PY{o}{=} \PY{p}{(}\PY{l+m+mi}{500}\PY{p}{,}\PY{l+m+mi}{100}\PY{p}{,}\PY{k+kc}{False}\PY{p}{)}
         \PY{n}{statistics} \PY{o}{=} \PY{n}{get\PYZus{}stats}\PY{p}{(}\PY{n}{Optimizer\PYZus{}by\PYZus{}class}\PY{p}{,} \PY{l+m+mi}{21}\PY{p}{,} \PY{n}{args}\PY{p}{,}\PY{o}{*}\PY{o}{*}\PY{n}{additional\PYZus{}arguments}\PY{p}{)}
\end{Verbatim}

    \begin{Verbatim}[commandchars=\\\{\}]
{\color{incolor}In [{\color{incolor}35}]:} \PY{n}{pprint}\PY{p}{(}\PY{n}{statistics}\PY{p}{)}
\end{Verbatim}

    \begin{Verbatim}[commandchars=\\\{\}]
\{'Best solution': \{'f': 0.025376897823943256,
                   'x': array([-0.00120468,  0.00352147, -0.0045472 , -0.00741508,  0.00987255,
        0.00907506,  0.0033641 , -0.00051152,  0.00412521,  0.00694506])\},
 'Mean': 3.1769907426285546,
 'Median': 2.922135369198639,
 'Standard deviation': 1.9952015696873395,
 'Worst solution': \{'f': 8.32256538117469,
                    'x': array([-1.25405474, -0.44577715, -0.25906636,  4.17355772, -0.77957849,
        0.80523909,  2.54725241, -2.06024977, -2.65550508, -0.35793377])\}\}

    \end{Verbatim}

    Las últimas dos estrategias utilizan la misma configuración, la única
diferencia es la forma en que se han creado y los resultados obtenidos
pueden variar por los números aleatorios generados.


    % Add a bibliography block to the postdoc
    
    
    
    \end{document}
