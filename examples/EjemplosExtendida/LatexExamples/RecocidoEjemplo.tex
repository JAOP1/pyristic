
% Default to the notebook output style

    


% Inherit from the specified cell style.




    
\documentclass[11pt]{article}

    
    
    \usepackage[T1]{fontenc}
    % Nicer default font (+ math font) than Computer Modern for most use cases
    \usepackage{mathpazo}

    % Basic figure setup, for now with no caption control since it's done
    % automatically by Pandoc (which extracts ![](path) syntax from Markdown).
    \usepackage{graphicx}
    % We will generate all images so they have a width \maxwidth. This means
    % that they will get their normal width if they fit onto the page, but
    % are scaled down if they would overflow the margins.
    \makeatletter
    \def\maxwidth{\ifdim\Gin@nat@width>\linewidth\linewidth
    \else\Gin@nat@width\fi}
    \makeatother
    \let\Oldincludegraphics\includegraphics
    % Set max figure width to be 80% of text width, for now hardcoded.
    \renewcommand{\includegraphics}[1]{\Oldincludegraphics[width=.8\maxwidth]{#1}}
    % Ensure that by default, figures have no caption (until we provide a
    % proper Figure object with a Caption API and a way to capture that
    % in the conversion process - todo).
    \usepackage{caption}
    \DeclareCaptionLabelFormat{nolabel}{}
    \captionsetup{labelformat=nolabel}

    \usepackage{adjustbox} % Used to constrain images to a maximum size 
    \usepackage{xcolor} % Allow colors to be defined
    \usepackage{enumerate} % Needed for markdown enumerations to work
    \usepackage{geometry} % Used to adjust the document margins
    \usepackage{amsmath} % Equations
    \usepackage{amssymb} % Equations
    \usepackage{textcomp} % defines textquotesingle
    % Hack from http://tex.stackexchange.com/a/47451/13684:
    \AtBeginDocument{%
        \def\PYZsq{\textquotesingle}% Upright quotes in Pygmentized code
    }
    \usepackage{upquote} % Upright quotes for verbatim code
    \usepackage{eurosym} % defines \euro
    \usepackage[mathletters]{ucs} % Extended unicode (utf-8) support
    \usepackage[utf8x]{inputenc} % Allow utf-8 characters in the tex document
    \usepackage{fancyvrb} % verbatim replacement that allows latex
    \usepackage{grffile} % extends the file name processing of package graphics 
                         % to support a larger range 
    % The hyperref package gives us a pdf with properly built
    % internal navigation ('pdf bookmarks' for the table of contents,
    % internal cross-reference links, web links for URLs, etc.)
    \usepackage{hyperref}
    \usepackage{longtable} % longtable support required by pandoc >1.10
    \usepackage{booktabs}  % table support for pandoc > 1.12.2
    \usepackage[inline]{enumitem} % IRkernel/repr support (it uses the enumerate* environment)
    \usepackage[normalem]{ulem} % ulem is needed to support strikethroughs (\sout)
                                % normalem makes italics be italics, not underlines
    \usepackage{mathrsfs}
    

    
    
    % Colors for the hyperref package
    \definecolor{urlcolor}{rgb}{0,.145,.698}
    \definecolor{linkcolor}{rgb}{.71,0.21,0.01}
    \definecolor{citecolor}{rgb}{.12,.54,.11}

    % ANSI colors
    \definecolor{ansi-black}{HTML}{3E424D}
    \definecolor{ansi-black-intense}{HTML}{282C36}
    \definecolor{ansi-red}{HTML}{E75C58}
    \definecolor{ansi-red-intense}{HTML}{B22B31}
    \definecolor{ansi-green}{HTML}{00A250}
    \definecolor{ansi-green-intense}{HTML}{007427}
    \definecolor{ansi-yellow}{HTML}{DDB62B}
    \definecolor{ansi-yellow-intense}{HTML}{B27D12}
    \definecolor{ansi-blue}{HTML}{208FFB}
    \definecolor{ansi-blue-intense}{HTML}{0065CA}
    \definecolor{ansi-magenta}{HTML}{D160C4}
    \definecolor{ansi-magenta-intense}{HTML}{A03196}
    \definecolor{ansi-cyan}{HTML}{60C6C8}
    \definecolor{ansi-cyan-intense}{HTML}{258F8F}
    \definecolor{ansi-white}{HTML}{C5C1B4}
    \definecolor{ansi-white-intense}{HTML}{A1A6B2}
    \definecolor{ansi-default-inverse-fg}{HTML}{FFFFFF}
    \definecolor{ansi-default-inverse-bg}{HTML}{000000}

    % commands and environments needed by pandoc snippets
    % extracted from the output of `pandoc -s`
    \providecommand{\tightlist}{%
      \setlength{\itemsep}{0pt}\setlength{\parskip}{0pt}}
    \DefineVerbatimEnvironment{Highlighting}{Verbatim}{commandchars=\\\{\}}
    % Add ',fontsize=\small' for more characters per line
    \newenvironment{Shaded}{}{}
    \newcommand{\KeywordTok}[1]{\textcolor[rgb]{0.00,0.44,0.13}{\textbf{{#1}}}}
    \newcommand{\DataTypeTok}[1]{\textcolor[rgb]{0.56,0.13,0.00}{{#1}}}
    \newcommand{\DecValTok}[1]{\textcolor[rgb]{0.25,0.63,0.44}{{#1}}}
    \newcommand{\BaseNTok}[1]{\textcolor[rgb]{0.25,0.63,0.44}{{#1}}}
    \newcommand{\FloatTok}[1]{\textcolor[rgb]{0.25,0.63,0.44}{{#1}}}
    \newcommand{\CharTok}[1]{\textcolor[rgb]{0.25,0.44,0.63}{{#1}}}
    \newcommand{\StringTok}[1]{\textcolor[rgb]{0.25,0.44,0.63}{{#1}}}
    \newcommand{\CommentTok}[1]{\textcolor[rgb]{0.38,0.63,0.69}{\textit{{#1}}}}
    \newcommand{\OtherTok}[1]{\textcolor[rgb]{0.00,0.44,0.13}{{#1}}}
    \newcommand{\AlertTok}[1]{\textcolor[rgb]{1.00,0.00,0.00}{\textbf{{#1}}}}
    \newcommand{\FunctionTok}[1]{\textcolor[rgb]{0.02,0.16,0.49}{{#1}}}
    \newcommand{\RegionMarkerTok}[1]{{#1}}
    \newcommand{\ErrorTok}[1]{\textcolor[rgb]{1.00,0.00,0.00}{\textbf{{#1}}}}
    \newcommand{\NormalTok}[1]{{#1}}
    
    % Additional commands for more recent versions of Pandoc
    \newcommand{\ConstantTok}[1]{\textcolor[rgb]{0.53,0.00,0.00}{{#1}}}
    \newcommand{\SpecialCharTok}[1]{\textcolor[rgb]{0.25,0.44,0.63}{{#1}}}
    \newcommand{\VerbatimStringTok}[1]{\textcolor[rgb]{0.25,0.44,0.63}{{#1}}}
    \newcommand{\SpecialStringTok}[1]{\textcolor[rgb]{0.73,0.40,0.53}{{#1}}}
    \newcommand{\ImportTok}[1]{{#1}}
    \newcommand{\DocumentationTok}[1]{\textcolor[rgb]{0.73,0.13,0.13}{\textit{{#1}}}}
    \newcommand{\AnnotationTok}[1]{\textcolor[rgb]{0.38,0.63,0.69}{\textbf{\textit{{#1}}}}}
    \newcommand{\CommentVarTok}[1]{\textcolor[rgb]{0.38,0.63,0.69}{\textbf{\textit{{#1}}}}}
    \newcommand{\VariableTok}[1]{\textcolor[rgb]{0.10,0.09,0.49}{{#1}}}
    \newcommand{\ControlFlowTok}[1]{\textcolor[rgb]{0.00,0.44,0.13}{\textbf{{#1}}}}
    \newcommand{\OperatorTok}[1]{\textcolor[rgb]{0.40,0.40,0.40}{{#1}}}
    \newcommand{\BuiltInTok}[1]{{#1}}
    \newcommand{\ExtensionTok}[1]{{#1}}
    \newcommand{\PreprocessorTok}[1]{\textcolor[rgb]{0.74,0.48,0.00}{{#1}}}
    \newcommand{\AttributeTok}[1]{\textcolor[rgb]{0.49,0.56,0.16}{{#1}}}
    \newcommand{\InformationTok}[1]{\textcolor[rgb]{0.38,0.63,0.69}{\textbf{\textit{{#1}}}}}
    \newcommand{\WarningTok}[1]{\textcolor[rgb]{0.38,0.63,0.69}{\textbf{\textit{{#1}}}}}
    
    
    % Define a nice break command that doesn't care if a line doesn't already
    % exist.
    \def\br{\hspace*{\fill} \\* }
    % Math Jax compatibility definitions
    \def\gt{>}
    \def\lt{<}
    \let\Oldtex\TeX
    \let\Oldlatex\LaTeX
    \renewcommand{\TeX}{\textrm{\Oldtex}}
    \renewcommand{\LaTeX}{\textrm{\Oldlatex}}
    % Document parameters
    % Document title
    \title{RecocidoEjemplo}
    
    
    
    
    

    % Pygments definitions
    
\makeatletter
\def\PY@reset{\let\PY@it=\relax \let\PY@bf=\relax%
    \let\PY@ul=\relax \let\PY@tc=\relax%
    \let\PY@bc=\relax \let\PY@ff=\relax}
\def\PY@tok#1{\csname PY@tok@#1\endcsname}
\def\PY@toks#1+{\ifx\relax#1\empty\else%
    \PY@tok{#1}\expandafter\PY@toks\fi}
\def\PY@do#1{\PY@bc{\PY@tc{\PY@ul{%
    \PY@it{\PY@bf{\PY@ff{#1}}}}}}}
\def\PY#1#2{\PY@reset\PY@toks#1+\relax+\PY@do{#2}}

\expandafter\def\csname PY@tok@w\endcsname{\def\PY@tc##1{\textcolor[rgb]{0.73,0.73,0.73}{##1}}}
\expandafter\def\csname PY@tok@c\endcsname{\let\PY@it=\textit\def\PY@tc##1{\textcolor[rgb]{0.25,0.50,0.50}{##1}}}
\expandafter\def\csname PY@tok@cp\endcsname{\def\PY@tc##1{\textcolor[rgb]{0.74,0.48,0.00}{##1}}}
\expandafter\def\csname PY@tok@k\endcsname{\let\PY@bf=\textbf\def\PY@tc##1{\textcolor[rgb]{0.00,0.50,0.00}{##1}}}
\expandafter\def\csname PY@tok@kp\endcsname{\def\PY@tc##1{\textcolor[rgb]{0.00,0.50,0.00}{##1}}}
\expandafter\def\csname PY@tok@kt\endcsname{\def\PY@tc##1{\textcolor[rgb]{0.69,0.00,0.25}{##1}}}
\expandafter\def\csname PY@tok@o\endcsname{\def\PY@tc##1{\textcolor[rgb]{0.40,0.40,0.40}{##1}}}
\expandafter\def\csname PY@tok@ow\endcsname{\let\PY@bf=\textbf\def\PY@tc##1{\textcolor[rgb]{0.67,0.13,1.00}{##1}}}
\expandafter\def\csname PY@tok@nb\endcsname{\def\PY@tc##1{\textcolor[rgb]{0.00,0.50,0.00}{##1}}}
\expandafter\def\csname PY@tok@nf\endcsname{\def\PY@tc##1{\textcolor[rgb]{0.00,0.00,1.00}{##1}}}
\expandafter\def\csname PY@tok@nc\endcsname{\let\PY@bf=\textbf\def\PY@tc##1{\textcolor[rgb]{0.00,0.00,1.00}{##1}}}
\expandafter\def\csname PY@tok@nn\endcsname{\let\PY@bf=\textbf\def\PY@tc##1{\textcolor[rgb]{0.00,0.00,1.00}{##1}}}
\expandafter\def\csname PY@tok@ne\endcsname{\let\PY@bf=\textbf\def\PY@tc##1{\textcolor[rgb]{0.82,0.25,0.23}{##1}}}
\expandafter\def\csname PY@tok@nv\endcsname{\def\PY@tc##1{\textcolor[rgb]{0.10,0.09,0.49}{##1}}}
\expandafter\def\csname PY@tok@no\endcsname{\def\PY@tc##1{\textcolor[rgb]{0.53,0.00,0.00}{##1}}}
\expandafter\def\csname PY@tok@nl\endcsname{\def\PY@tc##1{\textcolor[rgb]{0.63,0.63,0.00}{##1}}}
\expandafter\def\csname PY@tok@ni\endcsname{\let\PY@bf=\textbf\def\PY@tc##1{\textcolor[rgb]{0.60,0.60,0.60}{##1}}}
\expandafter\def\csname PY@tok@na\endcsname{\def\PY@tc##1{\textcolor[rgb]{0.49,0.56,0.16}{##1}}}
\expandafter\def\csname PY@tok@nt\endcsname{\let\PY@bf=\textbf\def\PY@tc##1{\textcolor[rgb]{0.00,0.50,0.00}{##1}}}
\expandafter\def\csname PY@tok@nd\endcsname{\def\PY@tc##1{\textcolor[rgb]{0.67,0.13,1.00}{##1}}}
\expandafter\def\csname PY@tok@s\endcsname{\def\PY@tc##1{\textcolor[rgb]{0.73,0.13,0.13}{##1}}}
\expandafter\def\csname PY@tok@sd\endcsname{\let\PY@it=\textit\def\PY@tc##1{\textcolor[rgb]{0.73,0.13,0.13}{##1}}}
\expandafter\def\csname PY@tok@si\endcsname{\let\PY@bf=\textbf\def\PY@tc##1{\textcolor[rgb]{0.73,0.40,0.53}{##1}}}
\expandafter\def\csname PY@tok@se\endcsname{\let\PY@bf=\textbf\def\PY@tc##1{\textcolor[rgb]{0.73,0.40,0.13}{##1}}}
\expandafter\def\csname PY@tok@sr\endcsname{\def\PY@tc##1{\textcolor[rgb]{0.73,0.40,0.53}{##1}}}
\expandafter\def\csname PY@tok@ss\endcsname{\def\PY@tc##1{\textcolor[rgb]{0.10,0.09,0.49}{##1}}}
\expandafter\def\csname PY@tok@sx\endcsname{\def\PY@tc##1{\textcolor[rgb]{0.00,0.50,0.00}{##1}}}
\expandafter\def\csname PY@tok@m\endcsname{\def\PY@tc##1{\textcolor[rgb]{0.40,0.40,0.40}{##1}}}
\expandafter\def\csname PY@tok@gh\endcsname{\let\PY@bf=\textbf\def\PY@tc##1{\textcolor[rgb]{0.00,0.00,0.50}{##1}}}
\expandafter\def\csname PY@tok@gu\endcsname{\let\PY@bf=\textbf\def\PY@tc##1{\textcolor[rgb]{0.50,0.00,0.50}{##1}}}
\expandafter\def\csname PY@tok@gd\endcsname{\def\PY@tc##1{\textcolor[rgb]{0.63,0.00,0.00}{##1}}}
\expandafter\def\csname PY@tok@gi\endcsname{\def\PY@tc##1{\textcolor[rgb]{0.00,0.63,0.00}{##1}}}
\expandafter\def\csname PY@tok@gr\endcsname{\def\PY@tc##1{\textcolor[rgb]{1.00,0.00,0.00}{##1}}}
\expandafter\def\csname PY@tok@ge\endcsname{\let\PY@it=\textit}
\expandafter\def\csname PY@tok@gs\endcsname{\let\PY@bf=\textbf}
\expandafter\def\csname PY@tok@gp\endcsname{\let\PY@bf=\textbf\def\PY@tc##1{\textcolor[rgb]{0.00,0.00,0.50}{##1}}}
\expandafter\def\csname PY@tok@go\endcsname{\def\PY@tc##1{\textcolor[rgb]{0.53,0.53,0.53}{##1}}}
\expandafter\def\csname PY@tok@gt\endcsname{\def\PY@tc##1{\textcolor[rgb]{0.00,0.27,0.87}{##1}}}
\expandafter\def\csname PY@tok@err\endcsname{\def\PY@bc##1{\setlength{\fboxsep}{0pt}\fcolorbox[rgb]{1.00,0.00,0.00}{1,1,1}{\strut ##1}}}
\expandafter\def\csname PY@tok@kc\endcsname{\let\PY@bf=\textbf\def\PY@tc##1{\textcolor[rgb]{0.00,0.50,0.00}{##1}}}
\expandafter\def\csname PY@tok@kd\endcsname{\let\PY@bf=\textbf\def\PY@tc##1{\textcolor[rgb]{0.00,0.50,0.00}{##1}}}
\expandafter\def\csname PY@tok@kn\endcsname{\let\PY@bf=\textbf\def\PY@tc##1{\textcolor[rgb]{0.00,0.50,0.00}{##1}}}
\expandafter\def\csname PY@tok@kr\endcsname{\let\PY@bf=\textbf\def\PY@tc##1{\textcolor[rgb]{0.00,0.50,0.00}{##1}}}
\expandafter\def\csname PY@tok@bp\endcsname{\def\PY@tc##1{\textcolor[rgb]{0.00,0.50,0.00}{##1}}}
\expandafter\def\csname PY@tok@fm\endcsname{\def\PY@tc##1{\textcolor[rgb]{0.00,0.00,1.00}{##1}}}
\expandafter\def\csname PY@tok@vc\endcsname{\def\PY@tc##1{\textcolor[rgb]{0.10,0.09,0.49}{##1}}}
\expandafter\def\csname PY@tok@vg\endcsname{\def\PY@tc##1{\textcolor[rgb]{0.10,0.09,0.49}{##1}}}
\expandafter\def\csname PY@tok@vi\endcsname{\def\PY@tc##1{\textcolor[rgb]{0.10,0.09,0.49}{##1}}}
\expandafter\def\csname PY@tok@vm\endcsname{\def\PY@tc##1{\textcolor[rgb]{0.10,0.09,0.49}{##1}}}
\expandafter\def\csname PY@tok@sa\endcsname{\def\PY@tc##1{\textcolor[rgb]{0.73,0.13,0.13}{##1}}}
\expandafter\def\csname PY@tok@sb\endcsname{\def\PY@tc##1{\textcolor[rgb]{0.73,0.13,0.13}{##1}}}
\expandafter\def\csname PY@tok@sc\endcsname{\def\PY@tc##1{\textcolor[rgb]{0.73,0.13,0.13}{##1}}}
\expandafter\def\csname PY@tok@dl\endcsname{\def\PY@tc##1{\textcolor[rgb]{0.73,0.13,0.13}{##1}}}
\expandafter\def\csname PY@tok@s2\endcsname{\def\PY@tc##1{\textcolor[rgb]{0.73,0.13,0.13}{##1}}}
\expandafter\def\csname PY@tok@sh\endcsname{\def\PY@tc##1{\textcolor[rgb]{0.73,0.13,0.13}{##1}}}
\expandafter\def\csname PY@tok@s1\endcsname{\def\PY@tc##1{\textcolor[rgb]{0.73,0.13,0.13}{##1}}}
\expandafter\def\csname PY@tok@mb\endcsname{\def\PY@tc##1{\textcolor[rgb]{0.40,0.40,0.40}{##1}}}
\expandafter\def\csname PY@tok@mf\endcsname{\def\PY@tc##1{\textcolor[rgb]{0.40,0.40,0.40}{##1}}}
\expandafter\def\csname PY@tok@mh\endcsname{\def\PY@tc##1{\textcolor[rgb]{0.40,0.40,0.40}{##1}}}
\expandafter\def\csname PY@tok@mi\endcsname{\def\PY@tc##1{\textcolor[rgb]{0.40,0.40,0.40}{##1}}}
\expandafter\def\csname PY@tok@il\endcsname{\def\PY@tc##1{\textcolor[rgb]{0.40,0.40,0.40}{##1}}}
\expandafter\def\csname PY@tok@mo\endcsname{\def\PY@tc##1{\textcolor[rgb]{0.40,0.40,0.40}{##1}}}
\expandafter\def\csname PY@tok@ch\endcsname{\let\PY@it=\textit\def\PY@tc##1{\textcolor[rgb]{0.25,0.50,0.50}{##1}}}
\expandafter\def\csname PY@tok@cm\endcsname{\let\PY@it=\textit\def\PY@tc##1{\textcolor[rgb]{0.25,0.50,0.50}{##1}}}
\expandafter\def\csname PY@tok@cpf\endcsname{\let\PY@it=\textit\def\PY@tc##1{\textcolor[rgb]{0.25,0.50,0.50}{##1}}}
\expandafter\def\csname PY@tok@c1\endcsname{\let\PY@it=\textit\def\PY@tc##1{\textcolor[rgb]{0.25,0.50,0.50}{##1}}}
\expandafter\def\csname PY@tok@cs\endcsname{\let\PY@it=\textit\def\PY@tc##1{\textcolor[rgb]{0.25,0.50,0.50}{##1}}}

\def\PYZbs{\char`\\}
\def\PYZus{\char`\_}
\def\PYZob{\char`\{}
\def\PYZcb{\char`\}}
\def\PYZca{\char`\^}
\def\PYZam{\char`\&}
\def\PYZlt{\char`\<}
\def\PYZgt{\char`\>}
\def\PYZsh{\char`\#}
\def\PYZpc{\char`\%}
\def\PYZdl{\char`\$}
\def\PYZhy{\char`\-}
\def\PYZsq{\char`\'}
\def\PYZdq{\char`\"}
\def\PYZti{\char`\~}
% for compatibility with earlier versions
\def\PYZat{@}
\def\PYZlb{[}
\def\PYZrb{]}
\makeatother


    % Exact colors from NB
    \definecolor{incolor}{rgb}{0.0, 0.0, 0.5}
    \definecolor{outcolor}{rgb}{0.545, 0.0, 0.0}



    
    % Prevent overflowing lines due to hard-to-break entities
    \sloppy 
    % Setup hyperref package
    \hypersetup{
      breaklinks=true,  % so long urls are correctly broken across lines
      colorlinks=true,
      urlcolor=urlcolor,
      linkcolor=linkcolor,
      citecolor=citecolor,
      }
    % Slightly bigger margins than the latex defaults
    
    \geometry{verbose,tmargin=1in,bmargin=1in,lmargin=1in,rmargin=1in}
    
    

    \begin{document}
    
    
    \maketitle
    
    

    
    \section{Recocido Simulado}\label{recocido-simulado}

La librería \textbf{Pyristic} incluye una clase llamada
\emph{SimulatedAnnealing} que facilita la implementación de la
metaheurística basada en \emph{Recocido Simulado}. Para poder utilizar
esta clase, se debe realizar lo siguiente:

\begin{enumerate}
\def\labelenumi{\arabic{enumi}.}
\item
  Definir:

  \begin{itemize}
  \tightlist
  \item
    La función objetivo \(f\).
  \item
    La lista de restricciones.
  \end{itemize}
\item
  Crear una clase que herede de \emph{SimulatedAnnealing}.
\item
  Sobreescribir las siguientes funciones de la clase SimulatedAnnealing:

  \begin{itemize}
  \tightlist
  \item
    get\_neighbor (requerido)
  \item
    update\_temperature (opcional)
  \end{itemize}
\end{enumerate}

A continuación se muestran los elementos que se deben importar.
Posteriormente, se resolverán dos problemas de optimización combinatoria
usando la clase \emph{SimulatedAnnealing}.

    \begin{Verbatim}[commandchars=\\\{\}]
{\color{incolor}In [{\color{incolor}1}]:} \PY{k+kn}{import} \PY{n+nn}{sys}
        \PY{k+kn}{import} \PY{n+nn}{os}
        
        \PY{n}{library\PYZus{}path} \PY{o}{=} \PY{l+s+s2}{\PYZdq{}}\PY{l+s+s2}{/home/dell/Documentos/Git\PYZus{}proejcts/optimizacion\PYZhy{}con\PYZhy{}metaheuristicas/}\PY{l+s+s2}{\PYZdq{}}
        \PY{c+c1}{\PYZsh{}library\PYZus{}path = \PYZdq{}/Users/adrianamenchacamendez/Documentos/enes\PYZus{}morelia/papime/optimizacion\PYZhy{}con\PYZhy{}metaheuristicas/\PYZdq{}}
        \PY{n}{sys}\PY{o}{.}\PY{n}{path}\PY{o}{.}\PY{n}{append}\PY{p}{(}\PY{n}{os}\PY{o}{.}\PY{n}{path}\PY{o}{.}\PY{n}{abspath}\PY{p}{(}\PY{n}{library\PYZus{}path}\PY{p}{)}\PY{p}{)}
\end{Verbatim}

    \begin{Verbatim}[commandchars=\\\{\}]
{\color{incolor}In [{\color{incolor}2}]:} \PY{k+kn}{from} \PY{n+nn}{optimpy}\PY{n+nn}{.}\PY{n+nn}{heuristic}\PY{n+nn}{.}\PY{n+nn}{SimulatedAnnealing\PYZus{}search} \PY{k}{import} \PY{n}{SimulatedAnnealing}
        \PY{k+kn}{from} \PY{n+nn}{optimpy}\PY{n+nn}{.}\PY{n+nn}{utils}\PY{n+nn}{.}\PY{n+nn}{helpers} \PY{k}{import} \PY{o}{*}
        \PY{k+kn}{from} \PY{n+nn}{pprint} \PY{k}{import} \PY{n}{pprint}
        \PY{k+kn}{import} \PY{n+nn}{numpy} \PY{k}{as} \PY{n+nn}{np} 
        \PY{k+kn}{import} \PY{n+nn}{copy} 
\end{Verbatim}

    \subsection{\texorpdfstring{Clase
\texttt{SimulatedAnnealing}}{Clase SimulatedAnnealing}}\label{clase-simulatedannealing}

\paragraph{Variables}\label{variables}

\begin{itemize}
\tightlist
\item
  \emph{\textbf{logger.}} Diccionario con información relacionada a la
  búsqueda con las siguientes llaves:
\item
  \texttt{best\_individual.} Mejor individuo encontrado.
\item
  \texttt{best\_f.} El valor obtenido de la función objetivo de
  \texttt{individual}.
\item
  \texttt{temperature.} Temperatura inicial que se actualizará cada
  iteración.
\item
  \emph{\textbf{f.}} Función objetivo.
\item
  \emph{\textbf{Constraints.}} Lista de restricciones del problema. Las
  restricciones deben ser funciones que retornan True o False, indicando
  si cumple dicha restricción.
\end{itemize}

\paragraph{Métodos}\label{muxe9todos}

\begin{itemize}
\tightlist
\item
  \emph{\textbf{\_\_init\_\_.}} Inicializa la clase.
\end{itemize}

Argumentos: * \texttt{function.} Función objetivo. *
\texttt{constraints.} Lista con las restricciones del problema.

Valor de retorno: * Ninguno.

\begin{itemize}
\tightlist
\item
  \emph{\textbf{optimize.}} método principal, realiza la ejecución
  empleando la metaheurística llamada \texttt{SimulatedAnnealing}.
\end{itemize}

Argumentos: * \texttt{Init.} Solución inicial, se admite un arreglo de
\emph{numpy} o una función que retorne un arreglo de \emph{numpy}. *
\texttt{IniTemperature.} Valor de punto flotante que indica con que
temperatura inicia la búsqueda. * \texttt{eps.} Valor de punto flotante
que indica con que temperatura termina la búsqueda. * \texttt{**kwargs.}
Parámetros externos a la búsqueda.

Valor de retorno: * Ninguno

\begin{itemize}
\tightlist
\item
  \emph{\textbf{get\_neighbor.}} Genera una solución realizando una
  variación aleatoria en la solución actual.
\end{itemize}

Argumentos: * \texttt{x.} Arreglo de \emph{numpy} representando a la
solución actual. * \texttt{**kwargs} Parámetros externos a la búsqueda.

Valor de retorno: * Arreglo de \emph{numpy} representando la solución
generada.

\begin{itemize}
\tightlist
\item
  \emph{\textbf{update\_temperature.}} Función que decrementa la
  temperatura.
\end{itemize}

Argumentos: * \texttt{**kwargs} Parámetros externos a la búsqueda.

Valor de retorno: * La nueva temperatura.

    \subsection{Problema del agente
viajero}\label{problema-del-agente-viajero}

\begin{equation}
    \label{eq:TSP}
    \begin{array}{rll}
    \text{minimizar:} & f(x) = d(x_n, x_1) + \sum_{i=1}^{n-1} d(x_i, x_{i+1}) &  \\
    \text{tal que: } & x_i \in \{1,2,\cdots,n\} & \\
    \end{array}
\end{equation}

Donde: * \(d(x_i,x_j)\) es la distancia de la ciudad \(x_i\) a la ciudad
\(x_j\). * \(n\) es el número de ciudades. * \(x\) es una permutación de
las \(n\) ciudades.

A continuación vamos a definir una instancia de este problema utilizando
10 ciudades.

    \begin{Verbatim}[commandchars=\\\{\}]
{\color{incolor}In [{\color{incolor}3}]:} \PY{k+kn}{import} \PY{n+nn}{random} 
        \PY{k+kn}{import} \PY{n+nn}{math}
        
        \PY{n}{num\PYZus{}cities} \PY{o}{=} \PY{l+m+mi}{10}
        \PY{n}{dist\PYZus{}matrix} \PY{o}{=} \PYZbs{}
        \PY{p}{[}\PYZbs{}
        \PY{p}{[}\PY{l+m+mi}{0}\PY{p}{,}\PY{l+m+mi}{49}\PY{p}{,}\PY{l+m+mi}{30}\PY{p}{,}\PY{l+m+mi}{53}\PY{p}{,}\PY{l+m+mi}{72}\PY{p}{,}\PY{l+m+mi}{19}\PY{p}{,}\PY{l+m+mi}{76}\PY{p}{,}\PY{l+m+mi}{87}\PY{p}{,}\PY{l+m+mi}{45}\PY{p}{,}\PY{l+m+mi}{48}\PY{p}{]}\PY{p}{,}\PYZbs{}
        \PY{p}{[}\PY{l+m+mi}{49}\PY{p}{,}\PY{l+m+mi}{0}\PY{p}{,}\PY{l+m+mi}{19}\PY{p}{,}\PY{l+m+mi}{38}\PY{p}{,}\PY{l+m+mi}{32}\PY{p}{,}\PY{l+m+mi}{31}\PY{p}{,}\PY{l+m+mi}{75}\PY{p}{,}\PY{l+m+mi}{69}\PY{p}{,}\PY{l+m+mi}{61}\PY{p}{,}\PY{l+m+mi}{25}\PY{p}{]}\PY{p}{,}\PYZbs{}
        \PY{p}{[}\PY{l+m+mi}{30}\PY{p}{,}\PY{l+m+mi}{19}\PY{p}{,}\PY{l+m+mi}{0}\PY{p}{,}\PY{l+m+mi}{41}\PY{p}{,}\PY{l+m+mi}{98}\PY{p}{,}\PY{l+m+mi}{56}\PY{p}{,}\PY{l+m+mi}{6}\PY{p}{,}\PY{l+m+mi}{6}\PY{p}{,}\PY{l+m+mi}{45}\PY{p}{,}\PY{l+m+mi}{53}\PY{p}{]}\PY{p}{,}\PYZbs{}
        \PY{p}{[}\PY{l+m+mi}{53}\PY{p}{,}\PY{l+m+mi}{38}\PY{p}{,}\PY{l+m+mi}{41}\PY{p}{,}\PY{l+m+mi}{0}\PY{p}{,}\PY{l+m+mi}{52}\PY{p}{,}\PY{l+m+mi}{29}\PY{p}{,}\PY{l+m+mi}{46}\PY{p}{,}\PY{l+m+mi}{90}\PY{p}{,}\PY{l+m+mi}{23}\PY{p}{,}\PY{l+m+mi}{98}\PY{p}{]}\PY{p}{,}\PYZbs{}
        \PY{p}{[}\PY{l+m+mi}{72}\PY{p}{,}\PY{l+m+mi}{32}\PY{p}{,}\PY{l+m+mi}{98}\PY{p}{,}\PY{l+m+mi}{52}\PY{p}{,}\PY{l+m+mi}{0}\PY{p}{,}\PY{l+m+mi}{63}\PY{p}{,}\PY{l+m+mi}{90}\PY{p}{,}\PY{l+m+mi}{69}\PY{p}{,}\PY{l+m+mi}{50}\PY{p}{,}\PY{l+m+mi}{82}\PY{p}{]}\PY{p}{,}\PYZbs{}
        \PY{p}{[}\PY{l+m+mi}{19}\PY{p}{,}\PY{l+m+mi}{31}\PY{p}{,}\PY{l+m+mi}{56}\PY{p}{,}\PY{l+m+mi}{29}\PY{p}{,}\PY{l+m+mi}{63}\PY{p}{,}\PY{l+m+mi}{0}\PY{p}{,}\PY{l+m+mi}{60}\PY{p}{,}\PY{l+m+mi}{88}\PY{p}{,}\PY{l+m+mi}{41}\PY{p}{,}\PY{l+m+mi}{95}\PY{p}{]}\PY{p}{,}\PYZbs{}
        \PY{p}{[}\PY{l+m+mi}{76}\PY{p}{,}\PY{l+m+mi}{75}\PY{p}{,}\PY{l+m+mi}{6}\PY{p}{,}\PY{l+m+mi}{46}\PY{p}{,}\PY{l+m+mi}{90}\PY{p}{,}\PY{l+m+mi}{60}\PY{p}{,}\PY{l+m+mi}{0}\PY{p}{,}\PY{l+m+mi}{61}\PY{p}{,}\PY{l+m+mi}{92}\PY{p}{,}\PY{l+m+mi}{10}\PY{p}{]}\PY{p}{,}\PYZbs{}
        \PY{p}{[}\PY{l+m+mi}{87}\PY{p}{,}\PY{l+m+mi}{69}\PY{p}{,}\PY{l+m+mi}{6}\PY{p}{,}\PY{l+m+mi}{90}\PY{p}{,}\PY{l+m+mi}{69}\PY{p}{,}\PY{l+m+mi}{88}\PY{p}{,}\PY{l+m+mi}{61}\PY{p}{,}\PY{l+m+mi}{0}\PY{p}{,}\PY{l+m+mi}{82}\PY{p}{,}\PY{l+m+mi}{73}\PY{p}{]}\PY{p}{,}\PYZbs{}
        \PY{p}{[}\PY{l+m+mi}{45}\PY{p}{,}\PY{l+m+mi}{61}\PY{p}{,}\PY{l+m+mi}{45}\PY{p}{,}\PY{l+m+mi}{23}\PY{p}{,}\PY{l+m+mi}{50}\PY{p}{,}\PY{l+m+mi}{41}\PY{p}{,}\PY{l+m+mi}{92}\PY{p}{,}\PY{l+m+mi}{82}\PY{p}{,}\PY{l+m+mi}{0}\PY{p}{,}\PY{l+m+mi}{5}\PY{p}{]}\PY{p}{,}\PYZbs{}
        \PY{p}{[}\PY{l+m+mi}{48}\PY{p}{,}\PY{l+m+mi}{25}\PY{p}{,}\PY{l+m+mi}{53}\PY{p}{,}\PY{l+m+mi}{98}\PY{p}{,}\PY{l+m+mi}{82}\PY{p}{,}\PY{l+m+mi}{95}\PY{p}{,}\PY{l+m+mi}{10}\PY{p}{,}\PY{l+m+mi}{73}\PY{p}{,}\PY{l+m+mi}{5}\PY{p}{,}\PY{l+m+mi}{0}\PY{p}{]}\PY{p}{,}\PYZbs{}
        \PY{p}{]}
\end{Verbatim}

    \subsubsection{Función objetivo}\label{funciuxf3n-objetivo}

    \begin{Verbatim}[commandchars=\\\{\}]
{\color{incolor}In [{\color{incolor}4}]:} \PY{n+nd}{@checkargs}
        \PY{k}{def} \PY{n+nf}{f\PYZus{}salesman}\PY{p}{(}\PY{n}{x} \PY{p}{:} \PY{p}{(}\PY{n+nb}{list}\PY{p}{,}\PY{n}{np}\PY{o}{.}\PY{n}{ndarray}\PY{p}{)}\PY{p}{)} \PY{o}{\PYZhy{}}\PY{o}{\PYZgt{}} \PY{n+nb}{float}\PY{p}{:}
            \PY{k}{global} \PY{n}{dist\PYZus{}matrix}
            \PY{n}{total\PYZus{}dist} \PY{o}{=} \PY{n}{dist\PYZus{}matrix}\PY{p}{[}\PY{n}{x}\PY{p}{[}\PY{o}{\PYZhy{}}\PY{l+m+mi}{1}\PY{p}{]}\PY{p}{]}\PY{p}{[}\PY{l+m+mi}{0}\PY{p}{]}
            \PY{k}{for} \PY{n}{i} \PY{o+ow}{in} \PY{n+nb}{range}\PY{p}{(}\PY{l+m+mi}{1}\PY{p}{,}\PY{n+nb}{len}\PY{p}{(}\PY{n}{x}\PY{p}{)}\PY{p}{)}\PY{p}{:}
                \PY{n}{u}\PY{p}{,}\PY{n}{v} \PY{o}{=} \PY{n}{x}\PY{p}{[}\PY{n}{i}\PY{p}{]}\PY{p}{,} \PY{n}{x}\PY{p}{[}\PY{n}{i}\PY{o}{\PYZhy{}}\PY{l+m+mi}{1}\PY{p}{]}
                \PY{n}{total\PYZus{}dist}\PY{o}{+}\PY{o}{=} \PY{n}{dist\PYZus{}matrix}\PY{p}{[}\PY{n}{u}\PY{p}{]}\PY{p}{[}\PY{n}{v}\PY{p}{]}
        
            \PY{k}{return} \PY{n+nb}{float}\PY{p}{(}\PY{n}{total\PYZus{}dist}\PY{p}{)}
\end{Verbatim}

    \subsubsection{Restricciones}\label{restricciones}

Las restricciones se definen como una lista de funciones que retornan
valores \emph{booleanos}. Estos valores permitirán verificar si una
solución es factible o no.

En el caso del problema del agente viajero queremos comprobar que
estamos visitando todas las ciudades exactamente una vez.

    \begin{Verbatim}[commandchars=\\\{\}]
{\color{incolor}In [{\color{incolor}5}]:} \PY{n+nd}{@checkargs}
        \PY{k}{def} \PY{n+nf}{g\PYZus{}salesman}\PY{p}{(}\PY{n}{x} \PY{p}{:} \PY{n}{np}\PY{o}{.}\PY{n}{ndarray}\PY{p}{)} \PY{o}{\PYZhy{}}\PY{o}{\PYZgt{}} \PY{n+nb}{bool}\PY{p}{:}
        
            \PY{n}{size} \PY{o}{=} \PY{n+nb}{len}\PY{p}{(}\PY{n}{x}\PY{p}{)}
            \PY{n}{size\PYZus{}} \PY{o}{=} \PY{n+nb}{len}\PY{p}{(}\PY{n}{np}\PY{o}{.}\PY{n}{unique}\PY{p}{(}\PY{n}{x}\PY{p}{)}\PY{p}{)}
            \PY{k}{return} \PY{n}{size} \PY{o}{==} \PY{n}{size\PYZus{}}
        
        \PY{n}{constraints\PYZus{}list}\PY{o}{=} \PY{p}{[}\PY{n}{g\PYZus{}salesman}\PY{p}{]}
\end{Verbatim}

    \subsubsection{Solución inicial}\label{soluciuxf3n-inicial}

La estrategia utilizada para generar la solución inicial es la
siguiente:

\begin{enumerate}
\def\labelenumi{\arabic{enumi}.}
\item
  Introducir la ciudad \(1\) en la primera posición de nuestra
  permutación y crear un arreglo con las ciudades restantes.
\item
  Seleccionar la ciudad más cercana desde nuestra ubicación actual.
\item
  Retirar del arreglo la ciudad seleccionada y asignar la como nuestra
  ubicación actual; Posterior, repetir el punto 2.
\end{enumerate}

    \begin{Verbatim}[commandchars=\\\{\}]
{\color{incolor}In [{\color{incolor}6}]:} \PY{k}{def} \PY{n+nf}{getInitialSolutionTS}\PY{p}{(}\PY{n}{distance\PYZus{}matrix}\PY{p}{,} \PY{n}{total\PYZus{}cities}\PY{p}{)} \PY{o}{\PYZhy{}}\PY{o}{\PYZgt{}} \PY{n}{np}\PY{o}{.}\PY{n}{ndarray}\PY{p}{:}
            \PY{n}{Solution} \PY{o}{=} \PY{p}{[}\PY{l+m+mi}{0}\PY{p}{]}
            \PY{n}{remaining\PYZus{}cities}  \PY{o}{=} \PY{n+nb}{list}\PY{p}{(}\PY{n+nb}{range}\PY{p}{(}\PY{l+m+mi}{1}\PY{p}{,}\PY{n}{total\PYZus{}cities}\PY{p}{)}\PY{p}{)}
            
            \PY{k}{while} \PY{n+nb}{len}\PY{p}{(}\PY{n}{remaining\PYZus{}cities}\PY{p}{)} \PY{o}{!=} \PY{l+m+mi}{0}\PY{p}{:}
                \PY{n}{from\PYZus{}} \PY{o}{=} \PY{n}{Solution}\PY{p}{[}\PY{o}{\PYZhy{}}\PY{l+m+mi}{1}\PY{p}{]}
                \PY{n}{to\PYZus{}} \PY{o}{=} \PY{n}{remaining\PYZus{}cities}\PY{p}{[}\PY{l+m+mi}{0}\PY{p}{]}
                \PY{n}{dist} \PY{o}{=} \PY{n}{distance\PYZus{}matrix}\PY{p}{[}\PY{n}{from\PYZus{}}\PY{p}{]}\PY{p}{[}\PY{n}{to\PYZus{}}\PY{p}{]}
                
                \PY{k}{for} \PY{n}{i} \PY{o+ow}{in} \PY{n+nb}{range}\PY{p}{(}\PY{l+m+mi}{1}\PY{p}{,} \PY{n+nb}{len}\PY{p}{(}\PY{n}{remaining\PYZus{}cities}\PY{p}{)}\PY{p}{)}\PY{p}{:}
                    \PY{n}{distance} \PY{o}{=} \PY{n}{distance\PYZus{}matrix}\PY{p}{[}\PY{n}{from\PYZus{}}\PY{p}{]}\PY{p}{[}\PY{n}{remaining\PYZus{}cities}\PY{p}{[}\PY{n}{i}\PY{p}{]}\PY{p}{]}
                    \PY{k}{if} \PY{n}{distance} \PY{o}{\PYZlt{}} \PY{n}{dist}\PY{p}{:}
                        \PY{n}{to\PYZus{}} \PY{o}{=} \PY{n}{remaining\PYZus{}cities}\PY{p}{[}\PY{n}{i}\PY{p}{]}
                        \PY{n}{dist} \PY{o}{=} \PY{n}{distance}
                \PY{n}{Solution}\PY{o}{.}\PY{n}{append}\PY{p}{(}\PY{n}{to\PYZus{}}\PY{p}{)}
                \PY{n}{ind} \PY{o}{=} \PY{n}{remaining\PYZus{}cities}\PY{o}{.}\PY{n}{index}\PY{p}{(}\PY{n}{to\PYZus{}}\PY{p}{)}
                \PY{n}{remaining\PYZus{}cities}\PY{o}{.}\PY{n}{pop}\PY{p}{(}\PY{n}{ind}\PY{p}{)}
                
            \PY{k}{return} \PY{n}{np}\PY{o}{.}\PY{n}{array}\PY{p}{(}\PY{n}{Solution}\PY{p}{)}
\end{Verbatim}

    \begin{Verbatim}[commandchars=\\\{\}]
{\color{incolor}In [{\color{incolor}7}]:} \PY{n}{Path} \PY{o}{=} \PY{n}{getInitialSolutionTS}\PY{p}{(}\PY{n}{dist\PYZus{}matrix}\PY{p}{,}\PY{n}{num\PYZus{}cities}\PY{p}{)}
        \PY{n+nb}{print}\PY{p}{(}\PY{n}{Path}\PY{p}{)}
\end{Verbatim}

    \begin{Verbatim}[commandchars=\\\{\}]
[0 5 3 8 9 6 2 7 1 4]

    \end{Verbatim}

    \subsubsection{\texorpdfstring{Declaración de
\texttt{SimulatedAnnealing}}{Declaración de SimulatedAnnealing}}\label{declaraciuxf3n-de-simulatedannealing}

Para implementar una metaheurística basada en recocido simulado,
utilizando la librería \textbf{Pyristic}, es necesario crear una clase
que herede de la clase \texttt{SimulatedAnnealing}. En este ejemplo, la
nueva clase es llamada \emph{TravellingSalesman\_solver}.

La nueva clase debe sobreescribir la función \texttt{get\_neighbor}, de
lo contrario el algoritmo no va a funcionar.

    \begin{Verbatim}[commandchars=\\\{\}]
{\color{incolor}In [{\color{incolor}8}]:} \PY{k}{class} \PY{n+nc}{TravellingSalesman\PYZus{}solver}\PY{p}{(}\PY{n}{SimulatedAnnealing}\PY{p}{)}\PY{p}{:}
        
            \PY{n+nd}{@checkargs}
            \PY{k}{def} \PY{n+nf}{\PYZus{}\PYZus{}init\PYZus{}\PYZus{}}\PY{p}{(}\PY{n+nb+bp}{self}\PY{p}{,} \PY{n}{f\PYZus{}} \PY{p}{:} \PY{n}{function\PYZus{}type} \PY{p}{,} \PY{n}{constraints\PYZus{}}\PY{p}{:} \PY{n+nb}{list}\PY{p}{)}\PY{p}{:}
                \PY{n+nb}{super}\PY{p}{(}\PY{p}{)}\PY{o}{.}\PY{n+nf+fm}{\PYZus{}\PYZus{}init\PYZus{}\PYZus{}}\PY{p}{(}\PY{n}{f\PYZus{}}\PY{p}{,}\PY{n}{constraints\PYZus{}}\PY{p}{)}
                
            \PY{n+nd}{@checkargs}
            \PY{k}{def} \PY{n+nf}{get\PYZus{}neighbor}\PY{p}{(}\PY{n+nb+bp}{self}\PY{p}{,} \PY{n}{x} \PY{p}{:} \PY{n}{np}\PY{o}{.}\PY{n}{ndarray}\PY{p}{)} \PY{o}{\PYZhy{}}\PY{o}{\PYZgt{}} \PY{n}{np}\PY{o}{.}\PY{n}{ndarray}\PY{p}{:} 
                
                \PY{n}{x\PYZus{}} \PY{o}{=} \PY{n}{x}\PY{o}{.}\PY{n}{copy}\PY{p}{(}\PY{p}{)}
                \PY{n}{N} \PY{o}{=} \PY{n+nb}{len}\PY{p}{(}\PY{n}{x\PYZus{}}\PY{p}{)}
                \PY{n}{index1} \PY{o}{=} \PY{n}{random}\PY{o}{.}\PY{n}{randint}\PY{p}{(}\PY{l+m+mi}{1}\PY{p}{,} \PY{n}{N}\PY{o}{\PYZhy{}}\PY{l+m+mi}{1}\PY{p}{)}
                \PY{n}{index2} \PY{o}{=} \PY{n}{random}\PY{o}{.}\PY{n}{randint}\PY{p}{(}\PY{l+m+mi}{1}\PY{p}{,} \PY{n}{N}\PY{o}{\PYZhy{}}\PY{l+m+mi}{1}\PY{p}{)}
            
                \PY{k}{while} \PY{n}{index2} \PY{o}{==} \PY{n}{index1}\PY{p}{:}
                    \PY{n}{index2} \PY{o}{=} \PY{n}{random}\PY{o}{.}\PY{n}{randint}\PY{p}{(}\PY{l+m+mi}{1}\PY{p}{,} \PY{n}{N}\PY{o}{\PYZhy{}}\PY{l+m+mi}{1}\PY{p}{)}
            
                \PY{n}{v} \PY{o}{=} \PY{n}{x}\PY{p}{[}\PY{n}{index1}\PY{p}{]}
                \PY{n}{x\PYZus{}} \PY{o}{=} \PY{n+nb}{list}\PY{p}{(}\PY{n}{x\PYZus{}}\PY{p}{[}\PY{n}{v} \PY{o}{!=} \PY{n}{x\PYZus{}}\PY{p}{]}\PY{p}{)}
                \PY{n}{x\PYZus{}} \PY{o}{=} \PY{n}{x\PYZus{}}\PY{p}{[}\PY{p}{:}\PY{n}{index2}\PY{p}{]} \PY{o}{+} \PY{p}{[}\PY{n}{v}\PY{p}{]} \PY{o}{+} \PY{n}{x\PYZus{}}\PY{p}{[}\PY{n}{index2}\PY{p}{:}\PY{p}{]}
                \PY{k}{return} \PY{n}{np}\PY{o}{.}\PY{n}{array}\PY{p}{(}\PY{n}{x\PYZus{}}\PY{p}{)}
\end{Verbatim}

    La función \texttt{get\_neigbor} debe regresar una solución
\emph{vecina} de la solución actual, es decir, una variación de la
solución actual. Para nuestro ejemplo, una solución vecina de la
solución \(x\), la vamos a definir como sigue:

Se seleccionan dos indices distintos de manera aleatoria llamados \(i\)
y \(j\), donde, tomaremos el elemento \(x_i\) y desplazaremos las otras
posiciones de la solución \(x\) de modo que \(x_i\) se encuentre en la
posición \(j\) y esta nueva solución será retornada.

    \subsubsection{Ejecución de la
metaheurística}\label{ejecuciuxf3n-de-la-metaheuruxedstica}

Una vez definida la clase \emph{TravellingSalesman\_solver}, se crea una
instancia indicando en los parámetros la función objetivo y las
restricciones del problema. En este caso llamamos
\emph{TravellingSalesman} a la instancia creada.

    \begin{Verbatim}[commandchars=\\\{\}]
{\color{incolor}In [{\color{incolor}9}]:} \PY{n}{TravellingSalesman} \PY{o}{=}\PYZbs{}
        \PY{n}{TravellingSalesman\PYZus{}solver}\PY{p}{(} \PY{n}{f\PYZus{}salesman}\PY{p}{,} \PY{n}{constraints\PYZus{}list}\PY{p}{)}
\end{Verbatim}

    Finalmente, se llama la función \texttt{optimize} (esta función es la
misma para todas las clases en la librería). La función
\texttt{optimize} recibe tres parámetros:

\begin{itemize}
\tightlist
\item
  Solución inicial o función generadora de soluciones iniciales.
\item
  La temperatura inicial.
\item
  La temperatura final.
\end{itemize}

Vamos a utilizar los siguientes parámetros: * Emplearemos la solución
obtenida por la función \emph{getInitialSolutionTS}. * \(1000\) de
temperatura inicial. * \(0.1\) de temperatura final.

    \begin{Verbatim}[commandchars=\\\{\}]
{\color{incolor}In [{\color{incolor}10}]:} \PY{n}{TravellingSalesman}\PY{o}{.}\PY{n}{optimize}\PY{p}{(}\PY{n}{Path}\PY{p}{,} \PY{l+m+mf}{1000.0} \PY{p}{,} \PY{l+m+mf}{0.1}\PY{p}{)}
         \PY{n+nb}{print}\PY{p}{(}\PY{n}{TravellingSalesman}\PY{p}{)}
\end{Verbatim}

    \begin{Verbatim}[commandchars=\\\{\}]
Simulated Annealing: 
 f(X) = 271.0 
 X = [0 5 3 8 9 6 2 7 1 4] 
 

    \end{Verbatim}

    Para revisar el comportamiento de la metaheurística en determinado
problema, la librería \textbf{Pyristic} cuenta con una función llamada
\texttt{get\_stats}. Esta función se encuentra en \textbf{utils.helpers}
y recibe como parámetros:

\begin{itemize}
\tightlist
\item
  El objeto creado para ejecutar la metaheurística.
\item
  El número de veces que se quiere ejecutar la metaheurística.
\item
  Los argumentos que recibe la función \texttt{optimize} deben estar
  contenidos en una tupla.
\end{itemize}

La función \textbf{get\_stats} retorna un diccionario con algunas
estadísticas de las ejecuciones.

    \begin{Verbatim}[commandchars=\\\{\}]
{\color{incolor}In [{\color{incolor}11}]:} \PY{c+c1}{\PYZsh{}Ejecutamos get\PYZus{}stats 30 veces.}
         \PY{n}{args} \PY{o}{=} \PY{p}{(}\PY{n}{Path}\PY{p}{,} \PY{l+m+mf}{1000.0}\PY{p}{,} \PY{l+m+mf}{0.01}\PY{p}{)}
         \PY{n}{statistics} \PY{o}{=} \PY{n}{get\PYZus{}stats}\PY{p}{(}\PY{n}{TravellingSalesman}\PY{p}{,} \PY{l+m+mi}{30}\PY{p}{,} \PY{n}{args}\PY{p}{)}
\end{Verbatim}

    \begin{Verbatim}[commandchars=\\\{\}]
{\color{incolor}In [{\color{incolor}12}]:} \PY{n}{pprint}\PY{p}{(}\PY{n}{statistics}\PY{p}{)}
\end{Verbatim}

    \begin{Verbatim}[commandchars=\\\{\}]
\{'Best solution': \{'f': 248.0, 'x': array([0, 5, 3, 8, 9, 6, 2, 7, 4, 1])\},
 'Mean': 258.1333333333333,
 'Standard deviation': 11.035498277025022,
 'Worst solution': \{'f': 271.0, 'x': array([0, 5, 3, 8, 9, 6, 2, 7, 1, 4])\}\}

    \end{Verbatim}

    \subsection{Problema de la mochila}\label{problema-de-la-mochila}

\begin{equation}
  \label{eq:KP}
  \begin{array}{rll}
  \text{maximizar:} & f(\vec{x}) = \sum_{i=1}^{n} p_i \cdot x_{i} &  \\
  \text{donde: } & g_1(\vec{x}) = \sum_{i=1}^{n} w_i \cdot x_{i}  \leq c &  \\
          &  x_i \in \{0,1\} & i\in\{1,\ldots,n\}\\
  \end{array}
\end{equation}

    Para este problema vamos a crear una instancia de \(1000\) objetos,
donde, cada objeto estará definido: * \(p_{i} \in [50,100]\) *
\(w_{i} \in [5,100]\) * \(C = 9786\)

    \begin{Verbatim}[commandchars=\\\{\}]
{\color{incolor}In [{\color{incolor}13}]:} \PY{n}{n} \PY{o}{=} \PY{l+m+mi}{1000}
         \PY{n}{p} \PY{o}{=} \PY{n}{np}\PY{o}{.}\PY{n}{random}\PY{o}{.}\PY{n}{randint}\PY{p}{(}\PY{l+m+mi}{50}\PY{p}{,}\PY{l+m+mi}{101}\PY{p}{,}\PY{n}{n}\PY{p}{)}
         \PY{n}{w} \PY{o}{=} \PY{n}{np}\PY{o}{.}\PY{n}{random}\PY{o}{.}\PY{n}{randint}\PY{p}{(}\PY{l+m+mi}{5}\PY{p}{,}\PY{l+m+mi}{101}\PY{p}{,}\PY{n}{n}\PY{p}{)}
         \PY{n}{c} \PY{o}{=} \PY{l+m+mi}{9786}
\end{Verbatim}

    A continuación mostraremos dos algorimos basados en recocido simulado.
El primero de ellos es un diseño sencillo y el segundo es un diseño más
elaborado que obtiene mejores resultados.

    \subsubsection{Función objetivo}\label{funciuxf3n-objetivo}

Dado que la clase \texttt{SimulatedAnnealing} considera problemas de
minimización, es necesario convertir el problema de la mochila a un
problema de minimización. Para esto se multiplica el valor de la función
objetivo por -1.

    \begin{Verbatim}[commandchars=\\\{\}]
{\color{incolor}In [{\color{incolor}14}]:} \PY{n+nd}{@checkargs}
         \PY{k}{def} \PY{n+nf}{f}\PY{p}{(}\PY{n}{x} \PY{p}{:} \PY{n}{np}\PY{o}{.}\PY{n}{ndarray}\PY{p}{)} \PY{o}{\PYZhy{}}\PY{o}{\PYZgt{}} \PY{n+nb}{float}\PY{p}{:}
             \PY{k}{global} \PY{n}{p}
             \PY{k}{return} \PY{o}{\PYZhy{}}\PY{l+m+mf}{1.0}\PY{o}{*} \PY{n}{np}\PY{o}{.}\PY{n}{dot}\PY{p}{(}\PY{n}{x}\PY{p}{,}\PY{n}{p}\PY{p}{)}
\end{Verbatim}

    \subsubsection{Restricciones}\label{restricciones}

La restricción del problema de la mochila es seleccionar un número de
objetos sin exceder la capacidad de la mochila.

    \begin{Verbatim}[commandchars=\\\{\}]
{\color{incolor}In [{\color{incolor}15}]:} \PY{n+nd}{@checkargs} 
         \PY{k}{def} \PY{n+nf}{g1}\PY{p}{(}\PY{n}{x} \PY{p}{:} \PY{n}{np}\PY{o}{.}\PY{n}{ndarray}\PY{p}{)} \PY{o}{\PYZhy{}}\PY{o}{\PYZgt{}} \PY{n+nb}{bool}\PY{p}{:}
             \PY{k}{global} \PY{n}{w}\PY{p}{,}\PY{n}{c}
             \PY{n}{result} \PY{o}{=} \PY{n}{np}\PY{o}{.}\PY{n}{dot}\PY{p}{(}\PY{n}{x}\PY{p}{,}\PY{n}{w}\PY{p}{)}
             \PY{n}{g1}\PY{o}{.}\PY{n+nv+vm}{\PYZus{}\PYZus{}doc\PYZus{}\PYZus{}}\PY{o}{=}\PY{l+s+s2}{\PYZdq{}}\PY{l+s+si}{\PYZob{}\PYZcb{}}\PY{l+s+s2}{ \PYZlt{}= }\PY{l+s+si}{\PYZob{}\PYZcb{}}\PY{l+s+s2}{\PYZdq{}}\PY{o}{.}\PY{n}{format}\PY{p}{(}\PY{n}{result}\PY{p}{,}\PY{n}{c}\PY{p}{)}
             \PY{k}{return} \PY{n}{result} \PY{o}{\PYZlt{}}\PY{o}{=} \PY{n}{c}
         
         \PY{n}{constraints\PYZus{}list}\PY{o}{=} \PY{p}{[}\PY{n}{g1}\PY{p}{]}
\end{Verbatim}

    \subsubsection{Solución inicial}\label{soluciuxf3n-inicial}

Nuestra solución inicial es creada introduciendo objetos de manera
aleatoria, mientras no se exceda el peso de la mochila.

    \begin{Verbatim}[commandchars=\\\{\}]
{\color{incolor}In [{\color{incolor}16}]:} \PY{k}{def} \PY{n+nf}{getInitialSolution}\PY{p}{(}\PY{n}{NumObjects}\PY{o}{=}\PY{l+m+mi}{15}\PY{p}{)}\PY{p}{:}
             \PY{k}{global} \PY{n}{n}\PY{p}{,}\PY{n}{p}\PY{p}{,}\PY{n}{w}\PY{p}{,}\PY{n}{c}
             \PY{c+c1}{\PYZsh{}Empty backpack}
             \PY{n}{x} \PY{o}{=} \PY{p}{[}\PY{l+m+mi}{0} \PY{k}{for} \PY{n}{i} \PY{o+ow}{in} \PY{n+nb}{range}\PY{p}{(}\PY{n}{n}\PY{p}{)}\PY{p}{]}
             \PY{n}{weight\PYZus{}x} \PY{o}{=} \PY{l+m+mi}{0}
             
             \PY{c+c1}{\PYZsh{}Random order to insert objects.}
             \PY{n}{objects} \PY{o}{=} \PY{n+nb}{list}\PY{p}{(}\PY{n+nb}{range}\PY{p}{(}\PY{n}{n}\PY{p}{)}\PY{p}{)}
             \PY{n}{np}\PY{o}{.}\PY{n}{random}\PY{o}{.}\PY{n}{shuffle}\PY{p}{(}\PY{n}{objects}\PY{p}{)}
             
             \PY{k}{for} \PY{n}{o} \PY{o+ow}{in}  \PY{n}{objects}\PY{p}{[}\PY{p}{:}\PY{n}{NumObjects}\PY{p}{]}\PY{p}{:}
                 \PY{c+c1}{\PYZsh{}Check the constraint about capacity.}
                 \PY{k}{if} \PY{n}{weight\PYZus{}x} \PY{o}{+} \PY{n}{w}\PY{p}{[}\PY{n}{o}\PY{p}{]} \PY{o}{\PYZlt{}}\PY{o}{=} \PY{n}{c}\PY{p}{:}
                     \PY{n}{x}\PY{p}{[}\PY{n}{o}\PY{p}{]} \PY{o}{=} \PY{l+m+mi}{1}
                     \PY{n}{weight\PYZus{}x} \PY{o}{+}\PY{o}{=} \PY{n}{w}\PY{p}{[}\PY{n}{o}\PY{p}{]}
                     
             \PY{k}{return} \PY{n}{np}\PY{o}{.}\PY{n}{array}\PY{p}{(}\PY{n}{x}\PY{p}{)}
\end{Verbatim}

    \subsubsection{Ejecución de la
metaheurística}\label{ejecuciuxf3n-de-la-metaheuruxedstica}

    \begin{Verbatim}[commandchars=\\\{\}]
{\color{incolor}In [{\color{incolor}17}]:} \PY{k}{class} \PY{n+nc}{Knapsack\PYZus{}solver}\PY{p}{(}\PY{n}{SimulatedAnnealing}\PY{p}{)}\PY{p}{:}
         
             \PY{n+nd}{@checkargs}
             \PY{k}{def} \PY{n+nf}{\PYZus{}\PYZus{}init\PYZus{}\PYZus{}}\PY{p}{(}\PY{n+nb+bp}{self}\PY{p}{,} \PY{n}{f\PYZus{}} \PY{p}{:} \PY{n}{function\PYZus{}type} \PY{p}{,} \PY{n}{constraints\PYZus{}}\PY{p}{:} \PY{n+nb}{list}\PY{p}{)}\PY{p}{:}
                 \PY{n+nb}{super}\PY{p}{(}\PY{p}{)}\PY{o}{.}\PY{n+nf+fm}{\PYZus{}\PYZus{}init\PYZus{}\PYZus{}}\PY{p}{(}\PY{n}{f\PYZus{}}\PY{p}{,}\PY{n}{constraints\PYZus{}}\PY{p}{)}
                 
             \PY{k}{def} \PY{n+nf}{get\PYZus{}neighbor}\PY{p}{(}\PY{n+nb+bp}{self}\PY{p}{,} \PY{n}{x} \PY{p}{:} \PY{n}{np}\PY{o}{.}\PY{n}{ndarray}\PY{p}{,} \PY{o}{*}\PY{o}{*}\PY{n}{kwargs}\PY{p}{)} \PY{o}{\PYZhy{}}\PY{o}{\PYZgt{}} \PY{n}{np}\PY{o}{.}\PY{n}{ndarray}\PY{p}{:} 
                 
                 \PY{n}{x\PYZus{}} \PY{o}{=} \PY{n}{x}\PY{o}{.}\PY{n}{copy}\PY{p}{(}\PY{p}{)}
                 \PY{n}{N} \PY{o}{=} \PY{n+nb}{len}\PY{p}{(}\PY{n}{x\PYZus{}}\PY{p}{)}
                 \PY{k}{while}\PY{p}{(}\PY{k+kc}{True}\PY{p}{)}\PY{p}{:}
                     \PY{n}{ind} \PY{o}{=} \PY{n}{random}\PY{o}{.}\PY{n}{randint}\PY{p}{(}\PY{l+m+mi}{0}\PY{p}{,} \PY{n}{N}\PY{o}{\PYZhy{}}\PY{l+m+mi}{1}\PY{p}{)}
                     \PY{n}{x\PYZus{}}\PY{p}{[}\PY{n}{ind}\PY{p}{]} \PY{o}{\PYZca{}}\PY{o}{=} \PY{l+m+mi}{1}
                     
                     \PY{k}{if}\PY{p}{(}\PY{n+nb+bp}{self}\PY{o}{.}\PY{n}{is\PYZus{}valid}\PY{p}{(}\PY{n}{x\PYZus{}}\PY{p}{)}\PY{p}{)}\PY{p}{:}
                         \PY{k}{break}
                     
                     \PY{n}{x\PYZus{}}\PY{p}{[}\PY{n}{ind}\PY{p}{]} \PY{o}{\PYZca{}}\PY{o}{=} \PY{l+m+mi}{1}
                 
                 \PY{k}{return} \PY{n}{np}\PY{o}{.}\PY{n}{array}\PY{p}{(}\PY{n}{x\PYZus{}}\PY{p}{)}
         
             \PY{k}{def} \PY{n+nf}{update\PYZus{}temperature}\PY{p}{(}\PY{n+nb+bp}{self}\PY{p}{,} \PY{o}{*}\PY{o}{*}\PY{n}{kwargs}\PY{p}{)} \PY{o}{\PYZhy{}}\PY{o}{\PYZgt{}} \PY{n+nb}{float}\PY{p}{:}
                 \PY{k}{return} \PY{n+nb+bp}{self}\PY{o}{.}\PY{n}{logger}\PY{p}{[}\PY{l+s+s1}{\PYZsq{}}\PY{l+s+s1}{temperature}\PY{l+s+s1}{\PYZsq{}}\PY{p}{]} \PY{o}{*} \PY{l+m+mf}{0.99}
\end{Verbatim}

    Parte clave de este algoritmo es la temperatura. La temperatura es una
función que varía de acuerdo al tiempo \(t\), donde, al inicio de la
búsqueda permite aceptar con mayor probabilidad soluciones peores.

La clase \texttt{SimulatedAnnealing} define esta función como
\texttt{update\_temperature} que está implementada por defecto. En este
ejemplo vamos a definir nuestra función de cambio lineal
\(T(t + 1) = \sigma T(t)\)

    La función \texttt{get\_neighbor} tomará la solución \(x\) (arreglo de
0's y 1's), donde, el valor de retorno será la solución \(x\) con una
posición aleatoria \(i\) que será reemplazada por el valor 0 si la
posición \(x_i\) es 1, sino, será 1.

    \begin{Verbatim}[commandchars=\\\{\}]
{\color{incolor}In [{\color{incolor}18}]:} \PY{n}{Knapsack\PYZus{}} \PY{o}{=} \PY{n}{Knapsack\PYZus{}solver}\PY{p}{(}\PY{n}{f}\PY{p}{,} \PY{p}{[}\PY{n}{g1}\PY{p}{]}\PY{p}{)}
         \PY{n}{Knapsack\PYZus{}}\PY{o}{.}\PY{n}{optimize}\PY{p}{(}\PY{n}{getInitialSolution}\PY{p}{,}\PY{l+m+mf}{1000.0}\PY{p}{,}\PY{l+m+mf}{0.1}\PY{p}{)}
\end{Verbatim}

    \begin{Verbatim}[commandchars=\\\{\}]
{\color{incolor}In [{\color{incolor}19}]:} \PY{n+nb}{print}\PY{p}{(}\PY{n}{Knapsack\PYZus{}}\PY{p}{)}
\end{Verbatim}

    \begin{Verbatim}[commandchars=\\\{\}]
Simulated Annealing: 
 f(X) = -15402.0 
 X = [0 0 0 0 0 0 0 0 0 0 1 1 0 0 1 0 0 0 0 1 0 0 1 0 1 0 0 0 0 0 0 0 0 0 0 0 0
 0 0 1 1 0 0 0 1 0 0 1 0 1 0 1 0 0 0 0 0 0 0 0 0 1 0 0 0 0 0 0 0 1 0 0 0 0
 0 1 0 0 0 1 1 1 0 0 1 0 0 0 0 0 0 1 1 0 0 0 0 0 0 0 0 1 0 0 0 0 0 0 0 1 0
 0 0 0 0 1 0 0 0 0 0 0 0 1 0 0 0 0 1 0 0 0 0 0 0 0 0 0 0 0 0 0 0 1 0 0 0 1
 0 0 1 0 0 0 0 0 0 0 1 0 0 0 0 0 0 1 0 0 0 0 0 0 0 0 0 0 0 1 0 0 0 0 0 0 0
 0 0 0 0 0 0 0 0 0 0 0 1 0 0 0 0 0 0 0 0 1 1 0 0 0 0 0 0 1 0 1 0 0 1 0 0 0
 0 0 0 0 0 1 0 0 0 0 0 0 0 0 0 0 1 0 0 0 0 1 0 0 0 0 0 0 0 1 0 0 0 0 0 1 0
 0 0 1 0 1 0 0 0 0 0 0 0 1 0 0 1 0 0 0 0 0 1 0 0 0 0 0 0 0 1 0 1 0 0 1 0 1
 0 0 0 0 0 1 0 0 1 1 0 0 0 0 1 0 0 0 0 0 1 0 1 0 1 0 0 0 0 0 1 1 0 0 0 0 0
 0 0 0 1 0 0 1 0 1 0 0 0 0 1 0 0 0 0 0 0 0 0 0 0 0 1 0 0 1 0 1 0 1 0 1 0 0
 0 0 0 0 1 0 0 0 0 0 0 1 1 0 0 0 0 0 0 1 0 0 1 0 1 0 0 0 0 1 0 0 1 0 0 1 0
 0 0 0 0 0 0 1 0 0 1 0 0 0 0 1 0 1 0 0 1 0 0 0 0 1 0 0 1 0 0 0 0 0 0 0 1 1
 1 0 0 0 0 0 0 0 0 0 0 1 0 0 0 0 0 0 0 0 1 0 0 0 1 0 0 1 0 0 1 0 0 1 0 1 0
 0 0 1 0 0 0 1 0 0 0 0 1 0 1 1 0 0 1 0 1 0 0 0 0 0 0 0 0 0 0 0 0 1 1 1 1 1
 0 1 0 1 0 0 0 0 0 0 1 0 0 1 0 0 0 0 1 0 0 1 0 0 0 0 0 0 1 0 0 0 0 0 1 0 1
 0 0 0 0 1 1 1 0 1 1 0 0 1 0 0 0 0 0 1 0 0 0 0 0 0 0 1 0 0 0 0 1 0 1 1 1 0
 1 0 0 1 0 0 0 0 0 1 0 0 0 1 0 0 1 0 0 1 1 0 1 0 1 0 0 0 0 1 0 0 0 1 1 0 1
 0 0 0 0 0 0 0 0 0 0 1 0 0 0 0 0 0 1 0 1 1 0 0 0 0 0 0 0 1 0 0 0 0 0 0 0 1
 0 1 0 0 0 0 0 1 0 0 1 0 0 0 0 0 0 0 0 0 0 0 0 1 0 0 1 1 0 1 0 0 1 1 0 0 0
 0 0 0 0 0 0 1 0 0 0 0 0 0 1 0 0 0 1 0 1 0 0 0 0 0 1 0 0 0 0 0 0 0 0 0 0 0
 0 0 0 1 0 0 0 0 0 0 0 0 0 0 0 0 0 0 0 0 0 0 0 0 0 1 0 1 1 0 1 0 1 0 0 0 0
 0 0 0 1 0 0 1 0 0 0 0 1 0 0 0 1 0 1 0 0 0 0 0 0 0 0 0 0 0 0 0 0 0 1 0 0 0
 0 0 0 0 0 0 0 0 1 0 0 1 0 0 0 0 0 0 0 0 1 0 0 1 0 1 0 0 0 0 0 0 1 0 0 0 0
 0 0 0 1 0 0 0 1 0 0 0 0 0 0 0 0 0 1 0 0 1 0 0 1 0 0 0 0 0 0 0 0 1 0 0 0 0
 0 0 0 1 1 0 0 0 1 0 0 0 0 0 0 0 0 0 1 0 1 0 0 1 0 0 0 0 0 0 0 0 0 0 0 1 0
 1 0 0 0 0 0 0 0 0 0 0 0 0 1 0 0 0 1 0 0 0 0 1 1 1 0 0 1 0 0 0 0 0 1 1 0 0
 0 1 1 1 0 0 0 0 0 0 0 1 0 1 0 0 0 0 0 0 0 0 0 0 0 1 0 0 0 0 0 0 1 0 0 0 0
 0] 
 Constraints: 
 9785 <= 9786 


    \end{Verbatim}

    Para realizar un estudio estadístico del comportamiento de la
metaheurística, la librería Pyristic cuenta con una función llamada
get\_stats. Esta función se encuentra en utils.helpers y recibe como
parámetros:

\begin{verbatim}
Objeto que realiza la ejecución de la metaheurística.
El número de veces que se quiere ejecutar la metaheurística.
Tupla con los argumentos que recibe la función optimize.
Argumentos adicionales a la búsqueda (opcional).
\end{verbatim}

La función get\_stats considera las solución regresada por la
metaheurística en cada ejecución y retorna un diccionario con la mejor y
peor solución encontrada, la media y desviación estándar del valor de la
función objetivo.

    \begin{Verbatim}[commandchars=\\\{\}]
{\color{incolor}In [{\color{incolor}20}]:} \PY{n}{args} \PY{o}{=} \PY{p}{(}\PY{n}{getInitialSolution}\PY{p}{,}\PY{l+m+mf}{1000.0}\PY{p}{,}\PY{l+m+mf}{0.01}\PY{p}{)}
         \PY{n}{statistics} \PY{o}{=} \PY{n}{get\PYZus{}stats}\PY{p}{(}\PY{n}{Knapsack\PYZus{}}\PY{p}{,} \PY{l+m+mi}{30}\PY{p}{,} \PY{n}{args}\PY{p}{)}
\end{Verbatim}

    \begin{Verbatim}[commandchars=\\\{\}]
{\color{incolor}In [{\color{incolor}21}]:} \PY{n}{pprint}\PY{p}{(}\PY{n}{statistics}\PY{p}{)}
\end{Verbatim}

    \begin{Verbatim}[commandchars=\\\{\}]
\{'Best solution': \{'f': -16780.0,
                   'x': array([1, 0, 0, 0, 0, 0, 0, 1, 0, 0, 0, 0, 0, 0, 0, 0, 1, 0, 0, 1, 0, 0,
       0, 0, 0, 0, 0, 1, 0, 0, 1, 0, 0, 0, 0, 0, 0, 0, 0, 0, 1, 0, 0, 0,
       1, 0, 0, 1, 1, 0, 0, 0, 0, 1, 0, 0, 0, 0, 0, 0, 0, 1, 0, 0, 0, 1,
       1, 0, 0, 0, 0, 0, 0, 0, 0, 1, 0, 1, 0, 0, 0, 0, 0, 1, 0, 0, 0, 0,
       0, 0, 1, 1, 0, 0, 0, 0, 0, 0, 0, 0, 0, 0, 0, 0, 0, 1, 1, 0, 0, 1,
       0, 0, 0, 0, 0, 0, 0, 0, 0, 1, 0, 0, 1, 0, 1, 1, 1, 0, 0, 0, 0, 0,
       0, 1, 1, 1, 0, 1, 0, 0, 1, 0, 0, 0, 0, 0, 0, 1, 0, 0, 1, 0, 0, 1,
       0, 0, 1, 1, 0, 0, 0, 1, 0, 0, 0, 0, 0, 0, 0, 0, 0, 1, 0, 0, 0, 0,
       0, 1, 0, 0, 0, 0, 0, 0, 0, 0, 0, 1, 0, 1, 0, 1, 0, 0, 1, 0, 0, 0,
       1, 1, 1, 0, 0, 0, 0, 0, 0, 1, 0, 1, 0, 0, 0, 0, 0, 0, 0, 1, 1, 0,
       0, 1, 0, 0, 0, 0, 0, 0, 0, 1, 1, 0, 0, 0, 0, 0, 0, 0, 1, 0, 1, 0,
       1, 1, 0, 0, 0, 0, 0, 0, 0, 0, 0, 1, 1, 0, 0, 0, 0, 0, 0, 1, 0, 0,
       0, 0, 0, 0, 1, 1, 0, 0, 0, 0, 1, 0, 0, 1, 1, 1, 0, 0, 1, 0, 1, 0,
       0, 0, 0, 0, 0, 1, 0, 1, 0, 0, 0, 0, 0, 0, 0, 0, 0, 0, 0, 0, 0, 1,
       0, 1, 0, 0, 0, 0, 0, 0, 0, 0, 0, 0, 0, 0, 1, 0, 0, 1, 0, 0, 1, 0,
       0, 1, 1, 0, 0, 0, 0, 0, 1, 0, 0, 0, 0, 0, 0, 1, 1, 0, 0, 0, 0, 1,
       0, 0, 1, 0, 0, 0, 0, 0, 0, 0, 0, 1, 0, 1, 0, 0, 0, 0, 0, 0, 1, 1,
       0, 0, 0, 0, 0, 0, 0, 0, 0, 0, 1, 1, 0, 1, 0, 0, 0, 0, 0, 0, 1, 0,
       0, 1, 0, 0, 0, 0, 0, 0, 0, 0, 0, 0, 0, 0, 1, 0, 0, 0, 0, 0, 0, 0,
       0, 0, 0, 0, 1, 1, 0, 1, 1, 0, 0, 0, 0, 0, 1, 0, 0, 0, 1, 0, 1, 0,
       1, 0, 0, 0, 0, 0, 0, 0, 0, 0, 0, 0, 1, 0, 0, 0, 1, 0, 0, 1, 0, 0,
       1, 0, 0, 0, 0, 0, 1, 1, 0, 0, 0, 0, 0, 0, 0, 1, 0, 1, 0, 0, 1, 1,
       0, 0, 0, 1, 0, 0, 1, 0, 0, 0, 0, 0, 0, 0, 1, 1, 0, 0, 0, 0, 0, 0,
       0, 1, 0, 1, 0, 1, 0, 0, 1, 0, 1, 0, 0, 0, 0, 0, 1, 1, 0, 0, 1, 0,
       1, 0, 0, 0, 1, 0, 0, 0, 1, 0, 1, 0, 0, 0, 0, 0, 0, 0, 0, 1, 0, 0,
       0, 0, 0, 0, 0, 0, 0, 0, 0, 0, 1, 0, 0, 0, 0, 0, 0, 0, 0, 0, 1, 0,
       1, 0, 0, 0, 1, 1, 0, 1, 0, 0, 0, 0, 1, 0, 1, 1, 0, 1, 0, 0, 0, 1,
       0, 0, 0, 0, 0, 0, 1, 0, 0, 0, 0, 0, 0, 0, 0, 0, 0, 0, 1, 0, 0, 0,
       0, 0, 0, 0, 0, 1, 0, 0, 0, 0, 1, 0, 0, 0, 0, 0, 0, 1, 0, 0, 1, 0,
       1, 0, 1, 0, 0, 0, 1, 0, 0, 1, 0, 1, 0, 1, 0, 0, 0, 1, 1, 1, 0, 1,
       0, 0, 1, 0, 0, 0, 0, 1, 0, 0, 0, 0, 0, 1, 1, 0, 0, 0, 0, 0, 0, 0,
       0, 1, 0, 0, 1, 0, 0, 0, 0, 0, 0, 0, 0, 0, 0, 0, 0, 1, 0, 1, 0, 1,
       0, 0, 1, 0, 0, 0, 1, 1, 1, 0, 0, 0, 0, 1, 0, 0, 0, 0, 1, 0, 0, 0,
       1, 0, 0, 0, 0, 0, 0, 0, 0, 0, 1, 0, 0, 1, 0, 1, 0, 0, 0, 1, 1, 0,
       0, 0, 0, 0, 0, 0, 0, 0, 0, 1, 0, 1, 0, 0, 1, 0, 0, 1, 0, 1, 0, 1,
       1, 0, 1, 1, 1, 0, 0, 0, 0, 0, 0, 0, 0, 0, 1, 0, 0, 1, 0, 0, 0, 0,
       0, 0, 0, 0, 1, 0, 0, 0, 0, 0, 0, 0, 1, 0, 1, 0, 0, 0, 0, 0, 0, 0,
       0, 1, 0, 0, 0, 0, 0, 0, 0, 0, 0, 0, 0, 0, 0, 0, 0, 0, 0, 0, 0, 0,
       0, 0, 0, 0, 0, 1, 1, 0, 0, 1, 0, 0, 0, 0, 0, 0, 0, 0, 0, 0, 1, 1,
       0, 0, 0, 0, 1, 1, 0, 0, 0, 0, 1, 0, 1, 0, 0, 0, 0, 0, 0, 0, 0, 1,
       1, 0, 0, 0, 0, 0, 1, 0, 0, 0, 0, 0, 0, 0, 0, 0, 0, 0, 0, 1, 0, 0,
       1, 0, 1, 0, 0, 0, 0, 0, 0, 0, 0, 0, 0, 0, 0, 0, 0, 0, 0, 1, 1, 0,
       0, 1, 0, 0, 0, 0, 0, 1, 0, 0, 0, 0, 0, 1, 0, 0, 0, 1, 0, 1, 0, 0,
       0, 0, 0, 0, 0, 0, 0, 0, 1, 0, 0, 1, 1, 0, 0, 1, 0, 0, 0, 0, 0, 0,
       0, 0, 0, 0, 1, 0, 0, 0, 0, 1, 0, 0, 0, 0, 0, 0, 1, 0, 0, 0, 0, 0,
       0, 0, 1, 0, 0, 1, 0, 0, 0, 0])\},
 'Mean': -15396.4,
 'Standard deviation': 488.9018715447917,
 'Worst solution': \{'f': -14012.0,
                    'x': array([0, 0, 0, 0, 0, 0, 0, 1, 0, 0, 0, 0, 0, 0, 0, 1, 0, 0, 1, 0, 0, 0,
       1, 1, 0, 0, 1, 0, 0, 0, 0, 0, 0, 0, 1, 0, 1, 1, 0, 0, 0, 0, 0, 0,
       0, 0, 0, 0, 0, 0, 0, 0, 0, 0, 0, 0, 1, 0, 1, 0, 0, 0, 1, 0, 1, 0,
       0, 0, 1, 0, 0, 0, 0, 0, 0, 0, 0, 0, 1, 0, 0, 1, 0, 0, 0, 0, 0, 1,
       1, 1, 0, 1, 0, 1, 0, 0, 0, 0, 0, 0, 1, 0, 0, 1, 0, 0, 0, 0, 0, 0,
       0, 0, 0, 0, 1, 0, 0, 0, 0, 1, 0, 0, 0, 0, 1, 0, 1, 0, 0, 1, 0, 0,
       0, 1, 0, 0, 0, 0, 0, 1, 0, 1, 0, 0, 0, 0, 0, 0, 0, 0, 0, 0, 0, 0,
       0, 0, 1, 0, 0, 0, 0, 0, 0, 0, 0, 0, 0, 0, 0, 0, 0, 0, 0, 0, 0, 0,
       0, 1, 1, 0, 0, 0, 0, 0, 0, 0, 0, 0, 0, 0, 0, 1, 0, 1, 0, 0, 0, 0,
       0, 0, 0, 0, 1, 0, 0, 1, 0, 1, 0, 1, 0, 0, 0, 0, 0, 0, 1, 0, 0, 0,
       0, 0, 1, 0, 0, 0, 0, 0, 0, 0, 1, 0, 0, 0, 0, 0, 1, 0, 0, 0, 0, 0,
       0, 0, 0, 0, 0, 0, 0, 0, 0, 0, 1, 1, 0, 0, 0, 0, 0, 0, 0, 0, 1, 0,
       0, 0, 0, 0, 0, 0, 0, 0, 0, 0, 0, 0, 1, 0, 0, 0, 0, 0, 0, 1, 0, 0,
       0, 0, 0, 0, 0, 0, 0, 0, 0, 0, 0, 0, 1, 0, 1, 0, 1, 0, 1, 0, 0, 0,
       0, 1, 1, 0, 0, 0, 0, 0, 0, 0, 0, 0, 0, 0, 0, 0, 1, 1, 0, 1, 0, 0,
       0, 0, 0, 0, 0, 0, 0, 0, 0, 0, 0, 0, 1, 0, 0, 0, 0, 0, 0, 0, 0, 0,
       1, 0, 1, 0, 0, 0, 1, 0, 0, 1, 0, 0, 1, 1, 0, 0, 0, 0, 0, 0, 0, 0,
       1, 1, 0, 0, 0, 0, 1, 0, 0, 0, 0, 1, 0, 1, 0, 0, 0, 0, 1, 0, 0, 0,
       0, 0, 0, 0, 0, 0, 0, 0, 0, 0, 0, 0, 0, 0, 0, 0, 0, 0, 1, 1, 0, 0,
       0, 1, 1, 0, 1, 0, 0, 0, 0, 1, 0, 0, 0, 0, 0, 0, 0, 1, 0, 1, 0, 0,
       1, 0, 1, 1, 0, 0, 1, 0, 1, 1, 0, 0, 0, 0, 0, 0, 0, 0, 0, 0, 0, 0,
       1, 1, 1, 0, 0, 1, 1, 0, 0, 0, 0, 0, 0, 0, 0, 0, 0, 1, 0, 0, 0, 1,
       0, 0, 0, 1, 1, 0, 0, 0, 0, 0, 0, 0, 0, 0, 0, 1, 0, 0, 0, 0, 0, 0,
       0, 1, 0, 1, 0, 0, 0, 1, 1, 0, 1, 1, 0, 0, 0, 0, 0, 0, 0, 1, 0, 0,
       1, 0, 0, 0, 0, 0, 0, 0, 0, 0, 0, 0, 0, 0, 0, 1, 0, 0, 0, 1, 0, 0,
       1, 0, 0, 0, 0, 0, 1, 0, 0, 0, 0, 0, 0, 0, 0, 0, 0, 0, 1, 0, 0, 1,
       0, 0, 0, 0, 0, 1, 0, 0, 0, 0, 0, 0, 1, 0, 0, 0, 0, 0, 0, 1, 1, 0,
       0, 0, 0, 0, 0, 0, 1, 1, 0, 0, 1, 1, 0, 0, 0, 0, 0, 0, 0, 0, 0, 0,
       0, 0, 0, 0, 0, 0, 0, 0, 0, 0, 0, 0, 0, 1, 0, 0, 0, 0, 1, 0, 1, 0,
       1, 0, 0, 0, 0, 0, 0, 0, 0, 0, 0, 1, 0, 1, 0, 0, 0, 0, 0, 0, 0, 1,
       0, 0, 0, 0, 0, 1, 0, 0, 0, 1, 1, 0, 1, 0, 1, 0, 0, 0, 0, 1, 1, 0,
       0, 0, 1, 0, 0, 0, 0, 0, 0, 0, 1, 0, 0, 0, 0, 0, 0, 0, 0, 1, 0, 1,
       0, 0, 0, 1, 0, 1, 0, 1, 0, 0, 0, 0, 0, 0, 0, 0, 0, 0, 0, 0, 0, 0,
       0, 0, 0, 0, 0, 0, 0, 0, 0, 1, 0, 0, 1, 0, 0, 0, 0, 0, 1, 0, 0, 0,
       0, 0, 0, 0, 0, 0, 1, 0, 0, 0, 0, 0, 0, 1, 0, 1, 0, 0, 0, 0, 0, 0,
       1, 0, 0, 0, 0, 0, 0, 0, 0, 0, 0, 1, 0, 0, 0, 1, 0, 0, 0, 0, 0, 0,
       0, 0, 0, 0, 0, 1, 0, 0, 1, 0, 1, 0, 0, 1, 1, 0, 0, 0, 1, 0, 0, 0,
       0, 0, 0, 0, 0, 0, 1, 0, 0, 0, 0, 0, 1, 0, 0, 0, 0, 0, 0, 0, 0, 0,
       0, 0, 1, 1, 0, 0, 0, 0, 0, 1, 0, 0, 0, 0, 0, 1, 0, 0, 0, 0, 0, 0,
       0, 1, 0, 0, 0, 0, 1, 0, 0, 0, 0, 0, 0, 0, 1, 0, 0, 0, 0, 0, 0, 0,
       0, 0, 0, 1, 0, 0, 0, 0, 1, 0, 1, 0, 0, 0, 0, 0, 0, 1, 0, 0, 0, 0,
       0, 1, 0, 0, 0, 1, 0, 0, 0, 0, 0, 0, 1, 0, 0, 0, 0, 0, 0, 0, 0, 0,
       1, 0, 1, 1, 1, 1, 0, 0, 0, 0, 1, 0, 0, 0, 0, 0, 0, 0, 0, 0, 0, 1,
       0, 0, 0, 1, 0, 0, 0, 1, 0, 0, 1, 0, 0, 1, 0, 0, 0, 1, 1, 0, 0, 1,
       1, 0, 0, 0, 0, 1, 0, 1, 0, 1, 1, 0, 0, 0, 0, 0, 0, 0, 0, 0, 0, 1,
       0, 0, 0, 1, 1, 0, 0, 0, 0, 0])\}\}

    \end{Verbatim}

    Se puede observar que los resultados obtenidos no son tan buenos como
los obtenidos con la metaheurística basada en Búsqueda Tabú. A
continuación se harán algunas modificaciones en el diseño, con el
objetivo de mejorar los resultados.

    \subsubsection{Solución inicial}\label{soluciuxf3n-inicial}

En el diseño anterior, no se consideró el beneficio de introducir un
objeto determinado a la mochila según su valor y peso. Una forma de
hacerlo es la siguiente:

\begin{enumerate}
\def\labelenumi{\arabic{enumi}.}
\item
  Asignar a cada objeto su valor por unidad de peso:
  \(\frac{p_{i}}{w_{i}}\).
\item
  Ordenar todos los objetos con respecto a este indicador.
\item
  Seleccionar los objetos que tenga mayor valor en este indicador sin
  exceder la capacidad.
\end{enumerate}

    \begin{Verbatim}[commandchars=\\\{\}]
{\color{incolor}In [{\color{incolor}22}]:} \PY{k}{def} \PY{n+nf}{Init\PYZus{}GISP}\PY{p}{(}\PY{p}{)}\PY{p}{:}
             \PY{k}{global} \PY{n}{n}\PY{p}{,}\PY{n}{p}\PY{p}{,}\PY{n}{w}\PY{p}{,}\PY{n}{c}
             \PY{n}{Arr\PYZus{}}\PY{o}{=}\PY{p}{[}\PY{p}{]}
             \PY{k}{for} \PY{n}{i} \PY{o+ow}{in} \PY{n+nb}{range}\PY{p}{(}\PY{n}{n}\PY{p}{)}\PY{p}{:}
                 \PY{n}{Arr\PYZus{}}\PY{o}{.}\PY{n}{append}\PY{p}{(}\PY{p}{(}\PY{n}{p}\PY{p}{[}\PY{n}{i}\PY{p}{]}\PY{o}{/}\PY{n}{w}\PY{p}{[}\PY{n}{i}\PY{p}{]}\PY{p}{,}\PY{n}{i}\PY{p}{)}\PY{p}{)}
             
             \PY{n}{Arr\PYZus{}}\PY{o}{.}\PY{n}{sort}\PY{p}{(}\PY{n}{reverse}\PY{o}{=}\PY{k+kc}{True}\PY{p}{)}
             \PY{k}{return} \PY{n}{Arr\PYZus{}}
         \PY{n}{GISP\PYZus{}Arr} \PY{o}{=} \PY{n}{Init\PYZus{}GISP}\PY{p}{(}\PY{p}{)}
         
         \PY{k}{def} \PY{n+nf}{getInitialSolution2}\PY{p}{(}\PY{p}{)}\PY{p}{:}
             \PY{k}{global} \PY{n}{n}\PY{p}{,}\PY{n}{p}\PY{p}{,}\PY{n}{w}\PY{p}{,}\PY{n}{c}\PY{p}{,} \PY{n}{GISP\PYZus{}Arr}
             \PY{n}{X} \PY{o}{=} \PY{p}{[}\PY{l+m+mi}{0} \PY{k}{for} \PY{n}{i} \PY{o+ow}{in} \PY{n+nb}{range}\PY{p}{(}\PY{n}{n}\PY{p}{)}\PY{p}{]}
             \PY{n}{current\PYZus{}weight} \PY{o}{=} \PY{l+m+mi}{0}
             \PY{k}{for} \PY{n}{i} \PY{o+ow}{in} \PY{n+nb}{range}\PY{p}{(}\PY{n}{n}\PY{p}{)}\PY{p}{:}
                 \PY{n}{ind} \PY{o}{=} \PY{n}{GISP\PYZus{}Arr}\PY{p}{[}\PY{n}{i}\PY{p}{]}\PY{p}{[}\PY{l+m+mi}{1}\PY{p}{]}
                 \PY{k}{if} \PY{n}{current\PYZus{}weight}\PY{o}{+} \PY{n}{w}\PY{p}{[}\PY{n}{ind}\PY{p}{]} \PY{o}{\PYZlt{}}\PY{o}{=} \PY{n}{c}\PY{p}{:}
                     \PY{n}{current\PYZus{}weight}\PY{o}{+}\PY{o}{=}\PY{n}{w}\PY{p}{[}\PY{n}{ind}\PY{p}{]}
                     \PY{n}{X}\PY{p}{[}\PY{n}{ind}\PY{p}{]} \PY{o}{=} \PY{l+m+mi}{1}
             \PY{k}{return} \PY{n}{np}\PY{o}{.}\PY{n}{array}\PY{p}{(}\PY{n}{X}\PY{p}{)}
\end{Verbatim}

    Para crear un nuevo vecino, seguiremos la misma estrategia que en el
diseño anterior. Sin embargo, si la nueva solución no es válida vamos a
realizar lo siguiente:

\begin{enumerate}
\def\labelenumi{\arabic{enumi}.}
\tightlist
\item
  Fase de reparación.
\item
  Fase de mejoramiento.
\end{enumerate}

\textbf{Fase de reparación:} En esta fase retiramos objetos que se
encuentran en la mochila mientras la capacidad actual exceda el límite
definido \(c\). Los objetos son retirados de acuerdo a los valores más
pequeños en el indicador definido (\(\frac{p_{i}}{w_{i}}\)).

\textbf{Fase de mejoramiento:} En esta fase se introducen objetos a la
mochila siempre y cuando no excedan la capacidad. Estos objetos son
introducidos priorizando aquellos con los valores más grandes en el
indicador definido (\(\frac{p_{i}}{w_{i}}\)).

    \begin{Verbatim}[commandchars=\\\{\}]
{\color{incolor}In [{\color{incolor}23}]:} \PY{k}{class} \PY{n+nc}{Knapsack\PYZus{}solver2}\PY{p}{(}\PY{n}{SimulatedAnnealing}\PY{p}{)}\PY{p}{:}
         
             \PY{n+nd}{@checkargs}
             \PY{k}{def} \PY{n+nf}{\PYZus{}\PYZus{}init\PYZus{}\PYZus{}}\PY{p}{(}\PY{n+nb+bp}{self}\PY{p}{,} \PY{n}{f\PYZus{}} \PY{p}{:} \PY{n}{function\PYZus{}type} \PY{p}{,} \PY{n}{constraints\PYZus{}}\PY{p}{:} \PY{n+nb}{list}\PY{p}{)}\PY{p}{:}
                 \PY{n+nb}{super}\PY{p}{(}\PY{p}{)}\PY{o}{.}\PY{n+nf+fm}{\PYZus{}\PYZus{}init\PYZus{}\PYZus{}}\PY{p}{(}\PY{n}{f\PYZus{}}\PY{p}{,}\PY{n}{constraints\PYZus{}}\PY{p}{)}
                 
                 
             \PY{k}{def} \PY{n+nf}{generate\PYZus{}neighbor}\PY{p}{(}\PY{n+nb+bp}{self}\PY{p}{,} \PY{n}{x}\PY{p}{:} \PY{n}{np}\PY{o}{.}\PY{n}{ndarray}\PY{p}{)} \PY{o}{\PYZhy{}}\PY{o}{\PYZgt{}} \PY{n}{np}\PY{o}{.}\PY{n}{ndarray}\PY{p}{:}
                 \PY{n}{x\PYZus{}} \PY{o}{=} \PY{n}{x}\PY{o}{.}\PY{n}{copy}\PY{p}{(}\PY{p}{)}
                 \PY{n}{N} \PY{o}{=} \PY{n+nb}{len}\PY{p}{(}\PY{n}{x\PYZus{}}\PY{p}{)}
                 \PY{n}{ind} \PY{o}{=} \PY{n}{random}\PY{o}{.}\PY{n}{randint}\PY{p}{(}\PY{l+m+mi}{0}\PY{p}{,} \PY{n}{N}\PY{o}{\PYZhy{}}\PY{l+m+mi}{1}\PY{p}{)}
                 \PY{n}{x\PYZus{}}\PY{p}{[}\PY{n}{ind}\PY{p}{]} \PY{o}{\PYZca{}}\PY{o}{=} \PY{l+m+mi}{1}
                 
                 \PY{k}{return} \PY{n}{x\PYZus{}}
             
             \PY{k}{def} \PY{n+nf}{repair\PYZus{}neighbor}\PY{p}{(}\PY{n+nb+bp}{self}\PY{p}{,} \PY{n}{x}\PY{p}{:} \PY{n}{np}\PY{o}{.}\PY{n}{ndarray}\PY{p}{)} \PY{o}{\PYZhy{}}\PY{o}{\PYZgt{}} \PY{n}{np}\PY{o}{.}\PY{n}{ndarray}\PY{p}{:}
                 \PY{k}{global} \PY{n}{GISP\PYZus{}Arr}\PY{p}{,}\PY{n}{w}\PY{p}{,}\PY{n}{c}\PY{p}{,}\PY{n}{n}
                 
                 \PY{c+c1}{\PYZsh{}Get the total weight}
                 \PY{n}{total\PYZus{}weight}\PY{o}{=}\PY{l+m+mi}{0}
                 \PY{k}{for} \PY{n}{i} \PY{o+ow}{in} \PY{n+nb}{range}\PY{p}{(}\PY{n}{n}\PY{p}{)}\PY{p}{:}
                     \PY{k}{if} \PY{n}{x}\PY{p}{[}\PY{n}{i}\PY{p}{]}\PY{p}{:}
                         \PY{n}{total\PYZus{}weight} \PY{o}{+}\PY{o}{=} \PY{n}{w}\PY{p}{[}\PY{n}{i}\PY{p}{]}
                 
                 \PY{k}{for} \PY{n}{i} \PY{o+ow}{in} \PY{n+nb}{range}\PY{p}{(}\PY{n}{n}\PY{p}{)}\PY{p}{:}
                     \PY{n}{ind} \PY{o}{=} \PY{n}{GISP\PYZus{}Arr}\PY{p}{[}\PY{n}{n} \PY{o}{\PYZhy{}} \PY{p}{(}\PY{l+m+mi}{1}\PY{o}{+}\PY{n}{i}\PY{p}{)}\PY{p}{]}\PY{p}{[}\PY{l+m+mi}{1}\PY{p}{]} \PY{c+c1}{\PYZsh{}Lowest}
                     
                     \PY{k}{if} \PY{n}{x}\PY{p}{[}\PY{n}{ind}\PY{p}{]}\PY{p}{:}
                         \PY{n}{total\PYZus{}weight} \PY{o}{\PYZhy{}}\PY{o}{=} \PY{n}{w}\PY{p}{[}\PY{n}{ind}\PY{p}{]}
                         \PY{n}{x}\PY{p}{[}\PY{n}{ind}\PY{p}{]} \PY{o}{=} \PY{l+m+mi}{0}
                     
                     \PY{k}{if} \PY{n}{total\PYZus{}weight} \PY{o}{\PYZlt{}}\PY{o}{=} \PY{n}{c}\PY{p}{:}
                         \PY{k}{break}
                         
             \PY{k}{def} \PY{n+nf}{improve\PYZus{}neighbor}\PY{p}{(}\PY{n+nb+bp}{self}\PY{p}{,} \PY{n}{x}\PY{p}{:} \PY{n}{np}\PY{o}{.}\PY{n}{ndarray}\PY{p}{)} \PY{o}{\PYZhy{}}\PY{o}{\PYZgt{}} \PY{n}{np}\PY{o}{.}\PY{n}{ndarray}\PY{p}{:}
                 \PY{k}{global} \PY{n}{GISP\PYZus{}Arr}\PY{p}{,}\PY{n}{w}\PY{p}{,}\PY{n}{c}\PY{p}{,}\PY{n}{n}
                 
                 \PY{c+c1}{\PYZsh{}Get the total weight}
                 \PY{n}{total\PYZus{}weight}\PY{o}{=}\PY{l+m+mi}{0}
                 \PY{k}{for} \PY{n}{i} \PY{o+ow}{in} \PY{n+nb}{range}\PY{p}{(}\PY{n}{n}\PY{p}{)}\PY{p}{:}
                     \PY{k}{if} \PY{n}{x}\PY{p}{[}\PY{n}{i}\PY{p}{]}\PY{p}{:}
                         \PY{n}{total\PYZus{}weight} \PY{o}{+}\PY{o}{=} \PY{n}{w}\PY{p}{[}\PY{n}{i}\PY{p}{]}
                         
                 \PY{k}{for} \PY{n}{i} \PY{o+ow}{in} \PY{n+nb}{range}\PY{p}{(}\PY{n}{n}\PY{p}{)}\PY{p}{:}
                     \PY{n}{ind} \PY{o}{=} \PY{n}{GISP\PYZus{}Arr}\PY{p}{[}\PY{n}{i}\PY{p}{]}\PY{p}{[}\PY{l+m+mi}{1}\PY{p}{]}
                     \PY{k}{if} \PY{n}{total\PYZus{}weight} \PY{o}{+} \PY{n}{w}\PY{p}{[}\PY{n}{ind}\PY{p}{]} \PY{o}{\PYZgt{}} \PY{n}{c}\PY{p}{:}
                         \PY{k}{continue}
                         
                     \PY{k}{if} \PY{n}{x}\PY{p}{[}\PY{n}{ind}\PY{p}{]} \PY{o}{==} \PY{l+m+mi}{0} \PY{p}{:}
                         \PY{n}{total\PYZus{}weight} \PY{o}{+}\PY{o}{=} \PY{n}{w}\PY{p}{[}\PY{n}{ind}\PY{p}{]}
                         \PY{n}{x}\PY{p}{[}\PY{n}{ind}\PY{p}{]} \PY{o}{=} \PY{l+m+mi}{1}
                         
                     
             \PY{n+nd}{@checkargs}
             \PY{k}{def} \PY{n+nf}{get\PYZus{}neighbor}\PY{p}{(}\PY{n+nb+bp}{self}\PY{p}{,} \PY{n}{x} \PY{p}{:} \PY{n}{np}\PY{o}{.}\PY{n}{ndarray}\PY{p}{)} \PY{o}{\PYZhy{}}\PY{o}{\PYZgt{}} \PY{n}{np}\PY{o}{.}\PY{n}{ndarray}\PY{p}{:} 
                 
                 \PY{n}{neighbor\PYZus{}} \PY{o}{=} \PY{n+nb+bp}{self}\PY{o}{.}\PY{n}{generate\PYZus{}neighbor}\PY{p}{(}\PY{n}{x}\PY{p}{)}
                     
                 \PY{k}{if}\PY{p}{(}\PY{n+nb+bp}{self}\PY{o}{.}\PY{n}{is\PYZus{}valid}\PY{p}{(}\PY{n}{neighbor\PYZus{}}\PY{p}{)}\PY{p}{)}\PY{p}{:}
                     \PY{k}{return} \PY{n}{neighbor\PYZus{}}
                 
                 \PY{c+c1}{\PYZsh{}RI strategy}
                 \PY{n+nb+bp}{self}\PY{o}{.}\PY{n}{repair\PYZus{}neighbor}\PY{p}{(}\PY{n}{neighbor\PYZus{}}\PY{p}{)}
                 \PY{n+nb+bp}{self}\PY{o}{.}\PY{n}{improve\PYZus{}neighbor}\PY{p}{(}\PY{n}{neighbor\PYZus{}}\PY{p}{)}
                 
                 \PY{k}{return} \PY{n}{neighbor\PYZus{}}
\end{Verbatim}

    Ejecutamos nuestra metaheurística:

    \begin{Verbatim}[commandchars=\\\{\}]
{\color{incolor}In [{\color{incolor}24}]:} \PY{n}{f}\PY{p}{(}\PY{n}{getInitialSolution2}\PY{p}{(}\PY{p}{)}\PY{p}{)}
\end{Verbatim}

\begin{Verbatim}[commandchars=\\\{\}]
{\color{outcolor}Out[{\color{outcolor}24}]:} -29887.0
\end{Verbatim}
            
    \begin{Verbatim}[commandchars=\\\{\}]
{\color{incolor}In [{\color{incolor}25}]:} \PY{n}{Knapsack\PYZus{}2} \PY{o}{=} \PY{n}{Knapsack\PYZus{}solver2}\PY{p}{(}\PY{n}{f}\PY{p}{,} \PY{p}{[}\PY{n}{g1}\PY{p}{]}\PY{p}{)}
\end{Verbatim}

    \begin{Verbatim}[commandchars=\\\{\}]
{\color{incolor}In [{\color{incolor}26}]:} \PY{n}{Knapsack\PYZus{}2}\PY{o}{.}\PY{n}{optimize}\PY{p}{(}\PY{n}{getInitialSolution2}\PY{p}{,}\PY{l+m+mf}{1000.0}\PY{p}{,}\PY{l+m+mf}{0.1}\PY{p}{)}
         \PY{n+nb}{print}\PY{p}{(}\PY{n}{Knapsack\PYZus{}2}\PY{p}{)}
\end{Verbatim}

    \begin{Verbatim}[commandchars=\\\{\}]
Simulated Annealing: 
 f(X) = -29901.0 
 X = [1 0 0 0 1 0 0 1 0 0 0 0 1 0 0 0 1 1 0 1 0 0 1 0 1 0 1 1 0 1 1 0 0 1 0 1 0
 1 1 0 1 0 1 0 0 1 0 1 1 1 0 0 0 0 0 0 0 0 0 0 0 0 0 1 1 1 1 1 0 0 1 0 1 1
 0 1 0 0 0 0 0 1 1 1 0 0 0 1 0 0 0 0 0 0 1 0 1 1 0 0 0 0 0 0 0 1 1 0 0 1 1
 0 0 0 0 0 0 0 0 1 0 1 1 0 0 1 0 0 0 0 0 1 0 1 0 1 0 0 0 0 1 0 0 0 0 0 0 1
 1 0 1 1 1 0 1 1 1 0 1 1 0 0 0 0 0 0 1 1 0 1 0 1 0 0 1 0 0 0 1 0 1 1 0 1 1
 1 1 1 0 1 0 0 0 1 1 0 1 0 1 1 0 0 1 1 1 0 0 0 0 1 0 1 0 1 1 0 1 0 0 0 1 1
 0 0 0 1 0 0 0 0 1 0 1 0 0 1 0 0 1 0 0 0 0 1 0 0 1 0 0 0 0 0 0 1 0 0 0 1 1
 1 0 1 1 0 1 1 0 0 0 1 0 0 0 0 0 1 0 0 0 0 0 0 0 1 0 0 1 0 0 0 0 1 1 1 1 0
 0 0 0 0 0 1 1 0 1 0 1 0 0 1 1 1 0 0 0 0 0 0 1 0 1 0 0 1 0 1 0 0 0 0 0 0 0
 0 0 1 1 0 0 1 1 0 1 0 1 1 1 1 0 0 0 0 0 0 1 0 0 1 1 0 0 0 0 0 0 1 0 0 0 0
 0 0 0 1 1 1 0 0 0 1 1 1 0 1 1 0 0 0 0 0 1 0 1 0 1 0 1 0 0 0 0 1 1 1 0 1 0
 1 0 1 1 0 0 0 1 1 0 1 1 0 0 0 0 1 0 1 1 1 0 0 0 0 0 0 0 0 0 0 0 0 1 1 1 1
 1 1 1 0 0 1 0 0 0 0 0 1 0 0 0 1 1 0 1 0 0 0 0 0 1 0 0 1 1 1 0 0 0 1 0 1 0
 1 1 1 0 1 1 1 0 0 1 0 1 0 1 0 0 1 0 1 0 1 0 1 0 0 0 1 0 0 0 1 0 0 0 1 1 1
 0 1 0 0 1 1 1 0 1 0 1 0 0 0 1 0 0 1 1 1 0 1 1 0 1 0 0 0 1 1 0 1 0 0 1 0 1
 0 0 0 0 1 1 0 0 0 1 0 0 0 0 0 1 1 0 0 0 1 0 1 0 0 0 0 0 0 0 0 1 1 1 1 0 0
 1 0 1 1 0 1 0 0 1 1 0 1 0 0 0 1 0 0 0 0 0 0 0 0 0 0 0 0 0 0 1 0 0 0 0 1 1
 1 0 0 1 1 0 0 1 1 1 1 0 0 0 0 0 1 0 1 0 1 0 1 0 0 0 1 1 0 0 1 0 0 1 0 0 1
 1 1 0 0 0 0 0 1 0 0 0 0 1 1 1 0 0 0 1 0 0 1 1 1 0 1 0 0 1 1 1 0 0 0 1 0 0
 1 0 0 1 1 0 0 1 1 1 0 0 1 0 1 0 1 1 0 0 0 0 0 1 0 1 1 0 0 0 0 0 0 0 0 0 1
 1 1 1 0 0 0 0 0 0 0 0 1 0 1 0 0 0 1 0 0 0 0 0 1 1 1 0 1 1 1 1 1 0 0 1 0 0
 1 0 0 0 0 1 0 0 0 1 1 1 0 0 0 1 0 0 0 0 0 0 0 0 0 0 0 1 1 0 0 0 1 0 1 0 0
 1 1 1 0 0 1 0 0 1 0 1 1 0 1 1 1 1 0 0 1 1 0 0 1 0 1 1 1 1 0 0 0 1 0 0 1 0
 1 0 0 1 1 1 1 1 0 0 0 1 1 0 0 0 1 1 1 0 0 0 0 1 0 1 0 0 0 1 0 1 1 1 0 0 0
 1 0 0 1 1 0 0 0 0 0 0 0 0 1 1 1 0 0 0 0 0 0 1 1 0 1 1 0 1 0 1 0 0 1 0 1 1
 0 0 0 0 0 1 0 0 1 1 1 1 0 1 0 1 0 1 0 0 1 1 0 1 1 1 0 1 1 1 0 1 1 0 1 0 0
 1 1 0 0 0 0 0 1 0 0 1 1 1 0 0 1 1 1 1 0 0 0 1 0 0 0 0 1 1 0 0 0 0 1 0 0 0
 0] 
 Constraints: 
 9786 <= 9786 


    \end{Verbatim}

    \subsection{Resultados}\label{resultados}

Vamos a comparar las distintas combinaciones entre la forma de generar
la solución inicial y la estrategia de generar un nuevo vecino.

\begin{enumerate}
\def\labelenumi{\arabic{enumi}.}
\tightlist
\item
  Solución por Indicador y RI estrategia.
\item
  Solución por Indicador y estrategia ingenua.
\item
  Solución ingenua y RI estrategia.
\item
  Solución ingenua y estrategia ingenua.
\end{enumerate}

    \subsubsection{Solución por Indicador y RI
estrategia.}\label{soluciuxf3n-por-indicador-y-ri-estrategia.}

    \begin{Verbatim}[commandchars=\\\{\}]
{\color{incolor}In [{\color{incolor}27}]:} \PY{n}{args} \PY{o}{=} \PY{p}{(}\PY{n}{getInitialSolution2}\PY{p}{,}\PY{l+m+mf}{1000.0}\PY{p}{,}\PY{l+m+mf}{0.01}\PY{p}{)}
         \PY{n}{statistics} \PY{o}{=} \PY{n}{get\PYZus{}stats}\PY{p}{(}\PY{n}{Knapsack\PYZus{}2}\PY{p}{,} \PY{l+m+mi}{30}\PY{p}{,} \PY{n}{args}\PY{p}{)}
\end{Verbatim}

    \begin{Verbatim}[commandchars=\\\{\}]
{\color{incolor}In [{\color{incolor}28}]:} \PY{n+nb}{print}\PY{p}{(}\PY{l+s+s2}{\PYZdq{}}\PY{l+s+s2}{f(x*) = }\PY{l+s+si}{\PYZob{}\PYZcb{}}\PY{l+s+s2}{ }\PY{l+s+se}{\PYZbs{}n}\PY{l+s+s2}{función objetivo promedio: }\PY{l+s+si}{\PYZob{}\PYZcb{}}\PY{l+s+s2}{ }\PY{l+s+se}{\PYZbs{}n}\PY{l+s+s2}{función objetivo de la peor solución: }\PY{l+s+si}{\PYZob{}\PYZcb{}}\PY{l+s+s2}{\PYZdq{}}\PY{o}{.}\PY{n}{format}\PY{p}{(}\PY{n}{statistics}\PY{p}{[}\PY{l+s+s1}{\PYZsq{}}\PY{l+s+s1}{Best solution}\PY{l+s+s1}{\PYZsq{}}\PY{p}{]}\PY{p}{[}\PY{l+s+s1}{\PYZsq{}}\PY{l+s+s1}{f}\PY{l+s+s1}{\PYZsq{}}\PY{p}{]}\PY{p}{,}\PY{n}{statistics}\PY{p}{[}\PY{l+s+s1}{\PYZsq{}}\PY{l+s+s1}{Mean}\PY{l+s+s1}{\PYZsq{}}\PY{p}{]}\PY{p}{,}\PY{n}{statistics}\PY{p}{[}\PY{l+s+s1}{\PYZsq{}}\PY{l+s+s1}{Worst solution}\PY{l+s+s1}{\PYZsq{}}\PY{p}{]}\PY{p}{[}\PY{l+s+s1}{\PYZsq{}}\PY{l+s+s1}{f}\PY{l+s+s1}{\PYZsq{}}\PY{p}{]}\PY{p}{)}\PY{p}{)}
\end{Verbatim}

    \begin{Verbatim}[commandchars=\\\{\}]
f(x*) = -29912.0 
función objetivo promedio: -29894.8 
función objetivo de la peor solución: -29887.0

    \end{Verbatim}

    \subsubsection{Solución por Indicador y estrategia
ingenua.}\label{soluciuxf3n-por-indicador-y-estrategia-ingenua.}

    \begin{Verbatim}[commandchars=\\\{\}]
{\color{incolor}In [{\color{incolor}29}]:} \PY{n}{args} \PY{o}{=} \PY{p}{(}\PY{n}{getInitialSolution2}\PY{p}{,}\PY{l+m+mf}{1000.0}\PY{p}{,}\PY{l+m+mf}{0.01}\PY{p}{)}
         \PY{n}{statistics} \PY{o}{=} \PY{n}{get\PYZus{}stats}\PY{p}{(}\PY{n}{Knapsack\PYZus{}}\PY{p}{,} \PY{l+m+mi}{30}\PY{p}{,} \PY{n}{args}\PY{p}{)}
\end{Verbatim}

    \begin{Verbatim}[commandchars=\\\{\}]
{\color{incolor}In [{\color{incolor}30}]:} \PY{n+nb}{print}\PY{p}{(}\PY{l+s+s2}{\PYZdq{}}\PY{l+s+s2}{f(x*) = }\PY{l+s+si}{\PYZob{}\PYZcb{}}\PY{l+s+s2}{ }\PY{l+s+se}{\PYZbs{}n}\PY{l+s+s2}{función objetivo promedio: }\PY{l+s+si}{\PYZob{}\PYZcb{}}\PY{l+s+s2}{ }\PY{l+s+se}{\PYZbs{}n}\PY{l+s+s2}{función objetivo de la peor solución: }\PY{l+s+si}{\PYZob{}\PYZcb{}}\PY{l+s+s2}{\PYZdq{}}\PY{o}{.}\PY{n}{format}\PY{p}{(}\PY{n}{statistics}\PY{p}{[}\PY{l+s+s1}{\PYZsq{}}\PY{l+s+s1}{Best solution}\PY{l+s+s1}{\PYZsq{}}\PY{p}{]}\PY{p}{[}\PY{l+s+s1}{\PYZsq{}}\PY{l+s+s1}{f}\PY{l+s+s1}{\PYZsq{}}\PY{p}{]}\PY{p}{,}\PY{n}{statistics}\PY{p}{[}\PY{l+s+s1}{\PYZsq{}}\PY{l+s+s1}{Mean}\PY{l+s+s1}{\PYZsq{}}\PY{p}{]}\PY{p}{,}\PY{n}{statistics}\PY{p}{[}\PY{l+s+s1}{\PYZsq{}}\PY{l+s+s1}{Worst solution}\PY{l+s+s1}{\PYZsq{}}\PY{p}{]}\PY{p}{[}\PY{l+s+s1}{\PYZsq{}}\PY{l+s+s1}{f}\PY{l+s+s1}{\PYZsq{}}\PY{p}{]}\PY{p}{)}\PY{p}{)}
\end{Verbatim}

    \begin{Verbatim}[commandchars=\\\{\}]
f(x*) = -29887.0 
función objetivo promedio: -29887.0 
función objetivo de la peor solución: -29887.0

    \end{Verbatim}

    \subsubsection{Solución ingenua y RI
estrategia.}\label{soluciuxf3n-ingenua-y-ri-estrategia.}

    \begin{Verbatim}[commandchars=\\\{\}]
{\color{incolor}In [{\color{incolor}31}]:} \PY{n}{args} \PY{o}{=} \PY{p}{(}\PY{n}{getInitialSolution}\PY{p}{,}\PY{l+m+mf}{1000.0}\PY{p}{,}\PY{l+m+mf}{0.01}\PY{p}{)}
         \PY{n}{statistics} \PY{o}{=} \PY{n}{get\PYZus{}stats}\PY{p}{(}\PY{n}{Knapsack\PYZus{}2}\PY{p}{,} \PY{l+m+mi}{30}\PY{p}{,} \PY{n}{args}\PY{p}{)}
\end{Verbatim}

    \begin{Verbatim}[commandchars=\\\{\}]
{\color{incolor}In [{\color{incolor}32}]:} \PY{n+nb}{print}\PY{p}{(}\PY{l+s+s2}{\PYZdq{}}\PY{l+s+s2}{f(x*) = }\PY{l+s+si}{\PYZob{}\PYZcb{}}\PY{l+s+s2}{ }\PY{l+s+se}{\PYZbs{}n}\PY{l+s+s2}{función objetivo promedio: }\PY{l+s+si}{\PYZob{}\PYZcb{}}\PY{l+s+s2}{ }\PY{l+s+se}{\PYZbs{}n}\PY{l+s+s2}{función objetivo de la peor solución: }\PY{l+s+si}{\PYZob{}\PYZcb{}}\PY{l+s+s2}{\PYZdq{}}\PY{o}{.}\PY{n}{format}\PY{p}{(}\PY{n}{statistics}\PY{p}{[}\PY{l+s+s1}{\PYZsq{}}\PY{l+s+s1}{Best solution}\PY{l+s+s1}{\PYZsq{}}\PY{p}{]}\PY{p}{[}\PY{l+s+s1}{\PYZsq{}}\PY{l+s+s1}{f}\PY{l+s+s1}{\PYZsq{}}\PY{p}{]}\PY{p}{,}\PY{n}{statistics}\PY{p}{[}\PY{l+s+s1}{\PYZsq{}}\PY{l+s+s1}{Mean}\PY{l+s+s1}{\PYZsq{}}\PY{p}{]}\PY{p}{,}\PY{n}{statistics}\PY{p}{[}\PY{l+s+s1}{\PYZsq{}}\PY{l+s+s1}{Worst solution}\PY{l+s+s1}{\PYZsq{}}\PY{p}{]}\PY{p}{[}\PY{l+s+s1}{\PYZsq{}}\PY{l+s+s1}{f}\PY{l+s+s1}{\PYZsq{}}\PY{p}{]}\PY{p}{)}\PY{p}{)}
\end{Verbatim}

    \begin{Verbatim}[commandchars=\\\{\}]
f(x*) = -29768.0 
función objetivo promedio: -29577.533333333333 
función objetivo de la peor solución: -29331.0

    \end{Verbatim}

    \subsubsection{Solución ingenua y estrategia
ingenua.}\label{soluciuxf3n-ingenua-y-estrategia-ingenua.}

    \begin{Verbatim}[commandchars=\\\{\}]
{\color{incolor}In [{\color{incolor}33}]:} \PY{n}{args} \PY{o}{=} \PY{p}{(}\PY{n}{getInitialSolution}\PY{p}{,}\PY{l+m+mf}{1000.0}\PY{p}{,}\PY{l+m+mf}{0.01}\PY{p}{)}
         \PY{n}{statistics} \PY{o}{=} \PY{n}{get\PYZus{}stats}\PY{p}{(}\PY{n}{Knapsack\PYZus{}}\PY{p}{,} \PY{l+m+mi}{30}\PY{p}{,} \PY{n}{args}\PY{p}{)}
\end{Verbatim}

    \begin{Verbatim}[commandchars=\\\{\}]
{\color{incolor}In [{\color{incolor}34}]:} \PY{n+nb}{print}\PY{p}{(}\PY{l+s+s2}{\PYZdq{}}\PY{l+s+s2}{f(x*) = }\PY{l+s+si}{\PYZob{}\PYZcb{}}\PY{l+s+s2}{ }\PY{l+s+se}{\PYZbs{}n}\PY{l+s+s2}{función objetivo promedio: }\PY{l+s+si}{\PYZob{}\PYZcb{}}\PY{l+s+s2}{ }\PY{l+s+se}{\PYZbs{}n}\PY{l+s+s2}{función objetivo de la peor solución: }\PY{l+s+si}{\PYZob{}\PYZcb{}}\PY{l+s+s2}{\PYZdq{}}\PY{o}{.}\PY{n}{format}\PY{p}{(}\PY{n}{statistics}\PY{p}{[}\PY{l+s+s1}{\PYZsq{}}\PY{l+s+s1}{Best solution}\PY{l+s+s1}{\PYZsq{}}\PY{p}{]}\PY{p}{[}\PY{l+s+s1}{\PYZsq{}}\PY{l+s+s1}{f}\PY{l+s+s1}{\PYZsq{}}\PY{p}{]}\PY{p}{,}\PY{n}{statistics}\PY{p}{[}\PY{l+s+s1}{\PYZsq{}}\PY{l+s+s1}{Mean}\PY{l+s+s1}{\PYZsq{}}\PY{p}{]}\PY{p}{,}\PY{n}{statistics}\PY{p}{[}\PY{l+s+s1}{\PYZsq{}}\PY{l+s+s1}{Worst solution}\PY{l+s+s1}{\PYZsq{}}\PY{p}{]}\PY{p}{[}\PY{l+s+s1}{\PYZsq{}}\PY{l+s+s1}{f}\PY{l+s+s1}{\PYZsq{}}\PY{p}{]}\PY{p}{)}\PY{p}{)}
\end{Verbatim}

    \begin{Verbatim}[commandchars=\\\{\}]
f(x*) = -16269.0 
función objetivo promedio: -15303.1 
función objetivo de la peor solución: -14115.0

    \end{Verbatim}

    \begin{Verbatim}[commandchars=\\\{\}]
{\color{incolor}In [{\color{incolor} }]:} 
\end{Verbatim}


    % Add a bibliography block to the postdoc
    
    
    
    \end{document}
